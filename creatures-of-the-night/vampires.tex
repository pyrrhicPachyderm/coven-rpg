\chapter{Vampires}
\chaplabel{vampires}

\dropcap{Bloodsuckers}.
Night-stalkers.
Burning in daylight; casting no shadow.
Unholy fiends, preying upon the innocent.
{\vampires} are all this, and worse.
But not all of them.

Some skulk in darkness like rats, feeding while their prey sleep.
Some are refined gentlemen, living in towering castles, and inviting young maidens to dinner.
Some are savage beasts, leaving whole villages awash in blood.
Some even try to maintain the life they once had, hiding their curse from friends and family, or seeking help to deal with it.

Although {\vampirism} is a curse of undeath---\discref{necromancy} without any \practitioner{necromancy}---it falls halfway between mere reanimation and true resurrection.
In truth, {\vampires} are a kind of \undeadref{souled}.
They are dead, but they retain their memories and personalities from life.
Even their souls are returned to their bodies.

Befitting {\vampirismpossessive} nature as a curse, however, the process is flawed.
The lack of a reflection betrays it the most.
The {\vampirepossessive} soul is twisted in some indefinable way.
It is hollow, lacking substance.

A {\vampire} has a hard time feeling fulfilled.
The bloodthirst is the most obvious manifestation, but it subtly penetrates all aspects of the {\vampirepossessive} being.
{\vampires} are often greedy, jealous, or depressed.
However much they acquire in their eternal lives, it will never quite feel like enough.
They will never feel content.
It's a steep price to pay for immortality.

\section{Vampire Traits}

{\vampirism} is generally the more alluring of the Night Curses.
It isn't \emph{all} drawbacks; at the very least, it offers immortality.
Those traits that are specifically weaknesses are described in the next section, while this section describes more general traits.

\section{Vampire Weaknesses}

Mythology is rife with tales about the weaknesses of {\vampires}.
Many of them are false, but, unfortunately for the {\vampires}, most are actually true.
The worst of these are burning in sunlight, and the inability to enter a dwelling without an invitation, but there are also several other annoyances that can quickly prove fatal.

Many of the purported weaknesses of {\vampires}---fire, or a stake to the heart---are a little unremarkable in that they will kill non-{\vampires} just as well.
There is still an element of truth in these tales however.
Powerful {\vampires} can develop a resistance to more conventional methods of murder, but these methods might still work against them.
These weaknesses are not listed here, as they are irrelevant to a newly-cursed {\vampire}, but are detailed in the feats that make them relevant.

\subsection{Sunlight}



\subsection{Invitation}

A {\vampire} cannot enter a dwelling, or other building, without an invitation.
A physical barrier seems to exist at the threshold for her; she can even be thrown against it.
Furthermore, she is prevented from influencing anything, or anyone, inside in the building through most mundane or magical means.
Items she throws, or arrows that she fires will bounce off the same barrier that she would, unable to enter the building or to damage it.

Nor can any magic she works affect anything inside: wind she is \discref{willing} stops at the barrier; her {\symlinks} transmit nothing while one end is inside; she cannot place a \featref{scrying} sensor inside; she cannot even work \discref{headology} upon the inhabitants.
The barrier even stretches into the {\mentalrealm}.
Her influence is essentially limited to mundane communication; the barrier does not block sound or light, so she can talk to the inhabitants as normal.
The {\vampirepossessive} familiar, golems she has animated, and undead under her control suffer from the same limitations that she does.
An \featref{animal-companion} is its own creature, however, and is unimpeded.
A {\vampirepossessive} brew also continues to work if taken inside; she is working no magic upon it after its creation.

For a {\vampire} to enter, the invitation must come from someone who would typically have some right to offer an invitation into the building.
The homeowner will certainly suffice---as will other invited guests, in many situations.
Someone who has, themselves, entered without an invitation certainly does not count.
Interestingly, this leaves some buildings without anyone who could validly invite the {\vampire} in.
For example, the owner of a tomb is typically too dead to offer an invitation.
Consecrated ground requires the permission of the god to whom it is dedicated, and gods aren't know for being talkative.
Consecrated ground is a special case in that a {\vampire} cannot cross it even if no building stands on it.

The invitation need not be verbally explicit---someone holding the door open, and standing aside to let you in will suffice---but it must be communicated to the {\vampire} in some fashion.
The invitation remains valid until it is revoked.
It can be revoked by anyone with the power to grant a valid invitation; it need not be the person who granted the invitation in the first place.
An invitation is also revoked if the person who granted it would no longer be able to do so; if they no longer own the house, for example, or if their own invitation into the building is revoked.

If the {\vampire} is inside the building when her invitation is revoked, she may remain there as long as she likes, but cannot re-enter once she leaves.
As long as she remains in the building, she is not subject to any of the other limitations on affecting the inhabitants.
Her familiar, golems, and undead cannot re-enter, however, even if she remains inside.

Despite all the rules, there will still come a time where it is ambiguous whether a {\vampire} can enter a building or not.
Is this construction really a building?
Is this person allowed to offer invitations?
Was that gesture really an invitation?
This limitation exists largely in a {\vampirepossessive} mind, so resolving ambiguous cases is largely a matter of convincing herself that her invitation is valid.
Treat this similarly to the rules for improvising magical equipment: have the player provide a justification, use it to set a TN, and call for a \attref{will} Test.

\subsection{Reflection}

A {\vampire} cannot be seen indirectly; she doesn't appear in mirrors, she doesn't cast a shadow, and she can't be seen by \featref{scrying}.
This may seem like an advantage, but the fact that her clothes still appear means it does little more than give away her {\vampiric} nature.

\subsection{Garlic}

A {\vampires} is dangerously allergic to \herb{garlic}{2}.
Against the skin, it merely burns a little, dealing a \dice{1} {\damagetest}.
Biting into a clove, however, deals a \dice{5} {\damagetest}.
This {\damagetest} sends the {\vampire} into {\shock} as normal; it doesn't cause immediate death, like sunlight.
If she swallows any garlic, however, it continues to burn inside her.
This is typically lethal.

\subsection{Running Water}


