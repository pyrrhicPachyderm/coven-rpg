\chapter{The Night Curses}
\chaplabel{introduction-creatures-of-the-night}

\dropcap{Midnight}.
The witching hour, many say.
But when the sun goes down, and the moon rises, there are far more terrible things abroad than witches.

Rabid beasts with wicked claws, howling their anger at the full moon.
Dead men walking, stalking their prey by night, drinking the blood of the innocent.
Victims of the Night Curses.
{\werewolves} and {\vampires}.

The two curses---{\lycanthropy} and {\vampirism}---are, truthfully, unrelated.
In fact, while {\lycanthropy} only afflicts the living, {\vampires} are a variety of undead.
But the two are often treated together, as a result of their similarities.
Both creatures emerge at night: {\vampires} to avoid the sun, and {\werewolves} as they are changed by the moon.
Both can turn innocent, friendly people into bloodthirsty monsters.
And both curses are spread, more or less, by a bite.

\section{A Mixed Curse}

Some people might find the idea of {\lycanthropy} or {\vampirism} to be appealing.
A transformed {\werewolf} is a mighty beast, capable of taking on several armed men.
And a {\vampire} is immortal, without any of the drawbacks of being a \undeadref{skeleton}.

But it is important to remember that they are still curses.
The drawbacks almost \emph{always} outweigh the benefits.
A transformed {\werewolf} loses its mind---it becomes worse than a mere animal; bloodthirsty, almost rabid.
A {\vampire} arguably has it worse.
At least a {\werewolfpossessive} curse only afflicts it once a month; a {\vampire} can never venture outside during the daytime.
And sunlight is far from their only weakness: garlic, running water, wooden stakes, even the inability to enter a place uninvited.

Those who rail against the curses tend to find that they grow quickly worse.
A {\werewolf} can't resist the urge to transform under a full moon for any longer than it can resist the need to breathe.
Those who try often find themselves transforming unprepared, and might slaughter their whole family, or village, before the sun rises again.
A {\vampire} who tries to resist the bloodthirst will inevitably find itself driven insane, to similarly murderous effect.
There are tales of a few who tamed the beast inside for long enough to find a cure, but they are far outnumbered by the tales of those who failed.

Those who embrace the curse, however, often find that they can make it work \emph{for} them.
They can derive strength from it, ameliorate its weakness, and even derive entirely new abilities from its power.
They will always suffer the worst of its drawbacks---these are unavoidable---but they can soon come to live with it.
They may even, eventually, come to count it as a blessing.

\section{Using the Night Curses}



\section{Curing the Curses}


