\chapter{The Night Curses}
\chaplabel{introduction-creatures-of-the-night}

\dropcap{Midnight}.
The witching hour, many say.
But when the sun goes down, and the moon rises, there are far more terrible things abroad than witches.

Rabid beasts with wicked claws, howling their anger at the full moon.
Dead men walking, stalking their prey by night, drinking the blood of the innocent.
Victims of the Night Curses.
\capital{\werewolves} and {\vampires}.

The two curses---{\lycanthropy} and {\vampirism}---are, truthfully, unrelated.
In fact, while {\lycanthropy} only afflicts the living, {\vampires} are a variety of undead.
But the two are often treated together, as a result of their similarities.
Both creatures emerge at night: {\vampires} to avoid the sun, and {\werewolves} as they are changed by the moon.
Both can turn innocent, friendly people into bloodthirsty monsters.
And both curses are spread, more or less, by a bite.

\section{A Mixed Curse}

Some people might find the idea of {\lycanthropy} or {\vampirism} to be appealing.
A transformed {\werewolf} is a mighty beast, capable of taking on several armed men.
And a {\vampire} is immortal, without any of the drawbacks of being a \undeadref{skeleton}.

But it is important to remember that they are still curses.
The drawbacks almost \emph{always} outweigh the benefits.
A transformed {\werewolf} loses its mind---it becomes worse than a mere animal; bloodthirsty, almost rabid.
A {\vampire} arguably has it worse.
At least a {\werewolfpossessive} curse only afflicts it once a month; a {\vampire} can never venture outside during the daytime.
And sunlight is far from their only weakness: garlic, running water, wooden stakes, even the inability to enter a place uninvited.

Those who rail against the curses tend to find that they grow quickly worse.
A {\werewolf} can't resist the urge to transform under a full moon for any longer than it can resist the need to breathe.
Those who try often find themselves transforming unprepared, and might slaughter their whole family, or village, before the sun rises again.
A {\vampire} who tries to resist the bloodthirst will inevitably find itself driven insane, to similarly murderous effect.
There are tales of a few who tamed the beast inside for long enough to find a cure, but they are far outnumbered by the tales of those who failed.

Those who embrace the curse, however, often find that they can make it work \emph{for} them.
They can derive strength from it, ameliorate its weakness, and even derive entirely new abilities from its power.
They will always suffer the worst of its drawbacks---these are unavoidable---but they can soon come to live with it.
They may even, eventually, come to count it as a blessing.

\section{Using the Night Curses}

Frankly, this would be a rather boring expansion book if it added nothing but two new kinds of monster for the player characters to run up against.
You can certainly use {\vampires} and {\werewolves} this way; either as antagonists, or as tragic victims in need of the protagonists' help.
However, becoming a {\vampire} or {\werewolf} is also designed to be a valid option for a player character.

\capital{\lycanthropy} and {\vampirism} are primarily curses, so they will, on the whole, hinder a character.
As such, there is no XP cost for becoming a {\werewolf} or {\vampire}.
Players may choose, with GM permission, for their witches to already be cursed at character creation as part of their backstory.
They may also be cursed during play, willingly or unwillingly, by another {\vampire} or {\werewolf}.
The means by which the curses are transmitted are detailed in \chapref{werewolves} and \chapref{vampires}.

Note that this willingness or unwillingness refers to the \emph{character}.
Inflicting {\lycanthropy} or {\vampirism} on a character against the player's will can lead to discontent.
\capital{\lycanthropy} is not so bad---the curse only comes around once a month, and can be handled with the help of the coven---but {\vampirism} is more potent, and can ruin a character.
With an enthusiastic player, however, a character contracting one of the Night Curses can make for a potent piece of plot.

A character afflicted with a Night Curse has a few options.
They may simply try to live with it, perhaps employing the help of the coven in avoiding its most terrible consequences.
They may seek a cure, and advice to those doing so is presented in the following section.
Or they may embrace the curse, working to tame and enhance its powers, while ameliorating its greatest weakness.
To this end, \chapref{werewolves} and \chapref{vampires} present a variety of feats, available to {\werewolves} and {\vampires} respectively.
These feats may be taken as normal by a character with the appropriate curse, and do not count as part of any discipline of magic.

These feats are the main reason, besides providing drama and plot hooks, why a character would choose for their witch to be afflicted with a Night Curse at character creation.
Being cursed in such a way always requires express permission from the GM, however.
The GM should always consider such requests carefully, and shouldn't feel guilty about declining them.

Player characters with a Night Curse certainly do not fit the tone of every game.
In fact, such a character could very easily come to define the tone of the game, as it revolves around their struggles to live with the curse, or to cure it.
There's nothing wrong with such a tone, as long as it doesn't steal the spotlight from the other characters.
In the extreme, a game where every member of the coven is subject to one Night Curse or the other could certainly prove fun.
But a game without any Night-Cursed characters at all should probably still be the norm.

In general, a {\vampire} character will have a more profound effect upon the game than a {\werewolf} character.
A {\vampirepossessive} weakness are far more numerous, and far more prevalent, than silver and full moons.
Similarly, the {\vampire} feats provide far more widespread benefits; a {\werewolf} will be almost entirely normal as long as they remain in human form.

\section{Curing the Curses}



%TODO: What happens to feats that require a curse when you cure it?
