\chapter{Familiars}
\chaplabel{familiars-expansion}

\section{Familiar Animals}

\familiar{Badger}{Badgers}{badger}{15}{
	\atttable{0}{1}{1}{1}{3}{1}{\negative 1}{1}
}{
	\speed{8}
}{
	\skillrefspeciality[2]{crafting}{Earthworker}, \skillref[1]{intimidation}
}{
	\familiarrefplural{badger} are striped, stocky, burrowing omnivores.
	They are normally docile, but have large claws, a strong bite, and a vicious streak when cornered.
	A \familiarrefpossessive{badger} network of underground tunnels, called a sett, can stretch for a mile or more.
	Witches who take \familiarrefplural{badger} for familiars are often reluctant to leave their territory, but will defend it to the death.
}{
	\ability{Bite \& Claws}{
		The \familiarref{badger} rolls 4 dice for \weaponref{unarmed} {\damagetests}.
	}
	
	\ability{Snuffling \& Rooting}{
		The \familiarref{badger} rolls an extra die on \skillref{perception} Tests relying on smell.
		It rolls a second extra die if the Test is to detect something buried or underground.
	}
	
	\ability{Burrowing}{
		The \familiarref{badger} can burrow through about a metre of loose earth in 2 minutes, or a metre of packed earth in 5 minutes.
		It leaves a tunnel behind it.
	}
	
	\familiaroption{Honey Badger}{10}{
		The \familiarref{badger} rolls 5 dice for \weaponref{unarmed} {\damagetests}.
		It increases its \statref{st} by 2.
	}
}

\familiar{Beehive}{Beehives}{bees}{40}{
	\atttable{\negative 5}{1}{3}{\negative 1}{3}{1}{\negative 1}{2}
}{
	\flyspeed{6}
}{
	\skillref[1]{botany}, \skillref[1]{flying}, \skillrefspeciality[1]{performance}{Dancer}, \skillref[1]{weaponry}
}{
	Bees are more than just social creatures.
	A bee, such as it is, barely has a mind at all.
	Only the \familiarref{bees}, as a whole, can be considered to have a real mind.
	As such, a witch does not take a bee as a familiar, but a \familiarref{bees}.
	
	Taming a \familiarref{bees} is no easy task, and the ritual to bind one as a familiar is further complicated by the distributed mind.
	Such a binding is an impressive feat, and a witch who has managed it can command a lot of respect from those who recognise this.
}{
	\ability{Hive}{
		While the hive stands and bees reside within, the swarm is not dead.
		Swarms of bees can leave the hive, though they cannot be away from the hive for more than a day.
		The loss of the swarm does not kill the \familiarref{bees}, although a \familiarref{bees} that loses many swarms in quick succession will not be able to provide more.
	}
	
	\ability{Swarm}{
		Being composed of many individuals, a swarm does not suffer injury in the same way most creatures.
		It is not subject to {\shock}, and is destroyed only when it has taken 15 {\damage}.
		However, the swarm grows depleted as it loses bees, and suffers a \negative{1} penalty to all Tests (including {\damagetests}) for every 3 {\damage} it has taken.
		A swarm can be healed only by being replenished from the hive.
	}
	
	\ability{Sting}{
		A swarm rolls 4 dice for \weaponref{unarmed} {\damagetests}, which are not affected by the swarm's \attref{might}.
		This {\damage} is dealt by injected venom.
		A successful attack by the swarm also deals 3 {\damage} to the swarm itself, as bees are killed by stinging.
	}
}

\familiar[Dragonfly/Damselfly]{Dragonfly}{Dragonflies}{dragonfly}{10}{
	\atttable{\negative 5}{3}{0}{0}{1}{1}{0}{\negative 1}
}{
	\flyspeed{15}
}{
	\skillref[2]{flying}, \skillref[1]{stealth}, \skillref[2]{weaponry}
}{
	\familiarrefplural{dragonfly}---and their cousins the damselflies---live for several years under the water, before they grow their wings and emerge.
	This adult stage only lasts a few months, and it is in this time that a witch must capture it and bind it as a familiar.
	It is a quick and nimble flier, and a voracious predator.
	Albeit only of other insects.
}{
	\ability{Tiny Predator}{
		The \familiarref{dragonfly} cannot effectively attack anything much larger than itself, but rolls 4 dice for \weaponref{unarmed} {\damagetests} when picking on something its own size.
	}
	
	\ability{Dartwing}{
		The \familiarrefpossessive{dragonfly} \statref{dr} is increased by 2.
	}
}

\familiar{Duck}{Ducks}{duck}{5}{
	\atttable{\negative 2}{1}{1}{2}{2}{1}{1}{0}
}{
	\speed{4}, \swimspeed{4}, \flyspeed{15}
}{
	\skillref[1]{flying}
}{
	Due to a duckling's tendency to imprint on humans, it is really quite easy for a witch to tame one and bind it as a familiar.
	Some \familiarrefplural{duck} find themselves as familiars within a day of hatching.
}{
	\ability{Waterfowl}{
		The \familiarref{duck} can not only swim, but can take off from the water's surface.
	}
}

\familiar[Hamster/Gerbil/Guinea Pig]{Hamster}{Hamsters}{hamster}{0}{
	\atttable{\negative 3}{1}{1}{1}{2}{1}{2}{0}
}{
	\speed{6}
}{
	\skillref[1]{socialising}
}{
	\familiarrefplural{hamster} and their ilk make popular pocket pets, and hence prime candidates for familiars.
	Certainly less repulsive than a \familiarref{rat}, a \familiarref{hamster} can even be disarmingly cute.
}{
	\ability{Keen Smell}{
		The \familiarref{hamster} rolls an extra die on \skillref{perception} Tests relying on smell.
	}
}

\familiar[Hedgehog/Porcupine]{Hedgehog}{Hedgehogs}{hedgehog}{0}{
	\atttable{\negative 3}{0}{1}{1}{3}{1}{1}{1}
}{
	\speed{6}
}{
	\noskills
}{
	\familiarrefplural{hedgehog} have prickly exteriors, but they're softies on the inside.
	Witches who take \familiarref{hedgehog} familiars are often much the same.
	A \familiarref{hedgehog} can curl into a ball for defence, stabbing an attacker with its spines.
}{
	\ability{Keen Smell}{
		The \familiarref{hedgehog} rolls an extra die on \skillref{perception} Tests relying on smell.
	}
	
	\ability{Spiny}{
		Any creature which hits the \familiarref{hedgehog} with an \weaponref{unarmed} \actionref{attack} suffers a \dice{2} {\damagetest}.
	}
	
	\ability{Curl}{
		The \familiarref{hedgehog} may curl into a ball on its {\turn}.
		It cannot move or take an {\action} that {\turn}, but gains \positive{1} \statref{res} until the start of its next {\turn}.
	}
}

\familiar{Serpent}{Serpents}{serpent}{15}{
	\atttable{\negative 1}{1}{1}{2}{2}{2}{2}{1}
}{
	\speed{6}
}{
	\skillref[2]{deception}, \skillref[1]{intimidation}, \skillref[1]{perception}, \skillref[1]{weaponry}
}{
	Silver-tongued, slithering and and sly, a \familiarref{serpent} is a favourite among some of the nastier witches.
}{
	\ability{Forked Tongue}{
		The \familiarref{serpent} rolls an extra die on \skillref{perception} Tests relying on smell or taste.
	}
	
	\ability{Bite}{
		The \familiarref{serpent} rolls 3 dice for \weaponref{unarmed} {\damagetests}.
	}
	
	\familiaroption{Viper}{10}{
		The \familiarref{serpent} can inject venom with its bite.
		If it does so, it rolls 5 dice for the {\damagetest}.
		The target must succeed on a {\tn} 15 \attref{might} Test or suffer paralysis over the next 5 minutes.
		Death often ensues, without medical attention.
	}
	
	\familiaroption{Constrictor}{10}{
		The \familiarref{serpent} gains a \attref{might} score of 1 (instead of \negative{1}), \skillref[1]{athletics}, and rolls an extra die on Tests to entangle or restrain creatures.
		It rolls 4 dice for \weaponref{unarmed} {\damagetests}, and may roll such a {\damagetest} without first making a Test to hit when it makes an \actionref{attack} against a target it has entangled.
		%TODO: Tie this into grappling rules, if they exist?
	}
}
