\chapter{Body, Mind \& Soul}
\chaplabel{body-mind-and-soul}

\dropcap{Witches} know that a person is not indivisible.
People, animals, and even plants are made up of three parts: the body, the {\mind}, and the {\soul}.
In the normal course of things, the three will never be parted until death.
But with magic, they may be split apart.
This chapter details the relationship between the three.

The distinction between body, {\mind}, and {\soul} is of the most relevance to \practitioners{projection} and \practitioners{necromancy}.
\capital{\discref{projection}} is the art of separating your {\mind} from your body, and \discref{necromancy} is fundamentally about the manipulation of {\souls}.

\section{The Body}
\seclabel{the-body}

A person's body is the most obvious part of them.
It's the only part you can see, touch, hear, or smell.
It includes all their organs, even their brain.

A body serves two important purposes.
Firstly, for most people, it is the only way they can interact with the world.
It contains their eyes and ears, by which they sense the world, and their hands and feet, by which they affect it.
Some witches learn to affect the world with their {\minds}, but for everyone else, they are powerless if divorced from their body.

Secondly, the body's vitality anchors a person's {\soul}.
As long as a person's heart beats and their brain whirrs, their {\soul} remains bound within their body, anchored to this world.
But when the heart and brain stop, the {\soul} escapes this world, passing beyond the veil and taking the {\mind} without it.
The body remains behind, now reduced to a lifeless corpse.

\section{The Mind}
\seclabel{the-mind}

A person's {\mind} is, arguably, the part that makes them \emph{them}.
It contains their personality, their skills, and all their memories.
It is the part that does all their thinking, makes their decisions, and directs their body in its actions.
When the {\mind} departs the body, the body is left without will or direction, and simply slumps, motionless.

When a player character's {\mind} is separated from their body, the player directs the character's {\mind}, for the body is left without any agency.
Thus, players would be wise to ensure their character's bodies are well protected before leaving them somewhere.

%TODO: Go and revising the wording in the "in absentia" feats to reflect this explanation.

The {\mind} is generated by the {\soul}.
Thus, without the {\soul}, there is no {\mind}, no personality, no \emph{person}.
When the body dies and the {\soul} departs beyond the veil, the {\mind} goes with it, taking the person from this world.

The {\mind} is less strongly tied to the {\soul} than the {\soul} is to the body.
A \practitioner{projection} can project their {\mind} away from their body and {\soul}, to wander the {\mentalrealm}, and perhaps even to {\possess} other bodies.
However, should their original body die and their {\soul} depart, their {\mind} is ripped from this world to go with it, no matter where their {\mind} currently is.

\section{The Soul}
\seclabel{the-soul}

The {\soul} is the least tangible part of a person.
It has no physical substance, nor does it have the self-consciousness that demonstrates the existence of a {\mind}.
Only {\ghosts}, and the workings of \practitioners{necromancy}, betray its existence at all.
Yet it is extremely important, for without a {\soul}, there can be no {\mind}.

The {\soul} is anchored to the body, and generates the {\mind}.
Thus, it is also the linchpin that binds the two together.
Within the normal course of someone's life, their {\soul} never leaves their body, even if their {\mind} wanders elsewhere.
This is true until the moment of death, at which point the {\soul} departs beyond the veil.
However, some {\souls} may linger after the death of the body, becoming {\ghosts}.
And some \discref{necromancy} may also interfere with the usual fate of the {\soul}.

\section{Statistics}

When a {\mind} is separated from a body, or a body is {\possessed} by another {\mind}, it may be necessary to calculate the statistics of the resulting entity.
\capital{\attref{might}}, \attref{grace}, \statref{resilience}, and \statref{speed} are properties of the body.
The other six attributes, all skills, and all feats are properties of the {\mind}.
The traits of familiars and other creatures usually properties of the body, but a few would more reasonably be considered to be properties of the {\mind}, or even a combination of the two; the GM should use their best judgement to decide.
\capital{\damage} is a property of the body, as are any ongoing effects of brews consumed by the body.

When a new {\mind} enters a body, simply combine the statistics of the body and the {\mind} to get the statistics of the resulting entity.
\capital{\statref{shock-threshold}} and \statref{dodge-rating} are a product of the combination of the body and {\mind}: recalculate them using the \attref{might} and \attref{grace} of the body, and the \attref{will} and \attref{ken} of the {\mind}, with the \statref{shock-threshold} reduced by any {\damage} sustained by the body.

A {\mind} without a body, loose in the {\mentalrealm}, does not have a \attref{might}, \attref{grace}, or \statref{speed}.
None of these statistics would be relevant in the {\mentalrealm} regardless: a {\mind} cannot affect the physical world without a body, and movement in the {\mentalrealm} is done by thinking, rather than walking or running.

A body without an occupying {\mind} does not have any attributes other than \attref{might} or \attref{grace}.
However, even these attributes are of reduced relevance, as it lies unmoving without an occupying {\mind}.
It cannot even stand or sit up on its own, and slumps wherever it is left.
An unoccupied body has no \statref{dodge-rating}; attacks automatically hit it.
An unoccupied body still has a \statref{shock-threshold}, which is calculated as though it had 0 \attref{might}.
