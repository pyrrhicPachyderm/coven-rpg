\discipline{Necromancy}{necromancy}{Necromancer}{Necromancers}

\section{Reanimation and Resurrection}

Many \practitioners{necromancy} draw a distinction between reanimation and resurrection.
Reanimation is a crude process, somewhat akin to \discref{golemancy}.
It's nothing more than the application of raw animating force to a corpse, to stand it up and get it shuffling around again.
The creature retains its instincts, its muscle memory and the like, but that's as much through what is left of its biology as it is through the will that animated it.
The results of reanimation are known as the undead.

Resurrection, by contrast, brings a creature back back to life in full, and the creature's {\soul} is restored to its body.
There may be a few ill effects of the process, not to mention whatever killed it in the first place, but these can typically be recovered from.
For all intents and purposes, the creature is just as much alive as it was in the first place.

In theory, at least.

The trouble is that nobody has ever achieved true resurrection.
Dozens of witches and hundreds of charlatans have all claimed to.
Many have even come incredibly close, but there has always been some slight snag.
The search, of course, continues, but many have given up all hope that it is possible.

\section{The Undead}

Undead, the products of reanimation, come in many different forms: \undeadrefplural{zombie}, \undeadrefplural{skeleton}, \undeadrefplural{ghoul}, and more.
The statistics and capabilities of an undead creature are ultimately based upon those of the creature whose corpse is raised to create it.
However, each kind of undead comes with its own modifications to those statistics.
The rules detailing these modifications, for undead in general and for each specific type, can be found in \chapref{undead}.

\subsection{Reanimation Rituals}
\seclabel{reanimation-rituals}

For every kind of undead, there is a different ritual required to animate it.
However, all these rituals share mark{\'e}d similarities.
And many \practitioners{necromancy}, who consider these rituals to form the heart of the discipline, learn a lot of techniques for improving them.

For convenience, the feats which modify these rituals refer to them collectively as the {\reanimationrituals}, and the feats which provide them are listed here.
\begin{itemize}
	\item \capital{\featref{animate-zombie}}
	\item \capital{\featref{animate-skeleton}}
	\item \capital{\featref{animate-ghoul}}
	\item \capital{\featref{animate-draugr}}
	\item \capital{\featref{animate-sea-draugr}}
	\item \capital{\featref{animate-fossil}}
	\item \capital{\featref{animate-fire-skeleton}}
	\item \capital{\featref{animate-shade}}
	\item \capital{\featref{animate-wraith}}
\end{itemize}
Obviously, you cannot perform any variant of a ritual unless you have the feat to perform that ritual in the first place.

\subsection{Controlling the Undead}

When you reanimate a corpse, the resulting undead is under your control.
You control it mentally, without need for verbal instructions or gestures.
However, you can only maintain control while you remain conscious, and nearby---at least near enough to see or otherwise sense it.
If you fall asleep, fall unconscious, die, get too far away from the creature, or leave your body for the {\mentalrealm}, you lose control over it.
A novice \practitioner{necromancy} has no way to regain control over an undead creature once she loses it, other than to kill the creature and reanimate the corpse again.

When you lose control over an undead, it regains free will.
It begins to act as an animal of its kind normally would, seeking out whichever food it would normally eat.
However, it is eternally, ravenously, and insatiably hungry.
It's generally considered good practice to put an undead down, rather than to lose control of it.

A witch who can raise an undead creature learns to control one, and only one, at a time.
When you reanimate a second creature, you must either relinquish control of the previous one, or immediately lose control of the new one.

\section{Embalming}
\seclabel{embalming}

Decomposition can be such a pain for a \practitioner{necromancy}, putting valuable corpses to waste.
Most corpses barely last more than a week before they are too rotted to make some kinds of undead, such as \undeadrefplural{zombie} and \undeadrefplural{ghoul}.
A \practitioner{necromancy} can always strip away the flesh and raise \undeadrefplural{skeleton}, but these might not suit her needs.
Instead, she might turn to {\embalming}.

Anyone can attempt to {\embalm} a corpse.
The process is a mixture of surgery, and treatment with substances that slow decomposition.
Various substances can be used, with varying effectiveness.
Soaking a corpse in strong alcohol can preserve it for a month or more.
Drying it with salt can preserve it indefinitely, as long as it is not wetted again.
Very long periods of preservation can be achieved with dedicated \featref{embalming-fluid}.

\capital{\embalming} a corpse typically takes a few hours, and requires a \testtypespeciality{ken}{crafting}{Embalmer} {\test}.
Failure means that the corpse, or some parts of it, won't be preserved, or at least won't last as long as they could.
Particularly bad results can cause {\damage} to corpse.

\capital{\embalming} does nothing to repair {\damage} to the corpse, or to reduce rotting that has already occurred.
It only slows or prevents further rotting.

\section{Souls}
\seclabel{souls}

\capital{\discref{necromancy}} deals with death, after the {\soul} has been separated from the body.
Reanimation uses only the body, while resurrection seeks to reunite body and {\soul}.
Another branch of \discref{necromancy}, however, deals with bare {\souls}---{\ghosts}.

Every living creature has a {\soul}, of some sort.
The more intelligent, reasoning, and self-aware the creature, the stronger the soul.
Truly sapient creatures---humans---are referred to as {\strongsouled}.
Other animals are {\weaksouled}.
The {\souls} of plants, fungi, are the like as so weak as not to exist for any practical purpose---they do not even pass as {\weaksouled}.

Familiars are a particularly unusual case.
They share a shard of their witch's {\soul}, and so are considered to be {\strongsouled} for most purposes.
But some magics affect them strangely, due to their shared soul.

The strength of a {\soul} is a double-edged sword to a \practitioner{necromancy}.
\capital{\strongsouls} are much more substantial and easier to work with for a \practitioner{necromancy} using {\ghosts}.
But they also maintain a link to their own body, brought about by their self awareness.
This makes any body that bore a {\strongsoul} in life much harder to reanimate---the {\soul} must be severed from it first, or it will interfere with the animating energies.

During life, the {\soul} remains in the body.
And at the point of death, it departs the body, and passes through the veil to the afterlife.
Unless something goes wrong.
Sometimes, a {\soul} becomes stuck on this side of the veil, trapped in the mortal realm as a {\ghost}.
\capital{\practitioners{necromancy}} can facilitate this, preventing a {\soul} from passing over, or even reaching through the veil to pull a {\soul} back.

One point of note is that it is the departure of a witch's {\soul} from this realm that kills her familiar.
If a witch's {\soul} remains in this realm, as a {\ghost} or in a {\phylactery}, her familiar can continue to survive.

\subsection{Ghosts}
\seclabel{ghosts}

\capital{\ghosts} are {\souls} that have been unable to cross the veil to the afterlife, and remain trapped in the mortal realm.
Some are the intentional creation of \practitioners{necromancy}, magically bound to this realm.
But some also occur naturally.

When a creature, especially a {\strongsouled} creature, dies with unfinished business, it may remain in this realm as a {\ghost}.
The more important the unfinished business---often a matter of vengeance---the more likely a {\ghost} is to form, and the more powerful it is likely to be.
It will remain as long as the business that trapped it remains unfinished.
When the business is finally resolved, it can cross over the veil to the afterlife.
The GM can use these natural {\ghosts} in creating stories, whether or not there is a \practitioner{necromancy} in the coven.

As a creature's {\soul}, a {\ghost} retains most of the capabilities it had in life.
It keeps all attributes and skills.
It is still the same person it was in life---it retains its personality and all its memories, and has total free will.
Even {\ghosts} created by \practitioners{necromancy} are not bound to the \practitionerpossessive{necromancy} will.
However, a {\ghost} that has been drawn back across the veil has no memories of the afterlife, beyond a vague recollection that it is more pleasant than the mortal realm.
It is aware that time has passed, but doesn't know how long.

Despite retaining its skills, personality, and free will, a {\ghost} loses almost all ability to affect the world.
It cannot lift, carry, or move anything; it cannot exert any force on any objects or creatures.
To most people, it is invisible and inaudible, unable to be sensed---it doesn't even appear in the {\mentalrealm}.
People might be able to feel some slight presence, but often won't even realise it represents a {\ghost}.

A few people do learn to sense {\ghosts}, and a few {\ghosts} learn to be sensed by everybody.
To those people who can see ghosts, they appear much as they did in life, except partially transparent, and in muted colours.
However, they tend to appear as they saw themselves, not as how they actually were.
As such, human {\ghosts} tend to appear clothed, wearing their favourite clothes, or the clothes they believed typified them.
It is this need for self-conception that causes {\strongsouls} to form {\ghosts} far more often than {\weaksouls}.

Much as {\ghosts} can't affect objects, they aren't affected by most objects.
They can walk through walls, and pass through other objects and creatures---except other {\ghosts}.
They will not pass through the ground or floor, though, even if they want to, and must still walk upon such a surface.
In water, they can swim much as they did in life.
And, although they can pass through walls, magical barriers like the \featref{circle-contain} will still impede them.

\capital{\ghosts} are also immune to almost everything.
They cannot be harmed in any fashion---they do not suffer {\damage} or {\shock}.
They do not need to eat, breathe, drink, or sleep.
They cannot be affected by any potions, poisons, diseases, or the like.
They still suffer {\exhaustion}, although they will never pass out from suffering too much.
They recover from {\exhaustion} as normal, except that they do not need to sleep, just to rest.

The inability to affect the world also means that {\ghosts} of witches lose most ability to work magic.
They lose the benefit of all feats, for all intents and purposes.
They have not lost the knowledge required, but the world simply refuses to respond to their will.

Note that the above rules describe a typical {\ghost}: the weakest kind, the kind a novice \practitioner{necromancy} might create.
Some {\ghosts} become more powerful, through help from a \practitioner{necromancy}, from sheer effort of will, or from learning to affect the world once again.
The GM may add other abilities to naturally-occurring {\ghosts} as they see fit, possibly using the {\ghost} related feats in this chapter as inspiration.

\subsection{Phylacteries}
\seclabel{phylacteries}

Creating a {\ghost} is not the only way to bind a {\soul} to this realm.
All told, it's actually a rather messy way, with the {\soul} still running around inconveniently.
It's far neater, some \practitioners{necromancy} say, to seal the {\soul} inside a clay jar.
This creates a {\phylactery}.

A {\phylactery} is an object containing a person or other creature's {\soul}.
\capital{\phylacteries} must be created from a clay jar, at least the size of a fist but possibly larger.
They are no more robust than the jars they are created from, and their destruction frees the {\soul} within, letting it pass into the afterlife as if the person had just died.
Each person only has one {\soul}, so can only inhabit one {\phylactery}.
However, nothing prevents one jar serving as a {\phylactery} to multiple people.

Being trapped in a {\phylactery} is a kind of hell.
The {\soul} remains conscious, and aware of the passage of time.
However, it experiences no sensations of any kind, and cannot act in any fashion.
It has nothing but its own thoughts for company---a state which can drive a person mad before all too long.

\capital{\practitioners{projection}} can still form {\interfaces} to the mind of a {\phylacterypossessive} inhabitant, perhaps providing some occasional relief from the isolation, but it cannot perform any \discref{projection} of its own.

\section{Feats}
\setfeatdefault{discipline}{necromancy}{false}
\unsetfeatdefault{tags}

\feat{Raise Zombie}{animate-zombie}{20}{
	\noprereq
}{
	You can restore a terrible facsimile of life to the body of a deceased animal, reanimating it as a \undeadref{zombie}.
	For now, you are limited to animals at least as large as a mouse, and no larger than a medium-size dog such as a bloodhound.
	You can't manage any creature that was {\strongsouled} in life, due to interference from the link that remains.
	
	\materials{An animal corpse, a \circleref{small}, a lit candle which the ritual extinguishes}
	
	The reanimation ritual takes five minutes, and must be performed in the dark.
}

\feat{Raise Skeleton}{animate-skeleton}{15}{
	\featref{animate-zombie}
}{
	After a few reanimations, most \undeadrefplural{zombie} are starting to come apart at the seams a bit.
	There comes a time when it's easier just to strip all the flesh off and make the bones stand up by themselves.
	You may reanimate the bones of an animal corpse as a \undeadref{skeleton}, subject to the same limitations as \featref{animate-zombie}.
	
	\materials{The bones of an animal corpse (with the flesh removed), a \circleref{small}, a lit candle which the ritual extinguishes}
	
	The reanimation ritual takes five minutes, and must be performed in the dark.
}

\feat{Raise Ghoul}{animate-ghoul}{20}{
	\skillref[1]{necromancy},
	\featref{animate-zombie}
}{
	\capital{\undeadrefplural{ghoul}} are faster and scarier than \undeadrefplural{zombie}, but also \emph{hungrier}.
	You may reanimate an animal corpse as a \undeadref{ghoul}, subject to the same limitations as \featref{animate-zombie}.
	
	\materials{An animal corpse, an additional corpse to be consumed by the \undeadref{ghoul}, a \circleref{small}, a lit candle which the ritual extinguishes}
	
	The reanimation ritual takes five minutes, and must be performed in the dark.
	At the conclusion of the ritual, the newly-arisen \undeadref{ghoul} must immediately be fed a complete corpse---of the same kind of animal as the ghoul---or it does not fall under the \practitionerpossessive{necromancy} control.
}

\feat{Raise Draugr}{animate-draugr}{20}{
	\skillref[1]{necromancy},
	\featref{animate-zombie}
}{
	By animating corpses as \undeadrefplural{draugr}, you can keep them around longer than mere \undeadrefplural{zombie}.
	You may reanimate a desiccated animal corpse as a \undeadref{draugr}, subject to the same limitations as \featref{animate-zombie}.
	
	\materials{An animal corpse {\embalmed} by desiccation (drying out), a \circleref{small}, a lit candle which the ritual extinguishes}
	
	The reanimation ritual takes five minutes, and must be performed in the dark.
}

\feat{Raise Sea-Draugr}{animate-sea-draugr}{10}{
	\skillref[1]{necromancy},
	\featref{animate-draugr}
}{
	A small variation on the ritual to animate \undeadrefplural{draugr} lets you create the opposite.
	You may reanimate a drowned, soaked animal corpse as a \undeadref{sea-draugr}, subject to the same limitations as \featref{animate-zombie}.
	
	\materials{A water-soaked animal corpse that died by drowning, a \circleref{small}, a lit candle which the ritual extinguishes}
	
	The reanimation ritual takes five minutes, and must be performed in the dark.
	The corpse must be kept soaked during the ritual, so keep a few buckets of water handy.
	And make sure the \materialref{ritual-circle} won't be washed away.
}

\feat{Raise Fossil}{animate-fossil}{20}{
	\skillref[1]{necromancy},
	\featref{animate-skeleton}
}{
	Fossilisation is naturally a slow process, but a dedicated \practitioner{necromancy} can accelerate the process.
	You may reanimate the bones of an animal corpse as a \undeadref{fossil}, subject to the same limitations as \featref{animate-zombie}.
	
	\materials{The bones of an animal corpse (with or without flesh) buried in a bog, a \circleref{small}, a small heap of finely crushed rock, a lit candle which the ritual extinguishes}
	
	Beginning the reanimation ritual requires five minutes, but the \undeadref{fossil} does not rise for 24 hours.
	For the entire 24 hours, the candle must remain lit, the \materialref{ritual-circle} must remain intact, and the area must remain dark.
	The witch need not be present for the whole duration, however.
	
	Over the course of the 24 hour period, the rock dust is drawn into the bog and incorporated into the bones, and any remaining flesh rots away.
	At the conclusion, the \undeadref{fossil} is animated and claws its way to the surface.
	The witch must be present at this point to assert immediate control over the \undeadref{fossil}, or it becomes uncontrolled.
}

\feat{Raise Blazing Skeleton}{animate-fire-skeleton}{20}{
	\skillref[2]{necromancy},
	\featref{animate-skeleton}
}{
	Playing with fire is dangerous, and playing with \undeadrefplural{fire-skeleton} is even worse.
	But you've decided it's worth the risk.
	You may reanimate the charred bones of an animal that died burning as a \undeadref{fire-skeleton}, subject to the same limitations as \featref{animate-zombie}.
	
	\materials{The charred bones of an animal that died burning (with the flesh removed), a \circleref{small}, a lit candle which the ritual extinguishes}
	
	The reanimation ritual takes five minutes, and must be performed in the dark.
	The ritual ignites the bones, but it requires fire to do it.
	Normally the candle suffices, but if the candle is substituted for a {\phylactery}, a flame must still be provided.
}

\feat{Raise Shade}{animate-shade}{20}{
	\skillref[1]{necromancy},
	\featref{animate-zombie}
}{
	A \undeadref{shade} is a valuable tool in a \practitionerpossessive{necromancy} arsenal, silent and deadly in the dark.
	You may reanimate an animal corpse as a \undeadref{shade}, subject to the same limitations as \featref{animate-zombie}.
	
	\materials{An animal corpse, a \circleref{small}, a lit candle which the ritual extinguishes}
	
	The reanimation ritual takes five minutes, and must be performed in the dark, \emph{at night}.
}

\feat{Raise Wraith}{animate-wraith}{20}{
	\skillref[2]{necromancy},
	\featref{animate-shade}
}{
	Using the victim of a violent death, and a sprig of \creatureref{stygian-nightshade} placed in its mouth, you can raise a powerful and violent variety of \undeadref{shade}: a \undeadref{wraith}.
	You may reanimate an animal corpse as a \undeadref{wraith}, subject to the same limitations as \featref{animate-zombie}.
	
	\materials{The corpse of an animal which died violently, \herbcreature{stygian-nightshade}{5}, a \circleref{small}, a lit candle which the ritual extinguishes}
	
	The reanimation ritual takes five minutes, and must be performed in the dark, \emph{at night}.
}

\feat{Deanimate}{deanimate}{10}{
	\featref{animate-zombie}
}{
	You can withdraw the animating force from a creature you have reanimated, returning it to death.
	This is far neater than beating it to death.
	
	\materials{An undead under your control, a \materialref{ritual-circle} of the same size required to initially animate the undead, an unlit candle which the ritual lights}
	
	The ritual takes five minutes, and must be performed in a brightly lit location.
	The undead must remain within the \materialref{ritual-circle} for the duration.
	You can affect multiple undead with one casting of the ritual, as long as they are in within the \materialref{ritual-circle} for the duration.
	The size of the \materialref{ritual-circle} required is determined by the size of the largest undead in the group.
}

\feat{Uncontrolled Deanimation}{deanimate-2}{10}{
	\featref{deanimate}
}{
	You can use the \featref{deanimate} ritual against even an uncontrolled undead.
}

\feat{Offensive Deanimation}{deanimate-3}{10}{
	\skillref[1]{necromancy},
	\featref{deanimate}
}{
	You can use the \featref{deanimate} ritual against any undead, be it under your control, under another \practitionerpossessive{necromancy} control, uncontrolled, or even a \undeadref{souled}.
}

\feat{Maintain Control}{undead-control}{10}{
	\featref{animate-zombie}
}{
	Keeping control of your undead can pose a real challenge, especially when you need to sleep.
	You've learned a ritual to place them in a sort of suspended animation, keeping them ready to reassert control over when you awake.
	
	\materials{An undead under your control, a \materialref{ritual-circle} of the same size required to initially animate the undead, a lit candle}
	
	The ritual takes five minutes, and the undead must remain within the \materialref{ritual-circle} for the duration of casting the ritual.
	When you have finished casting the ritual, the undead enters suspended animation; it simply remains in the \materialref{ritual-circle}, unmoving.
	It still counts against the number of undead you are controlling, but you need not remain nearby, or conscious.
	At any point when you are within range, you may end the ritual's effect and reassert control over the undead.
	
	The effect of the ritual lasts as long as the candle remains lit.
	With a large enough candle, you can get 2 or 3 days out it.
	If the candle is extinguished, or removed, you must be in a position to immediately resume active control of the undead---otherwise it becomes uncontrolled.
	
	If you have the ability to control multiple undead, you may affect multiple undead with one \materialref{ritual-circle} and one invocation of the ritual.
	The size of the \materialref{ritual-circle} required is determined by the size of the largest undead in the group.
	You may end the effect of the ritual on each undead individually.
}

\feat{Assert Control}{undead-control-2}{10}{
	\skillref[1]{necromancy},
	\featref{undead-control},
	\featref{deanimate-2}
}{
	Rather than drawing the animation out of undead, you can simply substitute your own animating force, taking control of the undead.
	This uses the \featref{deanimate} ritual, but requires an additional material: a lit candle which the ritual extinguishes---the same as animating an undead in the first place.
	
	On the completion of the ritual, instead of losing animation, the undead comes under your control.
	Undead you take control of this way still count against the maximum number of undead you can control, and exceeding this limit will free an earlier undead from your control, just as raising a new one would.
	
	Obviously, this is useless against an undead you already control, but it is useful in combination with \featref{deanimate-2} or \featref{deanimate-3}.
	You can even take control of a type of undead that you could not normally raise in this way.
	However, you cannot use this against undead that cannot normally be subject to a \practitionerpossessive{necromancy} control, such as \undeadrefplural{souled}---you must simply deanimate these.
}

\feat{Precision Control}{control-deanimate-small}{10}{
	\skillref[1]{necromancy},
	\featref{undead-large},
	\featref{undead-control} or \featref{deanimate}
}{
	You can use a \circleref{small} for the \featref{undead-control} or \featref{deanimate} rituals---assuming you have the feat to perform the ritual at all---regardless of the size of circle you would require to animate the creature in the first place.
	Note, however, that the undead must fit inside the circle, so you will need slightly bigger than a \circleref{small} for an elephant, or the like.
}

\feat{Rapid Control}{control-deanimate-fast}{20}{
	\skillref[2]{necromancy},
	\featref{undead-control} or \featref{deanimate}
}{
	You have become far faster at manipulating an undead's animating force---it can hardly be called a ritual anymore, though it still requires a \materialref{ritual-circle}.
	You can perform the \featref{undead-control} or \featref{deanimate} rituals in just one {\action}.
}

\feat{Contact Control}{control-deanimate-fast-2}{20}{
	\skillref[3]{necromancy},
	\featref{control-deanimate-small},
	\featref{control-deanimate-fast}
}{
	You can perform the \featref{undead-control} or \featref{deanimate} rituals without a \materialref{ritual-circle}, simply by touching the undead you want to affect.
	You still need all the requisite candles, however.
	Furthermore, you can only affect one undead per action if you do not use a \materialref{ritual-circle}.
}

\feat{Stitches}{undead-repair}{10}{
	\featref{animate-zombie}
}{
	Many reanimations and resurrections are ineffective on corpses which are too badly damaged.
	By sealing wounds, stitching severed parts back on, and gluing bones together, you can solve this.
	Any parts you reattach must come from the original creature.
	
	The repair and reanimation requires a {\test}, with the {\tn} determined by how badly damaged the corpse is, using your choice of \skillref{necromancy} or \skillref{healing}.
	A successful {\test} repairs at least enough {\damage} to restore the creature's \statref{shock-threshold} to 1, and a high result may repair even more.
	You must perform the repairs while the corpse is dead; you cannot repair it while it is animated.
}

\feat{Scraps}{undead-repair-2}{10}{
	\skillref[1]{necromancy},
	\skillref[1]{healing},
	\featref{undead-repair}
}{
	You can do more than stitch a damaged corpse back together; you can stitch \emph{several} corpses together.
	When using \featref{undead-repair}, you may assemble the corpse to be animated out of parts from different corpses.
	A corpse assembled out of several individually intact parts can be healthier than a single, damaged corpse.
	
	The pieces must all come from creatures of the same kind; all from humans, all from dogs, and so on.
	They must be assembled to form a creature of that kind; you cannot make a six-legged dog.
}

\feat{Chimera}{undead-repair-3}{25}{
	\skillref[3]{necromancy},
	\skillref[2]{healing},
	\featref{undead-repair-2}
}{
	You have mastered the art of assembling corpses, creating horrifying, chimeric monstrosities.
	When using \featref{undead-repair-2}, the parts needn't all come from the same kind of animal.
	They needn't form a normal creature, either; you could stitch extra legs on to a dog.
	
	The creature can typically use replacement anatomy easily; for example, if you replace a human's arms with a bear's.
	New anatomy, however---an extra pair of limbs, for example---may take several hours, or even days to learn.
	Neither a human with an animal mouth, nor an animal with a human mouth, can speak properly.
	
	The GM may invent a set of statistics for the resulting creature, based upon the component creatures and modified, as usual, by its kind of undeath.
}

\feat{Darning}{undead-repair-active}{15}{
	\skillref[1]{necromancy},
	\skillref[2]{healing},
	\featref{undead-repair}
}{
	You may make repairs to a corpse even while it is currently animated.
	Any {\tests} made to do so use \skillref{healing}.
	You may even reattach severed parts, though these must be the original parts unless you also have \featref{undead-repair-2}.
	\capital{\featref{undead-repair-3}} even allows you to attach parts from different kinds of creature.
}

\feat{Knitted Resurrection}{undead-repair-active-2}{15}{
	\skillref[2]{necromancy},
	\skillref[3]{healing},
	\featref{undead-repair-active}
}{
	You have discovered a means to resurrect dead tissue by attaching it to living tissue.
	This allows you to reattach severed parts to a person or animal.
	Any {\tests} made to do so use \skillref{healing}.
	
	This doesn't do much, if anything, to heal {\damage}; no more than normal surgery.
	The reattached parts, however, become living parts of the creature, for all intents and purposes.
	Some rot---up to about a week---is tolerable, though disgusting, and will be healed naturally after reattachment.
	
	The reattached parts must be the original parts, unless you also have \featref{undead-repair-2}.
	If you do have \featref{undead-repair-2}, however, you may replace failed organs, or broken limbs, with healthy ones from another creature.
	The target must remain alive throughout the entire process, so replacing a heart is incredibly difficult, and replacing a brain is impossible.
	\capital{\featref{undead-repair-3}} even allows you to attach parts from different kinds of creature.
}

\feat{Major Undead}{undead-large}{20}{
	\skillref[1]{necromancy},
	\featref{animate-zombie}
}{
	Larger bodies need more force to reanimate, but it's force you've learned to provide.
	When you perform a {\reanimationritual}, you may use a \circleref{medium} instead of a \circleref{small}, in order to ignore the upper size limit on the creature.
	You still cannot reanimate a {\strongsouled} creature, such as a human; you need \featref{undead-human} to do so.
}

\feat{Undead Head}{undead-head}{10}{
	\featref{animate-zombie}
}{
	Rather than bothering to reanimate larger creatures, you can just reanimate parts of them.
	The head, specifically, the seat of consciousness.
	
	You may reanimate a creature's severed head using a {\reanimationritual}.
	The usual restrictions apply; for example, you cannot reanimate a human unless you also have \featref{undead-human}.
	However, when evaluating whether you need \featref{undead-large} and a \circleref{medium}, consider only the size of the creature's head, not the whole creature.
	As such, any head short of an elephant's only needs a \circleref{small}.
	
	The resulting creature has all the limitations you would expect from a severed head.
	It can't move, and can only bite people who put their hands in its mouth.
	It has no \attref{might} or \attref{grace} scores for most purposes, though it retains its \attref{might} score for calculating its \statref{shock-threshold}, and for biting.
	It can still see, hear, and so on, and vocalise or speak as it could in life.
}

\feat{Partial Undead}{undead-part}{15}{
	\skillref[1]{necromancy},
	\featref{undead-head}
}{
	Sometimes, a \practitioner{necromancy} has to make do with just the parts of corpses that they can find.
	Sometimes, of course, they just want a hand.
	
	You may reanimate just part of a creature using a {\reanimationritual}.
	As with \featref{undead-head}, you cannot reanimate a human without \featref{undead-human}, and whether you need to use \featref{undead-large} is determined by the size of the part you reanimate.
	However small the part you reanimate, the result still counts as one undead under your control.
	
	The abilities of a reanimated part are determined by which part it is, and the GM should adjudicate this according to common sense.
	For example, a part without the head has no sense of sight, hearing, smell, or taste; only touch.
	Similarly, a single limb is likely to move more slowly than a full creature, and will suffer a considerable penalty to attacks---if it can attack at all.
	\capital{\statref{dodge-rating}} is also likely to be reduced.
	\capital{\statref{shock-threshold}} and \statref{resilience} are unlikely to be affected, however.
	Furthermore, the {\damage} the resulting undead part has sustained is determined only by the {\damage} to the relevant part, and it is not considered to have taken any {\damage} merely by being severed from the rest of the body.
	
	If you reanimate a part without a mouth, this causes problems when the creature breaks free of your control and becomes hungry.
	It cannot eat, so is behaviour will become increasingly erratic.
	It may begin to simply hunt and stockpile food, or something else, at the GM's discretion.
	Partial \undeadrefplural{ghoul} are even worse; if they cannot be fed, they cannot be controlled at all.
}

\feat{Sever Soul}{undead-human}{20}{
	\skillref[1]{necromancy},
	\featref{undead-large} or \featref{undead-head}
}{
	Reanimating a creature that was once {\strongsouled} has previously been impossible, due to interference from the residual link.
	You've learned to sever these links, and hence reanimate these creatures.
	
	You may reanimate a {\strongsouled} creature, such as a human, or an animal that was a familiar, using a {\reanimationritual}.
	You must use an iron blade as part of the ritual, to sever the link.
	
	Reanimating an entire human requires a \circleref{medium}, and \featref{undead-large}.
	Reanimating smaller parts of a human, using \featref{undead-head} or \featref{undead-part}, may not.
	
	A reanimated familiar has lost the link to its witch, and is now just a normal animal of its kind.
	See \featref{reanimate-familiar} to reanimate your familiar without losing this link.
}

\feat{Pull Yourself Together!}{skeleton-reassemble}{20}{
	\skillref[2]{necromancy},
	\skillref[1]{healing},
	\featref{animate-skeleton},
	\featref{undead-repair-active},
	\featref{undead-part}
}{
	By animating each bone of a \undeadref{skeleton} separately, you give it the ability to reassemble itself when scattered.
	When you animate a \undeadref{skeleton}, \undeadref{fossil}, or \undeadref{fire-skeleton}, you may give it this ability.
	This can make it quite hard to deanimate, however, so you can always choose to animate it without this ability.
	
	If an undead with this ability is destroyed by a {\damagetest}, or some other trauma that scatters its bones without destroying them, it does not lose animation.
	Over the next few minutes, the bones will draw themselves back together and reform the undead creature, which will begin normal operation again.
	The bones can be restrained to prevent this, but even so, will keep trying to draw themselves together, indefinitely.
	Even a \practitioner{necromancy} in control of the undead cannot give an order to prevent this; it is unresponsive to orders until reassembled.
	
	Thankfully, this does not entirely prevent deanimating the skeleton.
	It will lose animation if enough of the bones are broken, or, in the case of a \undeadref{fire-skeleton}, extinguished.
	Alternatively, the \featref{deanimate} ritual still works.
}

\feat{Undead Phoenix}{skeleton-reassemble-phoenix}{20}{
	\skillref[3]{necromancy},
	\skillref[1]{healing},
	\featref{skeleton-reassemble},
	\featref{animate-fire-skeleton}
}{
	You have learned an obscure ritual to emulate the legendary phoenix, creating an avian undead that is reborn from fire in an instant.
	When you use \featref{animate-fire-skeleton} on the bones of a flying bird, you may use this feat.
	If you do so, the resulting undead gains all the effects of \featref{skeleton-reassemble}, as well as several extra benefits.
	
	Firstly, despite its skeletal nature, the bird can fly just as well as it could in life.
	Secondly, its flame burns unstoppably; it cannot be extinguished in any way, except by deanimating it.
	Lastly, it can reassemble itself faster, taking only moments, not minutes.
	After it is destroyed, it reforms on its next {\turn}, though it cannot do anything else on that {\turn}.
	
	However, it is said that the phoenix is a unique bird---that only one ever exists at a time.
	While not quite true here, \emph{you} can certainly only create one.
	The last one you created must lose animation before you can animate another one---it is not enough to simply lose control of it.
}

\feat{Undead Familiar}{reanimate-familiar}{10}{
	\featref{animate-zombie}
}{
	While a {\soul} normally interferes with reanimating a creature, you've begun to figure out how to use it to your advantage, beginning on the path towards resurrection.
	Unfortunately, you can't actually summon any {\souls} back to their bodies yet.
	Not to worry, though, for you have quite ready access to one {\soul} in particular: your familiar's, so inextricably bound to your own.
	
	If your familiar dies and you can recover the corpse, you can reanimate it, paying no XP cost beside that required to purchase this feat in the first place.
	You may use any {\reanimationritual}, and it becomes the appropriate kind of \undeadref{souled}.
	
	Reanimating a familiar in this way does not prevent recovering it through the usual repetition of the binding ritual later (see the section \secref{familiar-injury-death}), although the normal XP cost must still be paid each time that method used.
}

\feat{Self-Sacrifice}{phylactery-self}{10}{
	\skillref[1]{necromancy}
}{
	Much of \discref{necromancy}---particularly efforts at resurrection---requires {\phylacteries}.
	They aren't easy to come by, however.
	The easiest way to acquire one, as is so often the case, is to make it yourself.
	Or, more precisely, to make it \emph{from} yourself.
	
	\materials{The clay jar to become the {\phylactery}, a drop of your own blood, \herb[deadly nightshade]{belladonna}{2}, a \circleref{medium}}
	
	The ritual requires 1 minute, and must be performed in a dark place.
	It is lethal, by its very aim---it tears your {\soul} free from your body and traps it in the {\phylactery}.
	
	Before killing yourself this way, you should have a plan to restore yourself.
	Examples include using the \featref{reanimate-souled} ritual simultaneously, a resurrection pact with a trustworthy friend who has the \featref{reanimate-souled} feat, having your familiar resurrect you (\featref{reanimate-souled-familiar}), or ensuring you can stick around to resurrect yourself (\featref{projection-phylactery}).
}

\feat{Imbue Soul}{reanimate-souled}{10}{
	\featref{reanimate-familiar},
	\featref{undead-large} or \featref{undead-head}
}{
	Creating a \undeadref{souled} has proven easier than you expected.
	The {\soul} almost leaps back into the body if given the chance.
	You may substitute a {\phylactery} for the lit candle in any {\reanimationritual}, and use the {\soul} within to reanimate the corpse as the appropriate kind of \undeadref{souled}.
	The {\soul} in the {\phylactery} must be the {\soul} from the body you are reanimating, however.
	
	Even when using this, you are subject to the usual limits of the {\reanimationritual} you use.
	As such, you need \featref{undead-large}, \featref{undead-head}, or \featref{undead-part} to reanimate a human.
	However, you may work with a {\strongsouled} creature even without \featref{undead-human}---you are incorporating the {\soul}, not severing it.
	
	The difficulty in using this feat, of course, lies in finding or creating a {\phylactery}.
	As such, this feat is only normally useful in conjunction with \featref{phylactery-self}. %TODO: List other useful feats.
	
	If you wish to reanimate \emph{yourself} as an undead, you may perform this ritual simultaneously with the \featref{phylactery-self} ritual.
}

\feat{Familiar Resurrection}{reanimate-souled-familiar}{10}{
	\skillref[1]{necromancy},
	\featref{reanimate-souled},
	\featref{phylactery-self}
}{
	Reanimating yourself is hard, what with being dead and all.
	So you've taught your familiar to do it for you.
	
	Your familiar learns to use any {\reanimationritual} you know, as well as \featref{reanimate-souled}.
	However, it is only the link between the {\soul} in your {\phylactery} and the sliver of the same {\soul} in your familiar that affords it the magical intuition to do this.
	As such, it can only perform the ritual using \emph{your} {\soul} and \emph{your} body.
}

\feat{Touching the Veil}{death-detection}{10}{
	\noprereq
}{
	When a {\soul} departs our world for the next, its passage disrupts the veil between worlds.
	A witch who knows what to look for can feel this disruption.
	
	You can feel where creatures have died, though this sense is dimmed by both distance and time.
	If you pass through the actual position of the death of a {\strongsouled} creature, you'll notice for up to about two weeks after it occurred.
	You automatically sense a {\strongsouled} death in the same room for a few days after it's happened, and in the same house for only about a day.
	The deaths of {\weaksouled} creatures fade even faster.
	
	A {\test} can reveal slightly older or more distant deaths, if you are searching for them.
	However, you can't gain any information about the identity of the victim or the cause of death.
	
	Locations of mass or repeated death can leave their traces lingering for much longer.
	The site of a battlefield or sacrificial altar may be felt for many years after.
}

\feat{Ghost Sight}{ghost-detect}{10}{
	\noprereq
}{
	To most people, {\ghosts} are invisible, inaudible, almost entirely entirely undetectable.
	But not to you.
	You can always see, hear, and even smell {\ghosts}.
	Still, however, you cannot touch them, and they cannot touch you.
	
	You can block out this sense, if a {\ghost} is annoying you.
	However, you cannot block out a {\ghost} that would be detectable even without this feat.
}

\feat{Exorcism}{exorcism}{10}{
	\featref{ghost-detect}
}{
	You may drive a {\ghost} out from this realm, forcing it across the veil to the afterlife.
	
	\materials{A \circleref{small}, three pinches of salt, three drops of water}
	
	The ritual takes 5 minutes, and the {\ghost} must remain within the \materialref{ritual-circle} for the duration.
	The ritual isn't unpleasant, but a {\ghost} with unfinished business may not want to move on, and may try to leave the circle.
}

\feat{Ghost Chains}{ghost-material}{15}{
	\skillref[1]{necromancy},
	\featref{exorcism}
}{
	Getting an enraged {\ghost} to hold still long enough for an \featref{exorcism} can be a real challenge, when you can't touch it.
	To this end, you have learned to enchant chains that can hold {\ghosts}.
	
	\materials{A \materialref{cold-iron} chain, salt}
	
	Enchanting the chain requires 5 minutes, and you must be touching the chain for the duration.
	It remains enchanted for 1 hour.
	During that time, it will touch {\ghosts}.
	They still cannot be harmed by it, but they can be restrained, and even moved around.
	
	The force that a {\ghost} can exert back upon the chain is very limited.
	It can't lift the chain, but it might be able to wriggle out of it if bound improperly.
}

\feat{Bound by Death}{ghost-creation}{10}{
	\featref{ghost-detect}
}{
	After a person dies, it takes a moment for their {\soul} to cross the veil.
	If you catch it in that moment, you can bind it to this world as a {\ghost}.
	
	\materials{A \materialref{cold-iron} nail, the corpse of a {\strongsouled} creature that died since the start of your previous {\turn}}
	
	This requires an {\action}, driving the nail into the corpse.
	The creature's {\ghost} emerges from the corpse and remains in the mortal realm until the next sunrise, then its crosses over to the afterlife.
	Note that this occurs regardless of whether it can see the sunrise.
}

\feat{Piercing the Veil}{ghost-creation-point}{15}{
	\skillref[2]{necromancy},
	\featref{ghost-creation},
	\featref{death-detection}
}{
	The opposite of \featref{exorcism}, you can reach through the veil and pluck a {\soul} back from the afterlife.
	You must do this where the veil has already been weakened: where the creature you want to make a {\ghost} of died.
	And you must do it before the veil heals: while you can still sense it with \featref{death-detection}.
	Lastly, it only works for {\strongsouled} creatures.
	
	\materials{A \circleref{small}, a \materialref{cold-iron} nail, three pinches of sugar, three drops of water}
	
	The ritual takes 5 minutes.
	At its completion, the {\soul} of the creature who died at the ritual's location is pulled back to this realm from the afterlife.
	It remains, as a {\ghost}, until the next sunrise, then fades back to the afterlife.
	Note that \featref{death-detection} doesn't give you any information as to \emph{who} died, so you may be in for a surprise when the {\ghost} arrives.
	
	Pulling a {\ghost} back through this weak point in the veil disrupts it badly, twisting it out of usefulness.
	It can still be sensed with \featref{death-detection}---even slightly more strongly---but it cannot be used for this ritual again.
}

\feat{Animal Ghosts}{ghost-creation-weak-soul}{10}{
	\skillref[1]{necromancy},
	\featref{ghost-creation}
}{
	A {\weaksoul} is harder to bind to this realm; it hardly has enough substance to for a {\ghost}.
	But you've learned to reinforce it with a little of your own will, to keep it around.
	You can create {\ghosts} of {\weaksouled} creatures using \featref{ghost-creation}---or using \featref{ghost-creation-point}, if you have it.
}
