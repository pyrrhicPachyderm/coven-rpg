\chapter{Character Creation Guide}
\chaplabel{character-creation-guide}

\section{Step Zero: Character Concept}

The most important part of a character is the concept.
Who is your character, what does she do?
You can try to flesh this all out now, or fill it in as you work through character creation.
Make sure your character is somebody you will enjoy role-playing.

Also make sure your witch fits into the coven; discuss this with the other players and your GM.
It can be quite painful for everyone involved playing an unscrupulous \practitioner{necromancy} in a coven of saccharine healers, or vice versa.
The GM should provide some idea of the tone intended for the game, to avoid this sort of trouble.
Diversity can also be good: make sure you know what you're letting yourselves in for if \emph{everybody} in the group wants to play a \practitioner{brewing}.

\section{Step One: Starting Experience}

Your GM will assign you an amount of experience (XP) to use during character creation.
By default, this is 120 XP, though the GM is free to adjust this to suit a different style of characters or campaign.

120 XP is suitable for witches who have just completed their apprenticeship and are taking over their own steading.
60 XP might be more suitable for witches who are still in an apprenticeship, and may have only just bound their familiar.
240 XP might be appropriate for witches with a few years of caring for a steading under their belt.
Even more experienced witches might require even more starting XP.
For fairness, the GM should probably give all characters the same starting XP, unless there is a good reason otherwise.

Make a note of how much XP you have, and keep track as you spend it later.
It can be well worth keeping a record of everything you have spent XP on over the course of character creation and any later development.

\section{Step Two: Attributes}

Attributes are a witch's broad, innate capabilities.
Is she skinny and lithe, or broad and well-muscled; quick-witted, or bullheaded; domineering, or silver-tongued?

At character creation, you have 20 points to spend on your character's attributes.
Set each of the eight attributes to between 0 and 4, inclusive, such that they sum to 20.

Attributes represent much more innate ability than skills.
While they can be developed through much hard work, they are not often passively improved over the course of a lifetime.
As such, it is not recommended to change the number of points available for attributes as readily as one might change the starting XP.

\section{Step Three: Skills}

A witch does not begin totally unskilled.
Select 1 {\generalskill}; you begin with 2 ranks in this skill.
Select an additional 3 {\generalskills}; you begin with 1 rank in each of these.
Lastly, select 1 {\specialityskill} (and your speciality in it); you begin with 1 rank in this.

The GM is also free to adjust the number of skills and ranks granted to starting characters.
\capital{\generalskills} represent general life experience, {\specialityskills} tend to result from vocational experience, and {\disciplineskills} represent experience with magic and witchcraft.
However, due to the nature of the skill discount system, ranks in {\disciplineskills} are better acquired through XP rather than granted at character creation.
After all, a \practitioner{druidcraft} or \practitioner{headology} won't need any {\disciplineskills}.

\section{Step Four: Familiar}

A witch's familiar is her essential and constant companion, and is usually bound early in her apprenticeship.
The available familiars are detailed in \chapref{familiars}.
Select one, spending XP (including the XP for any options) from your starting XP if necessary.

A witch can never have more than one familiar.
However, with the GM's approval, you may decline to select your familiar yet.
In this case, you begin play without a familiar and must acquire it during play, spending the necessary XP then.
This can be useful if you want a familiar with a high XP cost, but want to spend your starting XP on other things.

\section{Step Five: Spending Experience}

You should almost certainly have some XP left over after purchasing your witch's familiar, and now you can spend it.
There are three main things to spend XP on: attributes, skills, and feats.
The costs for \seclink{improving attributes}{improving-attributes} and \seclink{improving skills}{improving-skills} are given in \chapref{attributes-and-skills}.
Feats can be found in the chapters of \partref{disciplines}, and every feat lists its cost as part of its description.
You may also spend XP on attributes, skill, and feats for you familiar, as described in the \secref{familiar-advancement} section of \chapref{familiars}.

Feats from some disciplines are cheaper if you are sufficiently advanced in the skill that governs that discipline.
A feat is 5 XP cheaper if you have a higher level of the discipline's governing skill than is required for the feat.
This is only ever 5 XP, regardless of how much higher your skill is than necessary.
For example, a witch with \skillref[3]{brewing} who buys a \discref{brewing} feat that requires only \skillref[2]{brewing} gets a 5 XP discount.
A witch with \skillref[1]{brewing} or higher gets a 5 XP discount when buying a \discref{brewing} feat that does not require the \skillref{brewing} skill at all.

Some feats cost only 5 XP, providing the same discount on the skill as it costs to buy them.
This means they are essentially free to take if you plan to progress further in the discipline.

The disciplines of \discref{headology} and \discref{druidcraft} do not have governing skills, and can never benefit from this discount, but all other disciplines do.
Similarly, feats from the \chapnameref{familiars} chapter never benefit from discounts.

Don't worry about building the perfect witch and buying every feat you want at this point.
Many feats simply aren't accessible to a witch just starting out, and you will get more XP to spend as you play.
Similarly, don't worry if you have a little bit of XP left over; you'll be able to spend it once you get more.

\section{Step Six: Derived Statistics}
\statlabel{Resilience}{resilience}
\statlabel{Shock Threshold}{shock-threshold}
\statlabel{Dodge Rating}{dodge-rating}
\statlabel{Speed}{speed}

Once you attributes have been finalised, you can calculate your derived statistics.
Record them on your character sheet.
These are based upon your attributes, as listed in the following table, and will be used by some of the rules of the game.
Whenever your attributes change, you should also recalculate these statistics.

You should also calculate and record these statistics for your familiar, except for \statref{speed}, which is given among the familiar's statistics in \chapref{familiars}.

\simpletable{ll}{
	\toprule
	Statistic & Derivation\\
	\midrule
	\capital{\statref{resilience}} & $3$\\
	\capital{\statref{shock-threshold}} & $12 + \text{\attref{might}} + \text{\attref{will}}$\\
	\capital{\statref{dodge-rating}} & $8 + \text{\attref{grace}} + \text{\attref{heed}}$\\
	\capital{\statref{speed}} & $8 + \text{\attref{might}} + \text{\attref{grace}}$\\
	\bottomrule
}

\section{Steading}

Most witches have a steading.
This is the area a witch watches over, a region she defends and protects the inhabitants of.
The duties a witch has to her steading are numerous and varied, but typically involve healing the inhabitants and protecting them from threats of a magical nature.
Some witches also perform midwifing, care for the land itself, or even take it upon themselves to deal with non-magical threats, such as invading armies.
A witch's responsibilities are not limited to her steading, and nothing stops her from responding to threats outside it.
But inside it, everything is certainly her responsibility.

Decide whether your witch has a steading.
How big is it?
One village, several, or an entire kingdom?
What duties does she perform within it?
Do the inhabitants appreciate what she does for them?

Also discuss this with your GM, and the other players.
Has the GM already described a village that could be your steading?
It is not unheard of for witches to share a steading, although this can obviously lead to disagreements.
Do you share a steading with your coven, or have you carved the local region into one steading each?

\section{Example Character Creation}

This section follows Jane as she builds a character for a game, to provide an example.

Jane is playing the game with three friends, one of whom has offered to be the Game Master.
That leaves Jane and her other two friends playing a coven of three witches.
After a quick discussion, they decide to emulate the stereotypical coven of three: the maiden, the mother, and the crone.
Jane takes the role of mother.

Looking through the disciplines of magic available in \partref{disciplines}, Jane decides she'd like to try her hand at \discref{golemancy}.
In particular, she likes the idea of gingerbread golems, so she decides that her witch is also a baker.
She settles on the name Gertrude---Gertie for short---then moves onto the next step.

\subsection{Example Attributes}

It's Jane and her friends' first game, so the GM uses the standard starting experience, 120 XP, as well as the standard set of attribute and skill points.
Jane notes this down, then begins assigning her attributes.
She notes that \attref{will} is important for animating golems, and a mother should be wilful.
So she puts 4 points into \attref{will}; the maximum allowed at character creation.
She decides Gertie is fairly sociable; likeable, but stern when necessary.
So she puts 3 points into both \attref{charm} and \attref{presence}.
Gertie is no smarter than most people, so she takes 2 \attref{wit}; the human average.
She's worldly, however, and has a good memory, so she takes 3 \attref{ken}.

Jane decides that Gertie is a little unobservant, so she'll give her a below-average \attref{heed}.
She's not totally hopeless, though; not deserving a score of 0.
She settles on a score of 1.
She's now spent 16 of her 20 points on attributes, leaving 4 to split between \attref{might} and \attref{grace}.
Gertie is neither weak, nor clumsy, so Jane splits the points evenly---2 of each, average in both.

\atttable{2}{2}{3}{2}{4}{1}{3}{3}

\subsection{Example Skills}

With attributes out of the way, Jane turns to skills.
Gertie's {\specialityskill} is obvious; \skillrefspeciality{crafting}{Baking}.
She gets 1 rank in that.
Next, she looks at {\generalskills}.
Per the standard set, Gertie gets 2 ranks in one {\generalskill}, and 1 rank in each of three others.

Firstly, Jane notices that the \featref{gingerbread-golem} feat requires ginger, an \herbtype{3}.
\capital{\herbtypeplural{3}} require \skillref[1]{botany} to cultivate, so she gives Gertie that.
Next, she wants to give Gertie a few social skills; she settles on \skillref[2]{persuasion} and \skillref[1]{socialising}.
Jane picks \skillref[1]{healing} as Gertie's final skill, picked up from years of midwifery and tending to scrapes.

\subsection{Example Familiar}

Next, Jane picks Gertie's familiar.
She selects a \familiarref{dog}; a sheepdog, in particular.
He is a border collie, and she names him Oscar.
She pays 20 XP for a \familiarref{dog}, and an additional 5 XP for a sheepdog.
Subtracting this from her 120 starting experience, she has 95 XP to spend on other things.

\subsection{Example Experience Expenditure}

Jane starts with Gertie's feats.
Obviously, she needs \featref{gingerbread-golem}.
She wants more than one of them, so she also takes \featref{more-golems}.
Paying 15 XP for each of these, she now has 65 XP left.
She would also like to take \featref{more-small-golems}, but that requires \skillref[1]{golemancy}, and she doesn't have that yet.

The first rank of a {\disciplineskill} such as \skillref{golemancy} normally requires 40 XP.
However, this is discounted by buying feats from the relevant discipline.
Gertie now has 2 \discref{golemancy} feats, and each required 0 ranks of the \skillref{golemancy} skill.
So each one gives her a 5 XP discount on her next rank of \skillref{golemancy}, rank 1.
This reduces the cost of \skillref[1]{golemancy} to 30 XP, leaving Gertie with 35 XP after she buys it.
Now she can buy \featref{more-small-golems} for 15 XP, leaving her with 20 XP, and the ability to animate 4 golems at once.

Next, Jane likes the look of \featref{golem-reanimate}---this would allow her to carry a basket of gingerbread biscuits, and animate them on the go.
This requires \featref{golem-change-instructions} first, however.
Both feats cost 15 XP, for a total of 30; more than the 20 XP she has remaining.
Thankfully, Jane soon notices that the discounts come to help her here too.

\capital{\featref{golem-change-instructions}} and \featref{golem-reanimate} are both \discref{golemancy} feats, and neither require ranks in the \skillref{golemancy} skill.
Gertie now has 1 rank in that skill, however, which is more than is required for the feats.
As such, she gets a 5 XP discount on both feats.
This makes them 10 XP each, and she buys them both, using the last of her XP.

Gertie now has five feats (\featref{gingerbread-golem}, \featref{more-golems}, \featref{more-small-golems}, \featref{golem-change-instructions}, and \featref{golem-reanimate}) and a variety of skills (\skillref[2]{persuasion}, \skillref[1]{socialising}, \skillref[1]{botany}, \skillref[1]{healing}, \skillref[1]{golemancy}, and \skillrefspeciality[1]{crafting}{Baking}).

\subsection{Example Statistic Derivation}

The last mechanical step is to calculate Gertie's derived statistics.
Her \statref{resilience} is 3, as it is for all creatures of flesh and blood.
The others are calculated from the formulae and her attributes.
She has a \statref{shock-threshold} of 18 ($12+2+4$), a \statref{dodge-rating} of 11 ($8+2+1$), and a \statref{speed} of 12 ($8+2+2$).

\subsection{Example Description}

With that out of the way, Jane moves on to describing Gertie and her lifestyle.
Gertie is a little short, and a little plump---she likes her own baking a bit too much.
She is married to the village miller; a convenient arrangement for the business of her baking.
Jane names Gertie's husband Howard, making the couple Gertrude and Howard Miller.
They have five children, between the ages of 9 and 19.
Gertie herself is in her early 40s.
For a bit of fun, Jane decides that the couple's eldest daughter is getting married soon.
She informs the GM---the marriage might make for an interesting plot element.

Gertie's {\cottage} is about 10 minutes walk from the village of Alderback.
It sits at the bottom of a small hillock, with the windmill on top.
It's a little larger than many witches' cottages, to accommodate the family of seven, but it's far from extravagant.

Gertie's steading covers the village of Alderback and surrounding farms.
A single village is small, as steadings go, but Gertie is also the village baker, and a busy mother.
The other two members of her coven tend to the villages a few miles upstream and downstream, respectively.

\subsection{Example Equipment}

Gertie's most important piece of equipment is a wicker basket filled with gingerbread biscuits, all human-shaped and ready to be animated as golems.
She always keeps a couple of loaves and a few other biscuits in there too, for lunch, and for bribes.
She doesn't carry any weapons, preferring to diffuse situations by conversation.

Gertie is seen in her baking apron (embroidered ``World's Best Mum'') more often than not, although she wears a black cloak over it when going out.
Her pointed hat is unremarkable---black, a little faded---but it always has a spare gingerbread man tucked into the hatband.
Her broomstick lives in the cupboard, and doesn't see much use with such a small steading.

Gertie sources most of her own baking supplies.
The flour comes from her husband's business, while Gertie keeps a couple of dozen chickens for eggs, and a handful of sheep for milk and butter.
Her {\garden} grows a handful of herbs, most importantly ginger, for her gingerbread golems.
Ginger is an \herbtype{3}, but she can grow this using her \skillref[1]{botany} skill.
