\discipline{Headology}{headology}{Headologist}{Headologists}

\dropcapdiscref{headology} is really no magic at all.
Rather, it is the art of making other people use their own magic.

One common misconception among apprentice witches is that \discref{headology} is the ability to affect people's minds.
Their mentors must quickly disabuse them of this notion.
Everyone has the ability to affect people's minds, and uses it every day.
It's called talking.
It can make someone like you or hate you, make them smarter or more stupid, even make them believe that the sky is purple, if you're really good.
It's an incredibly ability---the most important one a witch can have, in the opinion of most.
Enough people, sufficiently motivated, can move mountains.
But talking, by itself, is not \discref{headology}.

\capital{\discref{headology}} is the step that comes after.
\capital{\discref{headology}} is making people's minds affect the world---letting them move mountains without all the shovels and wheelbarrows it normally requires.
Making them into \practitioners{willing}, without them even knowing it.

Every person, and even every animal, has the ability to affect the world through \discref{willing}.
Most never realise this, and would struggle to control the power even if they knew.
But a witch who knows the trick of it can unlock another person's ability.
And if she's convinced them of the correct things first, she can direct it with her words.
This is the basis of most \discref{headology}.

\section{Convincing People}

Almost every feat in \discref{headology} requires a witch to convince somebody of something, before it has any affect.
Talking to people is a complicated subject, and there are no strict rules for this.
As such \discref{headology} is more subject to the whims of the GM than many other disciplines.
The following paragraphs provide many guidelines for adjudicating this, but the GM should also apply common sense, and remember to ensure that everyone is having fun.
If the the \practitioner{headology} is consistently overshadowing the rest of the coven, it's probably proving too easy to convince people of things, and vice versa.
And if they've put on a particularly awesome show to convince someone, just let it work.

Firstly, if there is doubt as to whether a \practitioner{headology} has convinced someone, call for an {\opposedtest}.
On the part of the \practitioner{headology}, this will normally use \attref{charm} or \attref{presence}, and \skillref{persuasion} or \skillref{deception}.
\capital{\skillref{intimidation}} and \skillref{socialising} might come into it fairly often, as well.
On the part of the intended victim, this might use \attref{will}, to hold onto a conviction, or \testtype{heed}{insight}, to see through a trick.

\subsection{Modifiers to Convincing}

For anything but the simplest effects---such as \featref{curse}---simply stating something is not enough to convince someone, no matter how persuasive your tone.
In these circumstances, the GM should simply not let the victim be convinced, or at least apply a major penalty to the {\test} to convince them.
Often, some variety of evidence or trick is required.

For example, a prince who's simply told he's a frog is unlikely to fall for it.
But a prince who's told he's a frog, then gets knocked out, and wakes up in a pond with his skin covered in slime---he's more likely to buy it.
With the right evidence, a witch might not need to speak to the victim at all.
A prince who wakes up in a pond, surrounded by other frogs, all dressed in the armour of his personal guard---he's going to leap to his own conclusions.

As such, the GM should use the circumstances to put modifiers on rolls to convince people.
This should often be a negative modifier without a good argument or some evidence, while presenting a solid piece of evidence can give a positive modifier.
An elaborate---but solid and successfully executed---plan for convincing someone will often bypass the need for a direct {\test} altogether.
The type of effect being applied should also influence the modifier.
It is much easier to convince someone that they're under a simple bad-luck curse than that they're a frog.

Lastly, the \practitionerpossessive{headology} reputation can be important.
If the victim knows that she is a powerful witch, this can go a long way by itself.
If she specifically has a reputation for turning people into frogs, people are likely to believe her pretty easily when she says she's turning them into a frog too.
Even more so if they've just seen her do it to one of their friends.
Practically, this means that a \practitioner{headology} often needs to make it clear that she's a witch, by wearing {\thehat}.

\subsection{The Trick of Headology}
\seclabel{headology-trick}

One unfortunate catch of \discref{headology} is that it only works as long as the victim is unaware of quite what's being pulled on them.
As soon as someone realises that they'll only turn into a frog if they believe they're a frog, they'll never believe it.
Even if they try.
This means that it is impossible to \emph{willingly} be the subject of \discref{headology}.

It also means that a \practitioner{headology} needs to be careful not to let their victims catch on to what's happening.
This is not commonly a problem with normal folk, unless someone explicitly explains it them.
Superstition runs rife, and a witch who uses a lot of \discref{headology} is likely to provoke more fear and respect than understanding.
However, a \practitioner{headology} ought to maintain a certain mystique about her craft, to ensure no clever clogs goes digging too deeply.

Other witches, however, tend to catch on quite quickly.
A witch who has seen a particular trick of \discref{headology} used a couple of times tends to figure it out, and thereby become immune to it, whether she wants to or not.
Other \practitioners{headology} tend to be even quicker on the uptake, and are likely to catch on the very first time they see a trick, if the witch using it on them isn't careful.
A witch who knows and uses a trick herself can never be affected by it, except, perhaps, in the most exceptional caper of all time.

Perhaps mostly importantly, this means that you can never use your \discref{headology} on your own coven, unless you are careful to keep them in the dark about a new trick you've picked up.
Even then, it won't last long.

\section{Feats}

\feat{Curse}{curse}{15}{
	\noprereq
}{
	It's a well known fact that someone who believes they will fail is more likely to do so.
	It doesn't take a drop of magic to make that true, but not everyone knows how to leverage it.
	You do.
	
	If someone believes that you have cursed them, or even if you can convince them that they have been cursed by something else, they suffer bad luck.
	Whenever they make a {\test}, dice that roll a 3 count towards a critical failure.
	The GM is also encouraged to make their critical failures a little more dire.
	This bad luck persists as long as the supposed curse is present in their minds; it might help to remind them now and again.
	
	This only applies if they believe they are under a rather broad curse, or specifically a bad luck curse.
	An overly specific curse---for example, ``May your crops wither in your fields,'' or ``May your nose fall from your face''---does nothing to focus their mind on their own failure and will have no effect.
}

\feat{Total Failure}{curse-2}{20}{
	\featref{curse}
}{
	Worse than a mere \featref{curse}, you have learned to provoke total failure.
	If you can convince somewhat that they \emph{will} fail, or that what they are trying to do is impossible, then they will, indeed, fail.
	Any {\tests} they make---if in pursuit of a goal they believe they cannot possibly achieve---automatically fail.
}

\feat{Mentally Scarred}{headology-wound}{10}{
	\featref{curse}
}{
	You have mastered a more specific form of curse---a curse of physical wounding.
	
	If you can convince someone that they are wounded, they develop the wounds they believe they have.
	This directly causes {\damage}---not a {\damagetest}---appropriate to the kind of wound they develop.
}

\feat{Brain Death}{headology-kill}{15}{
	\featref{headology-wound}
}{
	Skipping the messy process of wounding, you can kill people with their minds.
	If you can convince someone that they are dead, they die.
	Unless you convince them that they have died of their wounds---in which case \featref{headology-wound} also kicks in---they die without a mark on their body.
}

\feat{Senseless Fool}{headology-remove-sense}{20}{
	\featref{headology-wound}
}{
	\featref{headology-wound} will blind someone if you convince them that you've scratched their eyes out, but there are more direct, less violent ways to render a person senseless.
	If you can convince someone that they have have lost a sense---that they are blind, for example, or deaf---then they lose that sense.
	They remain without that sense for as long as they remain convinced---likely until you convince them otherwise, considering that they \emph{have} now lost that sense.
}

\feat{Assured Success}{blessing}{15}{
	\noprereq
}{
	You can leverage the power of positive thinking in other people.
	If you can convince people that they \emph{will} succeed, they are more likely to.
	While they remain convinced of this, whenever they make a {\test}, dice that roll a 5 count towards a critical success.
	Note that this only functions as long as they believe they \emph{will} succeed---simply thinking they are blessed, or more likely to succeed, is not sufficient.
}

\feat{Blessing}{blessing-2}{20}{
	\featref{blessing}
}{
	The opposite of \featref{curse}, you can convince people that they are blessed.
	You can grant the benefit of \featref{blessing} simply be convincing people they are under some magical boon that makes them more likely to succeed.
}

\feat{Mind over Magic}{foil-magic}{15}{
	\noprereq
}{
	For all the magic circles and burning incense, magic ultimately comes from the mind.
	Not only do you know this, but you know \emph{how to exploit it}.
	
	If you can convince a practitioner of magic that their magic won't work, then it won't.
}

\feat{Doubt \& Despair}{foil-magic-2}{25}{
	\featref{foil-magic}
}{
	Under your tender care, even the smallest seed of doubt can flourish into a blossoming tree of failure.
	
	If you can make a practitioner so much as doubt the efficacy of their magic, or their own ability to work it, then the magic will either fail to work or, at the GM's option, backfire.
}

\feat{Mind Like a Razor}{headology-weapons-improvise}{10}{
	\featref{willing-tools-improvise}
}{
	If you can convince your foes that what you wield is a weapon, their flesh will believe you.
	You may treat an item you wield or throw as a \weaponref{knife}, \weaponref{hand-weapon}, or \weaponref{thrown-weapon} (depending on its size and whether you're throwing it) if you can convince the target that it can cut (or otherwise deal damage) like one.
	A demonstration against an inanimate object, or another foe, will often suffice.
	Even your bare hands can cut like \weaponrefplural{knife} if you convince your foes that they can.
}

\feat{Change Blindness}{headology-stealth}{10}{
	\noprereq
}{
	You may hide in plain sight by leveraging the fact that people don't \emph{expect} to see you there.
	This uses a \testtype{charm}{stealth} {\test}.
	You must remain silent and quite still, though you may creep around slowly.
	
	In order to make use of this feat, anyone you are hiding from must have no reason to expect to see you, or anyone.
	If they see much out of place---a drawer opened or a vase knocked over---they might look for whoever did it and will immediately spot you.
	Furthermore, you can only use it if the people you are hiding from have some degree of familiarity with the location; they must have seen it before, at least.
	Somebody entering a room for the very first time doesn't know what to expect and will see it as it is, you included.
	
	Lastly, somebody seeing a group or crowd of people has no reason not to expect other people with them.
	This feat does not allow you to hide in such a situation, unless everyone in the group has the feat.
	%TODO: Is there a feat that helps blending in with a crowd?
}

\feat{Elsewhere}{headology-stealth-2}{15}{
	\featref{headology-stealth}
}{
	While \featref{headology-stealth} lets you hide from people who aren't expecting \emph{anyone}, you've now figured out how to hide from people who aren't expecting \emph{you}.
	As long as someone is convinced \emph{you} won't be somewhere---for instance, you've told them you'll be somewhere else---they won't see you there.
	Note that it is not enough for them not to expect you there---except as falls under the purview of \featref{headology-stealth}---they must expect you not to be there.
	
	This only holds up as long as you aren't too too intrusive.
	For example, you shouldn't pass in front of something they are paying attention to, make any loud noises, or open any doors they are looking at.
	However, you might be able to get away with moving things around.
	Even if somebody notices that something has been moved, they ought not to suspect \emph{you} to have done it, as long as they still believe you are somewhere else.
	\capital{\tests} to avoid being noticed, if it is in doubt, use \testtype{charm}{stealth}.
	
	Furthermore, this feat does allow you to go unnoticed in a crowd, as long as the person watching has good reason to believe you won't there.
}

\feat{You Shall Not Pass!}{headology-barrier}{20}{
	\noprereq
}{
	You may erect barriers inside people's heads, allowing them to project them into reality.
	If you convince someone that they cannot pass some barrier, they become unable to.
	Even if they are thrown bodily against the barrier, they will bounce off it.
	This does not prevent them throwing stones, poking a stick, using magic across the barrier, or the like.
	
	The barrier can be of any shape or nature that you can convince the target of.
	For example, you might draw a line in the sand, convince them that they cannot enter a house, or tell them that they cannot touch you.
}

\feat{Can't Touch This}{headology-invincibility}{15}{
	\featref{headology-barrier}
}{
	If you convince someone you are invincible, they can't hurt you, even if they try.
	Their blows will miss, or even bounce straight off your skin.
	This works regardless of how they try to hurt you---by axe, by arrow, by pushing a boulder off a cliff at you, even by poisoning you.
	This doesn't protect you from anything that's not an attempt by a person, however; walking through brambles will still tear your skin, and likely break your illusion of invincibility.
}

\feat{Fake Sympathy}{headology-sympathetic-magic}{25}{
	\skillref[1]{sympathetic-magic},
	\featref{symlink-declare}
}{
	With \featref{symlink-declare}, you learned to create a {\symlink} through a small trick of \discref{headology}, making the target expect the link.
	Now, you've taken it a step further, abandoning the use of a {\symlink} entirely, and using raw \discref{headology} to achieve the same effects.
	
	You may establish a fake {\symlink} just by convincing the target that you have established one.
	They need not understand the actual mechanisms of \discref{sympathetic-magic}---in fact, it's probably better if they don't---they just need to know that by affecting the {\symbol}, you can affect them.
	It does not count towards your maximum number of {\symlinks}
	It lasts as long as the target continues to believe it does---as such, it is not subject to {\stress}.
	
	You may transmit any effects along this fake link that you could along a normal {\symlink}---anything you possess the appropriate \discref{sympathetic-magic} feat for.
	However, you may only do so by showing the target what you are doing, and even explaining it if necessary.
	For example, \featref{sympathetic-speak} is useless: if the target cannot hear the sounds anyway, they don't know what to expect, and receive nothing.
	
	This {\symlink} doesn't actually exist in any sense, so you cannot modify it in any way you could modify a normal {\symlink}, such as with \featref{symlink-reverse} or \featref{symlink-knot}.
	However, neither can anybody else modify it, and it is not impeded by anything that would impede a normal {\symlink}, unless the target is aware of and believes in such impedance.
	Similarly, it is not affected by {\stress}, and hence behaves like a {\strongsymlink}.
	Lastly, there is no limit on its duration; it lasts as long as the target continues to believe it does.
}

\feat{Placebo}{headology-brewing}{15}{
	\noprereq
}{
	Often, the promise of a cure is more important than the cure itself.
	You can save a lot of time brewing this way, if you just talk to people.
	
	If you know how to make a brew, and have a mixture of approximately the right size, consistency, and colour, you might be able to use that instead.
	If you can convince someone that what they're taking will have the effect of that brew, then it acts as that brew for them.
	This works not only with brews that you have a feat to make, but also the same minor remedies that you might otherwise make with a \skillref{brewing} {\test}.
	However, if a brew requires a feat to make, and you don't have that feat, this won't work.
	
	This only works if they are convinced at the time they take the brew; it can't work retroactively.
	As such, it's not all that much use for poisoning people.
}

\feat{Retroactive Placebo}{headology-brewing-2}{15}{
	\featref{headology-brewing}
}{
	If you try to convince someone that you've poisoned their wine, they're hardly likely to drink it.
	But if you convince someone that you'd poisoned the wine they've just drunk, they might well drop dead.
	
	You may use \featref{headology-brewing} even if you convince someone \emph{after} they take the brew.
	Bear in mind that, obviously, they're unlikely to believe you if there's no way you could have touched the mixture they drank.
	
	The time taken for the brew to kick in is counted from when they took it, not when you convince them.
	As such, the effect will often kick in immediately after you convince them.
	It doesn't matter if this means it kicks in late, as long as they can believe they've been resisting it, or it's rather slow-acting.
	Convincing someone that yesterday's poison is only now affecting them might be difficult, though.
	
	This also functions for \featref{headology-brewing-antidote} and \featref{headology-brewing-antidote-2}; you can convince people that a mixture was an {\antidote} after they take it.
}

\feat{Poison is in the Mind}{headology-brewing-antidote}{10}{
	\featref{headology-brewing}
}{
	Sometimes it's useful to end the effect of a potion without giving away that it was fake all along.
	In this case, you can give someone an {\antidote}.
	Just as fake as the original, of course.
	
	If you can convince someone that a mixture they take is an {\antidote}, it functions as one.
	It counteracts whichever mixtures you convince them that it will.
	However, it can \emph{only} counteract mixtures that were applied using \featref{headology-brewing} in the first place; it is not effective against any real brew.
}

\feat{Placebo Panacea}{headology-brewing-antidote-2}{20}{
	\featref{headology-brewing-antidote}
}{
	You can counteract even the deadliest poisons with plain water, if your powers of persuasion are up to scratch.
	When you use \featref{headology-brewing-antidote}, the fake {\antidote} may counteract \emph{any} brew, even a real one.
}

\feat{Thinking in Circles}{headology-ritual-circle}{10}{
	\Anyfeatmaterial[a]{ritual-circle}
}{
	You can essentially outsource your \discref{ritual-magic} to the minds of others.
	If you can convince the majority of the people present that a \materialref{ritual-circle} is the correct \materialref{ritual-circle} to perform a particular rite, then you---and only you---can use it to perform that rite.
	This obviously requires the majority of the people present to be people you \emph{can} trick with \discref{headology} (see the section \secref{headology-trick}).
	You will likely have to explain the rite, to ensure they know what to expect, and believe.
	
	You must have any feats necessary to perform the rite normally---this only allows you to use the wrong circle.
	The circle must be the correct size, and, as usual, cannot overlap or encompass another circle.
	A given circle can only be used in one rite at a time.
	This only works for \materialrefplural{ritual-circle}, not \materialrefplural{stone-circle}.
	
	With sufficiently gullible people, a circle consisting of a single line with no other symbols might be sufficient.
	This sort of circle can be scribed far faster than any circle with the proper symbols.
	
	The people only need to believe while the rite is performed.
	Any ongoing effects will continue even if their belief ends.
}

\feat{One For Circles}{headology-ritual-circle-2}{20}{
	\featref{headology-ritual-circle}
}{
	You don't need that much power to fake a \materialref{ritual-circle}.
	When you use \featref{headology-ritual-circle}, you only need to convince \emph{one} of the people present.
}

\feat{Circles For All}{headology-ritual-circle-other}{10}{
	\featref{headology-ritual-circle}
}{
	You may use \featref{headology-ritual-circle}---or \featref{headology-ritual-circle-2}, if you have it---on behalf of others.
	As long as enough of the people present are convinced that the \materialref{ritual-circle} is correct for a given rite, any witch may use the \materialref{ritual-circle} for that rite, if she knows it.
	Notably, \emph{you} don't need to know that rite, only the other witch does.
}
