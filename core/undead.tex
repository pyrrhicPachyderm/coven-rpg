\chapter{Mortuary}
\chaplabel{undead}

\section{Effects of Undeath}

The statistics and capabilities of an undead creature are based upon those of the creature whose corpse it is raised from.
However, each variety of undead comes with its own associated changes to these statistics, listed in the following sections.
Nonetheless, many changes are shared between all reanimated creatures, and are listed here.

Most reanimations do nothing to heal {\damage} to the corpse: both {\damage} suffered before and during death, and any further {\damage} done to the corpse since then.
If this reduces its \statref{shock-threshold} to 0 or below, the corpse is too mangled to successfully reanimate.
Some reanimations also require an unrotted corpse.
It usually takes a little over a week before a corpse becomes too rotted for such a reanimation, although temperature and moisture can alter this.
Corpses can be preserved by {\embalming}.

A reanimated creature loses its memory and identity.
It retains general knowledge such as how to hunt, but forgets such information as the location of its den, and loses any mannerisms that distinguished it in life.
It retains {\generalskills}, but loses {\specialityskills}, {\disciplineskills}, and feats.

Most reanimated creatures also lack many biological processes that they had in life.
They do not need to breathe, eat, drink, or sleep.
They are immune to poisons, diseases, and the like.
Additionally, they cannot heal themselves or be healed, and are unaffected by potions and such.
They do not suffer from {\shock}, and are simply deanimated if they suffer a {\damagetest} equalling or exceeding their \statref{shock-threshold}.
They still suffer from {\exhaustion}, although they don't need sleep to recover from it; just rest.
Lastly, they cannot produce any venom or other such substances, so any benefit of a venomous bite, sting, or the like is lost.

\section{Types of Undead}

\undead{Zombie}{Zombies}{zombie}{
	A \undeadref{zombie} is the simplest reanimation possible; the corpse, fully clothed in its own flesh, is simply stood up and walked around as it is.
	Is it a clumsy creature, with most of the mind rotted away as well, and it only grows worse as the corpse rots further.
	
	A corpse reanimated as a \undeadref{zombie} loses 2 points from all attributes except \attref{might} and \attref{will}.
	Its \statref{speed} is halved.
	Its \statref{shock-threshold} increases by 2, however.
	If it could fly, it is now too clumsy to do so.
	
	A corpse reanimated as a \undeadref{zombie} is not healed of any {\damage}.
	The reanimation also requires that the corpse is unrotted.
	Furthermore, it does nothing to slow the rot.
	A \undeadref{zombie} that rots too far loses animation.
}

\undead{Ghoul}{Ghouls}{ghoul}{
	A \undeadref{ghoul} retains greater mental and physical faculties than a \undeadref{zombie}, but this comes at a dangerous price.
	A \undeadref{ghoul} is sustained only by consuming the flesh of its own kind.
	For example, a rabbit \undeadref{ghoul} must consume rabbit flesh, and a human \undeadref{ghoul} must consume human flesh.
	
	A corpse reanimated as a \undeadref{ghoul} loses 2 points from its \attref{ken}, \attref{wit}, and \attref{charm} scores.
	It retains the ability to fly, if it could in life.
	
	A corpse reanimated as a \undeadref{ghoul} is not healed of any {\damage}.
	However, an animated \undeadref{ghoul} may heal {\damage} by consuming the flesh of its own kind.
	An entire corpse is sufficient to restore {\damage} equal to its maximum \statref{shock-threshold}, with smaller portions restoring proportionally smaller amounts.
	A ghoul can consume an entire corpse in less than a minute, and there is no end to its hunger: it could consume corpses for hours on end without being sated.
	\capital{\undeadrefplural{ghoul}} created from partial corpses, such as with \featref{undead-head}, need only eat their own mass to count it as a full corpse.
	
	Reanimating a corpse as a \undeadref{ghoul} also requires that it is unrotted.
	While animated and fed at least one full corpse each week, however, a \undeadref{ghoul} does not continue to rot.
	
	Lastly, a \undeadrefpossessive{ghoul} unnatural hunger makes it harder to control than most undead.
	It must be fed a full corpse at least once a week, or it always breaks free of the \practitionerpossessive{necromancy} control.
	Even a \undeadref{ghoul} reanimated as a \undeadref{souled} is not immune to this: it goes insane with hunger after a week, and does not return to sanity until it has fed again.
}

\undead{Draugr}{Draugar}{draugr}{
	With all the moisture drawn out of a corpse, it is not only prevented from rotting, but can also be freed from the bloated clumsiness that afflicts \undeadrefplural{zombie}.
	The result is a \undeadref{draugr} (plural \undeadrefplural{draugr}).
	The better preservation also grants it a better memory and senses.
	
	A corpse reanimated as a \undeadref{draugr} loses 2 points from its \attref{wit}, \attref{charm}, and \attref{presence} scores.
	It retains the ability to fly, if it could in life.
	
	A corpse reanimated as a \undeadref{draugr} is not healed of any {\damage}.
	The reanimation requires that the corpse is unrotted, and it also must be {\embalmed} by desiccation (drying out).
	This usually is usually done using salt, but can happen naturally to creatures that die in deserts.
	
	To remain animated, the \undeadref{draugr} must be kept dry; water bloats the corpse, starts it rotting again, and immediately ends its animation.
	It might manage a 30 second sprint through light rain, but heavier rain is too much.
	Given a heavy leather coat, it might just about be able to travel through rain, but the moisture will still get to it in a couple of hours.
	The \undeadref{draugr} is aware of this limitation, and will avoid moisture when not under a \practitionerpossessive{necromancy} direct control.
	
	\undeadrefplural{draugr} are often used to guard ancient tombs, sealed inside where water cannot intrude.
}

\undead{Sea-Draugr}{Sea-Draugar}{sea-draugr}{
	The \undeadrefplural{sea-draugr} are, in many ways, the complete opposite of the regular \undeadrefplural{draugr}.
	Creatures of seas and lakes, they can only be animated from the corpses of those who died by drowning.
	They revel in their water-bloated flesh, lurking beneath the surface and dragging their prey to join them in their watery grave.
	
	A \undeadref{sea-draugr} is subject to the same rules as a regular \undeadref{draugr}.
	It also gains a swimming speed equal to its land speed, or half its flying speed, whichever is greater.
	
	However, instead of remaining dry, an \undeadref{sea-draugr} must remain soaked.
	It functions best when immersed in water, and begins to weaken about five minutes after it emerges.
	After about ten minutes, it dries out too much and completely loses animation.
	
	Regular wetting can extend this time; it might get half an hour in rain, or even an indefinite time if the rain is sufficiently torrential.
	\emph{Continuous} attention using \featref{willing-water-vapour}, or a couple of uses of \featref{willing-water-vapour-2} every 5 minutes, also suffices.
	However, it must be completely immersed for at least 8 hours each day.
	Just like a \undeadref{draugr}, a \undeadref{sea-draugr} is aware of this limitation, and will seek out water.
	
	As long as the corpse remains animated as a \undeadref{sea-draugr}, and sufficiently wetted, it will not continue to rot.
}

\undead{Skeleton}{Skeletons}{skeleton}{
	A \undeadref{skeleton} is the result of reanimating only the bones of a creature, the flesh rotted or carved away.
	The bones arrange themselves in the air, supported by nothing but the will of the animating witch, and the creature's conviction in its own shape.
	The result is a creature far less clumsy than a zombie, but not so resilient.
	
	A corpse reanimated as a \undeadref{skeleton} loses 2 points from all attributes except \attref{grace} and \attref{will}.
	Its \statref{shock-threshold} is also reduced by 2, in addition to the loss from the reduced \attref{might}.
	The mere bones of wings are not sufficient to allow it to fly, if it previously could.
	It also sinks in water, but may move along the bottom.
	
	Requiring only the bones, a \undeadref{skeleton} is not affected by most {\damage} sustained by the corpse.
	Only a critical success on a {\damagetest}, or an intentional effort after death, will typically have broken any bones.
	Likewise, it is not affected by {\damage} in the course of its undeath; any blow insufficient to scatter it across the floor is insufficient to scratch its bones.
	
	A \undeadref{skeleton} lasts a long time without decomposing; at least a decade, and even longer if kept dry.
}

\undead{Living Fossil}{Living Fossils}{fossil}{
	A \undeadref{fossil} is much like a \undeadref{skeleton}, except that the bones have been mineralised, impregnated with stone.
	The essence of earth permeates the creature, strengthening it.
	
	A \undeadref{fossil} uses the same rules as a \undeadref{skeleton}, except that its \attref{might} is not reduced.
	Additionally, fossilised bones do not decay, lasting millennia and more.
}

\undead{Blazing Skeleton}{Blazing Skeletons}{fire-skeleton}{
	A \undeadref{fire-skeleton} can only be made from the charred bones of a creature that died burning.
	Flames race across its bones as it walks, and its eye sockets blaze like the sun.
	It spreads destruction wherever it steps, leaving fire and ash in its wake.
	
	A \undeadref{fire-skeleton} mostly uses the same rules as a \undeadref{skeleton}, with a handful of differences.
	Firstly, the bones used to animate must be charred by fire.
	This should not be enough to totally destroy the bones, but even badly fire-damaged bones have no ill effects on the resulting \undeadref{fire-skeleton}.
	
	Secondly, the \undeadref{fire-skeleton} always burns, as long as it is animated.
	It burns without fuel, and without damaging itself---in fact, it is immune to all harm from fire and heat.
	The fire goes out if the \undeadref{fire-skeleton} loses animation.
	Conversely, the \undeadref{fire-skeleton} loses animation if the fire is extinguished, such as by being immersed in water.
	The fire is somewhat robust, however; it can survive moderate rain, simply causing the droplets to boil away.
	
	The \undeadrefpossessive{fire-skeleton} flames produce heat, and can combust things, just like normal flame.
	They will ignite most combustible materials they touch.
	A \undeadrefpossessive{fire-skeleton} {\unarmed} attacks also {\ignite} the target, at \dice{2}, or add 1 die of {\fire} to a target who is already burning.
}

\undead{Shade}{Shades}{shade}{
	While it is truly an undead, a reanimated corpse, a \undeadrefplural{shade} bears some resemblance to {\ghosts}.
	It shares some of their insubstantial nature, being a part of this realm, but not truly a creature of it.
	It always appears to be wreathed in shadow, and every part of it is dark: black or grey.
	Its facial features are indistinct, or even absent.
	
	Although it is formed from a creature's body, a \undeadref{shade} is insubstantial.
	It often finds its fingers passing straight through objects, like shadows flitting over them.
	Although this makes it harder to affect the world, it also affords the \undeadref{shade} a degree of protection.
	Swords can pass right through it, without even disturbing it.
	
	Light, however, brings the \undeadref{shade} form, clarity.
	This makes it vulnerable.
	Worse still, sunlight can burn it away entirely, destroying it.
	
	A \undeadref{shade} loses 2 points from its \attref{ken}, \attref{charm}, and \attref{presence} scores.
	Furthermore, its insubstantial nature causes it to lose 5 points from its \attref{might} score.
	However, it suffers no penalties to vision in low-light conditions, or even complete darkness.
	And, in complete darkness, it is immune to all physical harm.
	
	Light, even dim light, makes the \undeadref{shade} vulnerable again.
	If the light falls only on part of its body, only that part is vulnerable.
	Sunlight, however, is worse.
	The \undeadref{shade} suffers a \dice{5} {\damagetest} every {\round} that it is exposed to direct sunlight.
	Reducing exposure can reduce the number of dice rolled for the {\damagetest}, but even if it wrapped entirely in thick black cloth, leaving just its eyes exposed so that it might see causes it to suffer a \dice{1} {\damagetest} every round.
	
	Reanimating a corpse as a \undeadref{shade} requires that it is unrotted, but it does not continue to rot while it is animated.
	When the \undeadref{shade} is deanimated, it leaves the corpse fully corporeal again, albeit with a slightly dark pallor, and still affected by any {\damage} the \undeadref{shade} suffered.
	If the \undeadref{shade} is deanimated in sunlight, however, it burns away entirely, leaving no corpse---not even ash.
}

\undead{Wraith}{Wraiths}{wraith}{
	A \undeadref{wraith} is the invention of a foul \practitioner{necromancy} from a bygone era.
	She sacrificed dozens of people to the darkness of a \creatureref{stygian-nightshade}, then recovered their flayed corpses for reanimation.
	The result was a variety of \undeadref{shade} that carried the \creaturerefpossessive{stygian-nightshade} wicked claws, able to rend flesh despite their intangibility.
	The process has been refined since, and needs nothing more than a sprig of \creatureref{stygian-nightshade}.
	However, it still only works on the corpses of creatures that died violent deaths.
	
	A \undeadref{wraith} appears just like a \undeadref{shade}, and uses all the same rules, except for two differences.
	Firstly, the \undeadref{wraith} can affect the physical world with full force; it loses no \attref{might}.
	
	Secondly, it sprouts wicked claws of stygian darkness, increasing the number of dice it rolls for {\unarmed} {\damagetests}.
	A creature with an effective {\unarmed} attack rolls 1 additional die, to a maximum of 5.
	A human, or other creature with proper hands, always rolls 5 dice.
	A creature without effective {\unarmed} attacks gains no benefit.
}

\undead{Haunt}{Haunts}{souled}{
	%TODO: Souled, Haunts, or something else?
	A \undeadref{souled} is the result of necromancy that is beginning to lift itself from mere reanimation towards the ideal of resurrection.
	It is the result of imbuing a {\soul} into a more conventionally reanimated undead such as a \undeadref{zombie} or \undeadref{skeleton}.
	It is subject to the usual modifications to its statistics, as appropriate to the kind of reanimation.
	
	However, a \undeadref{souled} retains its memories, identity, and free will, and is not subject to the usual hunger.
	It is not controlled by the \practitioner{necromancy} who reanimated it.
	Furthermore, its \attref{might} and \attref{grace} are the only attributes subject to change; the other six are always unchanged.
	It retains all its skills and feats.
	It is still subject to all the benefits and detriments of its loss of biological processes, such as immunity to suffocation, disease, and potions.
	Lastly, it is still subject to usual rules for {\damage} and rotting, so may require \featref{undead-repair}.
	
	A creature can only be reanimated as a single \undeadref{souled} at a time, even when using feats such as \featref{undead-part}.
	
	Names for \undeadrefplural{souled} vary considerably, with many \undeadrefplural{souled} themselves finding the term unpleasant.
	They may refer to themselves as the Souled, or using some other name.
}
