\discipline{Ritual Magic}{ritual-magic}{Ritualist}{Ritualists}

\section{Ritual Circle Augmentation}
\seclabel{ritual-circle-augmentation}

Normally, it is impossible to draw overlapping or concentric \materialrefplural{ritual-circle}.
However, some \materialrefplural{ritual-circle} are simple and adaptable enough that they can be used to {\augment} an existing circle.
Sometimes this changes the properties of the existing circle, sometimes it simply serves as two independent but coinciding circles.
Various feats will grant the ability to perform such {\augmentation}, and lay out its effects.

\capital{\augmenting} a circle takes as long as scribing another circle of the same size as the one being {\augmented}.
Each circle may only have one {\augmentation}; any more leave too many overlapping lines, and render the entire mess dysfunctional.

\section{Feats}

\feat{Delayed Ritual}{ritual-magic-delay}{10}{
	\Anyfeat{ritual-magic}
}{
	When you perform a ritual granted by a \discref{ritual-magic} feat, you may delay its activation.
	When you complete the ritual, select a length of time to delay it by---up to 5 minutes.
	The ritual automatically takes effect at that time.
	
	If the ritual has a focus, a target, or a material that must remain for the ritual's duration---such as the iron rod in the \featref{ritual-teleport}, or the object to be launched by the \featref{ritual-launch}---it must remain in place throughout the entire delay period, or the ritual fails to take effect.
	Similarly, the \materialref{ritual-circle}, \materialref{stone-circle}, or \materialref{standing-stone} must remain unbroken, or standing, as appropriate.
	Other materials, however, are only necessary to cast the ritual in the first place and may be removed during the delay period.
}

\feat{Extended Ritual Delay}{ritual-magic-delay-2}{15}{
	\skillref[1]{ritual-magic},
	\featref{ritual-magic-delay}
}{
	When you use \featref{ritual-magic-delay}, you may delay the ritual's activation by up to 24 hours.
}

\feat{Indefinite Ritual Delay}{ritual-magic-delay-3}{10}{
	\skillref[2]{ritual-magic},
	\featref{ritual-magic-delay}
}{
	When you use \featref{ritual-magic-delay}, there is no limit to the maximum time you may delay the ritual's activation.
}

\feat{Circle of Containment}{circle-contain}{15}{
	\noprereq
}{
	\materials{A \circleref{medium} (no larger), a pinch of sugar, a single-edged knife}
	%Sugar draws things in.
	
	Performing this rite takes about fifteen minutes, involving tracing the knife around the perimeter and sprinkling the sugar.
	You may perform the rite from inside or outside the \materialref{ritual-circle}.
	
	At the completion of the rite, the \materialref{ritual-circle} is sealed.
	Objects and creatures inside the \materialref{ritual-circle} cannot pass outside of it, and no magic from inside may affect the outside.
	The barrier extends even into the {\mentalrealm}.
	Light and sound from inside can still leave, however.
	Air may diffuse slowly across the boundary, preventing suffocation or problems with pressure.
	Furthermore, nothing inside the circle may disturb the \materialref{ritual-circle} itself.
	
	The barrier created by this ritual extends straight up from the perimeter of the circle, to a height of twice the circle's diameter, and is capped off by a dome at the top.
	The protection of the circle itself, on the ground, also prevents tunnelling out.
	
	This effect lasts 24 hours, but ends early if the \materialref{ritual-circle} is broken.
	The rite to renew the effect cannot be performed while the effect remains in place.
}

\feat{Circle of Exclusion}{circle-exclude}{15}{
	\noprereq
}{
	\materials{A \circleref{medium} (no larger), a pinch of salt, a single-edged knife}
	%Salt is a preservative, driving foulness away.
	
	This functions as \featref{circle-contain}, except the other way.
	Objects, creatures, and magical effects can pass out of the \materialref{ritual-circle}, but not into it, and the \materialref{ritual-circle} may be disturbed only from the inside.
}

\feat{Circle of Severance}{circle-contain-exclude}{10}{
	\featref{circle-contain} and \featref{circle-exclude}
	%The unconventional use of "and" is intentional here, in light of the fact that so many subsequent feats use "or".
}{
	\materials{A \circleref{medium} (no larger), a pinch of sugar, a pinch of salt, a double-edged knife}
	
	This functions as \featref{circle-contain}, except in both directions.
	Objects, creatures and magical effects cannot pass into or out of the \materialref{ritual-circle}, and the \materialref{ritual-circle} cannot be disturbed in any fashion.
}

\feat{Fast Barrier}{circle-barrier-speed}{20}{
	\skillref[2]{ritual-magic},
	\featref{circle-contain} or \featref{circle-exclude}
}{
	You may perform the rite to establish a \featref{circle-contain}, \featref{circle-exclude}, or \featref{circle-contain-exclude} in just 1 minute, assuming you have the feat to establish one in the first place.
	You still have to trace the entire perimeter of the circle with the knife, so you can never complete the rite faster than you can perform this.
}

\feat{Renew Barrier}{circle-barrier-renew}{10}{
	\featref{circle-contain} or \featref{circle-exclude}
}{
	You may repeat the rite to establish a \featref{circle-contain}, \featref{circle-exclude}, or \featref{circle-contain-exclude} while the effect of one is ongoing, assuming you have the feat to establish one in the first place.
	You may do so from inside or outside of the \materialref{ritual-circle}.
	The new duration runs from the completion of the repeated rite, though this cannot be used to reduce the remaining duration.
}

\feat{Stabilise Barrier}{circle-barrier-duration}{20}{
	\skillref[1]{ritual-magic},
	\featref{circle-barrier-renew}
}{
	You have learned to tweak the stability of your circles' barriers, making them endure longer, or collapse faster.
	When you establish or renew a \featref{circle-contain}, \featref{circle-exclude}, or \featref{circle-contain-exclude}, you may select its duration.
	You can make it last mere minutes, many years, or even indefinitely.
	
	There are limits on the stability of these circles, however.
	They are sustained by the natural magic of creatures on their open sides, and their duration cannot be extended more than 24 hours if there are no such creatures.
	This poses no problem a \featref{circle-contain}---there are always creatures outside.
	A \featref{circle-exclude} can only last more than 24 hours as long as there is a creature inside it, and collapses 24 hours after the last creature leaves it.
	A \featref{circle-contain-exclude} has no open side---its duration can never exceed 24 hours.
}

\feat{Miniature Barrier}{circle-barrier-small}{15}{
	\featref{circle-contain} or \featref{circle-exclude}
}{
	You may create a \featref{circle-contain}, \featref{circle-exclude}, or \featref{circle-contain-exclude} using a \circleref{small}, assuming you have the feat to establish one at all.
}

\feat{Maximise Barrier}{circle-barrier-large}{15}{
	\featref{circle-contain} or \featref{circle-exclude}
}{
	You may create a \featref{circle-contain}, \featref{circle-exclude}, or \featref{circle-contain-exclude} using a larger \materialref{ritual-circle}, assuming you have the feat to establish one at all.
	There are no upper limits to the size of the \materialref{ritual-circle} you may use, except those imposed by practicality.
	However, establishing or renewing the effect still requires tracing the perimeter with the relevant knife and sprinkling the correct substance.
	Very large perimeters may require longer than 15 minutes, and a large quantity of substance to sprinkle.
}

\feat{Circle Breaker}{circle-barrier-break}{10}{
	\featref{circle-contain} or \featref{circle-exclude}
}{
	Just as you can close a circle, you can open it.
	This way requires a knife of the same kind used to cast the circle in the first place---double-edged for a \featref{circle-contain-exclude}, or single-edged for the others.
	As an {\action}, you can touch the knife to the barrier of your \emph{own} \featref{circle-contain}, \featref{circle-exclude}, or \featref{circle-contain-exclude}, and end the ritual's effect.
	You may do so from either side of the barrier.
}

\feat{Circle Bypass}{circle-barrier-bypass}{10}{
	\featref{circle-barrier-break}
}{
	When you use \featref{circle-barrier-break}, you can open you circle's barrier temporarily, instead of breaking it entirely.
	If you do so, the barrier comes back up after 1 {\round}.
	This doesn't affect its duration; it will still end at the time it normally would have done.
}

\feat{Bypass Flicker}{circle-barrier-bypass-short}{10}{
	\skillref[1]{ritual-magic},
	\featref{circle-barrier-bypass}
}{
	When you use \featref{circle-barrier-bypass}, you can flicker the barrier down only very briefly, reducing the chance someone can follow you.
	If you do so, the barrier automatically comes back up at the end of the {\turn}.
	Anyone prepared to pass through---having taken the \actionref{ready} {\action}, in {\structuredtime}---can still do so before it comes back up.
}

\feat{Bypass Holder}{circle-barrier-bypass-long}{10}{
	\skillref[1]{ritual-magic},
	\featref{circle-barrier-bypass}
}{
	When you use \featref{circle-barrier-bypass}, you can hold the barrier down for longer, allowing more time to pass through.
	The barrier remains down as long you hold the knife in it---possibly for minutes, or hours.
	You must remain holding the knife; it doesn't work if you let it go.
	Unless you use \featref{circle-barrier-bypass-short}, the barrier still always stays down at least one {\round}.
}

\feat{Back Door}{circle-barrier-bypass-door}{10}{
	\skillref[1]{ritual-magic},
	\featref{circle-barrier-bypass}
}{
	\capital{\featref{circle-barrier-bypass}} compromises the circle's entire perimeter, a risky prospect.
	With more care, you can just cut a smaller door in the barrier.
	
	When you use \featref{circle-barrier-bypass}, you can trace the knife around a door in the barrier.
	The door may be any shape, although larger doors may require more than one {\action} to trace.
	The barrier only drops within this door.
}

\feat{Circle Intrusion}{circle-barrier-break-2}{20}{
	\skillref[2]{ritual-magic},
	\featref{circle-barrier-break}
}{
	You can use \featref{circle-barrier-break}---and \featref{circle-barrier-bypass}, if you have it---against circles cast by other witches.
	However, without your own back door in the circle, doing so is slow, and dangerous.
	
	Instead of an {\action}, breaking or bypassing a circle in this way takes 10 minutes, unless rushed.
	It \emph{always} requires a \testtype{ken}{ritual-magic} {\test}.
	In addition to the usual modifiers from improvised equipment, rushing, and the like, the {\tn} of this {\test} should be affected by the \skillref{ritual-magic} skill of the witch who cast the circle.
	
	Failing the {\test} not only means you fail to break or bypass the barrier; it violently destroys the knife used to cut the barrier, dealing a \dice{5} {\damagetest} to you as you hold it.
}

\feat{Barrier Augmentation}{circle-barrier-augment}{15}{
	\skillref[1]{ritual-magic},
	\featref{circle-contain} or \featref{circle-exclude}
}{
	The \featref{circle-contain} and \featref{circle-exclude} are very simple \materialrefplural{ritual-circle}, and it is a simple matter to adapt them as {\augmentations} to existing circles.
	You may scribe the \featref{circle-contain} or \featref{circle-exclude} as an {\augmentation} to an existing \materialref{ritual-circle}, provided you can use these circles in the size of the circle you are {\augmenting}.
	
	The {\augmentation} functions as a normal circle, and the rite to active it can be performed independently of any rites using the {\augmented} circle.
	However, while the barrier created by one of these circles is active, any rites using the {\augmented} circle must be performed from the \emph{open} side of the barrier circle---from outside a \featref{circle-contain} or from inside a \featref{circle-exclude}.
	Additionally, the {\augmentation} provides the same protection to the {\augmented} circle as it provides to itself, as long as it is active.
	
	The \featref{circle-contain}, \featref{circle-exclude} and \featref{circle-contain-exclude} are all incompatible, and these {\augmentations} cannot be used upon these circles.
}

\feat{Bypass Casting}{circle-barrier-augment-casting}{10}{
	\skillref[1]{ritual-magic},
	\featref{circle-barrier-augment},
	\featref{circle-barrier-bypass}
}{
	Normally, a barrier {\augmentation} interferes with casting the {\augmented} circle's rite; you can only do so from the open side of the barrier.
	But, armed with your \featref{circle-barrier-bypass}, you've learned to bypass this.
	
	If you cast a barrier circle using \featref{circle-barrier-augment}, you may perform the {\augmented} circle's rite from inside or outside the barrier circle.
	Just as for \featref{circle-barrier-break}, this requires a knife of the same kind as was required to cast the circle in the first place.
	
	Like \featref{circle-barrier-bypass}, this only works if \emph{you} cast the barrier circle.
	If you also have \featref{circle-barrier-break-2}, you may attempt this on barrier circles cast by other witches, requiring the usual {\test} and with the usual consequences for failure.
}

\feat{Severance Augmentation}{circle-barrier-augment-2}{10}{
	\skillref[2]{ritual-magic},
	\featref{circle-barrier-augment},
	\featref{circle-contain-exclude}
}{
	You may use the \featref{circle-contain-exclude} as an {\augmentation}, following the same rules as \featref{circle-barrier-augment}.
	Note that a \featref{circle-contain-exclude} has no open side, so you cannot cast the {\augmented} circle's rite while the barrier is up, without \featref{circle-barrier-augment-casting}.
}

\feat{Slow Time}{circle-time-slow}{10}{
	\featref{circle-exclude}
}{
	You may cause your \featref{circle-exclude} to leak time, slowing time inside it.
	If you do so, time flows up to 10 times more slowly inside the circle than outside it.
	You must activate this effect when you cast the circle, and cannot alter it afterwards.
	
	The duration of the circle is set from the perspective of the faster time.
	Therefore, by default, the circle lasts 24 hours from the faster perspective, and 2 hours 24 minutes from the slower perspective.
}

\feat{Fast Time}{circle-time-fast}{20}{
	\featref{circle-contain}
}{
	You may cause your \featref{circle-contain} to accumulate time, speeding up.
	This functions as \featref{circle-time-slow}, except that it modifies a \featref{circle-contain}, and time runs up to 10 times \emph{faster} inside.
}

\feat{Sever Time}{circle-time-either}{10}{
	\skillref[1]{ritual-magic},
	\featref{circle-time-slow} or \featref{circle-time-fast}
}{
	You may use \featref{circle-time-slow} or \featref{circle-time-fast} to slow down or speed up time within a \featref{circle-contain-exclude}.
	You must have the appropriate feat to cause the corresponding manipulation of time.
}

\feat{Ritual Fire}{ritual-fire}{10}{
	\noprereq
}{
	\materials{A \circleref{small} (no larger), a flame}
	
	This ritual takes about a minute, and lights a tiny flame into a merry campfire.
	The fire is large enough to fill a \circleref{small}, and produces enough heat to cook food, melt snow or keep a group of campers warm.
	It burns without fuel even in the coldest of conditions, though will gladly consume fuel thrown into it, or spread to nearby flammable material if it is available.
	
	The fire cannot be naturally extinguished and lasts 8 hours, or until the \materialref{ritual-circle} is broken.
	Any fires lit from the ritual fire and burning on regular fuel will continue afterwards.
}

\feat{Ritual Forge}{ritual-fire-2}{10}{
	\featref{ritual-fire}
}{
	You may use a \circleref{medium} (no larger) to create a \featref{ritual-fire}.
	When you do so, the fire fills the larger circle.
	It provides enough heat to keep a sizeable crowd warm on a winter night.
	Furthermore, it burns hot enough to be used as an iron forge, or a pottery kiln.
}

\feat{Pillar of Flame}{ritual-fire-burst}{15}{
	\skillref[2]{ritual-magic},
	\featref{ritual-fire}
}{
	Instead of allowing your \featref{ritual-fire} to burn its heat out over many hours, you may release it all in one moment of glorious conflagration.
	As an {\action}, you may throw \herb{dried corn kernels}{2} onto a \featref{ritual-fire}.
	On the following {\round}, the fire erupts in a massive pillar of flame.
	This extinguishes the \featref{ritual-fire}.
	
	The pillar has enough heat to reduce wood to ash, and enough force to blow masonry apart.
	Any creature caught in the pillar comes to a quick end.
	However, the blast is contained entirely within the radius of the \materialref{ritual-circle}; those outside feel nothing more than an uncomfortable wave of heat.
	
	The pillar from a \featref{ritual-fire} reaches 10 metres high: enough to blow through a couple of storeys of a house.
	The pillar from a \featref{ritual-fire-2} reaches about 100 metres, and can be seen from more than 30 kilometres away, particularly at night.
}

\feat{Fire on the Wall}{ritual-fire-vertical}{15}{
	\featref{ritual-fire}
}{
	You may draw and use the \materialref{ritual-circle} for a \featref{ritual-fire} on non-horizontal surfaces, even vertical surfaces, or on a ceiling.
	The flames ``rise'' away from the circle, regardless of whether that direction is actually up.
	Similarly, if used with \featref{ritual-fire-burst}, the pillar erupts perpendicular to the circle; not necessarily upwards.
	This can make it quite easy to use as a weapon.
}

\feat{Rite of Refrigeration}{ritual-cold}{10}{
	\noprereq
}{
	\materials{A \circleref{medium}, a \materialref{cold-iron} slab at least the size of your palm}
	
	This ritual takes about a minute to initiate.
	After that, it runs for 8 hours, until the \materialref{ritual-circle} is broken, or until the \materialref{cold-iron} slab is removed from the centre of the \materialref{ritual-circle}.
	
	While the ritual is active, the air temperature inside the \materialref{ritual-circle} drops to just below freezing, unless it is already lower.
	The air temperature cannot be raised in any way as long as the ritual remains active.
	Objects and creatures inside the circle are gradually chilled by the freezing air, but not affected by the ritual itself.
	Additionally, all fires inside the circle are extinguished, including fires that enter the circle while the ritual is active.
}

\feat{Mini Fridge}{ritual-cold-small}{10}{
	\featref{ritual-cold}
}{
	You may use the \featref{ritual-cold} with a \circleref{small}.
}

\feat{Snap Freeze}{ritual-cold-burst}{15}{
	\skillref[2]{ritual-magic},
	\featref{ritual-cold}
}{
	You can accelerate the \featref{ritual-cold}, sucking all the air from an area in an instant.
	As an {\action}, you may flick water droplets onto the iron slab powering a \featref{ritual-cold}.
	The snap freeze occurs on the following {\round}, ending the ritual's effect as it does so.
	
	Everything inside the \materialref{ritual-circle} is frozen.
	All water turns to ice, and frost forms on every surface.
	Any creature caught inside becomes a frozen statue; thoroughly dead.
	The temperature change is confined to within the circle; those outside feel nothing but a brief chill wind.
}

\feat{Rite of Excavation}{ritual-dig}{10}{
	\noprereq
}{
	Leveraging the right \materialref{ritual-circle}, you can excavate tonnes of dirt in seconds.
	
	\materials{A \circleref{small}, a trowel}
	
	The ritual takes only one {\action}, as you excavate a trowelful of the ground within the circle.
	This instantly excavates the ground inside the circle to a depth of 2 metres, destroying the material therein.
	This always destroys or disrupts the \materialref{ritual-circle} itself---for example, a circle scratched in the dirt will vanish as the dirt does, but a circle laid in rope above the dirt will fall, out of shape, into the hole.
	
	This ritual can excavate not only dirt, but any material you can scribe a \materialref{ritual-circle} upon, and dig a trowelful of.
	It will only excavate material of the same kind that you dig with the trowel.
	As such, it will not excavate solid rock, but can safely be used to dig for buried treasure.
	Furthermore, the ritual can never be used to excavate living creatures or plants.
}

\feat{Extended Excavation}{ritual-dig-2}{15}{
	\skillref[1]{ritual-magic},
	\featref{ritual-dig}
}{
	By using a shovel instead of a trowel---or using \featref{willing-tools} on your trowel---you may dig further with the \featref{ritual-dig}.
	If you do this, the ritual excavates to a depth of 20 metres.
}

\feat{Rite of Refill}{ritual-dig-reverse}{10}{
	\featref{ritual-dig}
}{
	The material destroyed by the \featref{ritual-dig} is not gone forever.
	The potential to return remains, and you can use another ritual to call it back into existence, and refill the hole.
	
	\materials{A \materialref{ritual-circle} filling the bottom of the hole, a trowel, a trowelful of the same material originally excavated}
	
	This ritual can only be performed in a hole originally excavated using the \featref{ritual-dig}.
	It requires only an {\action}, as you throw the trowelful of material into the \materialref{ritual-circle}.
	Immediately, all the material that was originally excavated returns, filling the hole.
	Anything inside the hole is buried, so you should perform the ritual from outside the hole.
	The \materialref{ritual-circle} is also buried, rendering it useless.
	
	Refilling a hole created using \featref{ritual-dig-2} requires a shovel, and a shovelful of material.
}

\feat{Rite of Launching}{ritual-launch}{10}{
	\noprereq
}{
	\materials{A \circleref{small}, a feather, the object or creature to be launched}
	
	As an {\action}, you tickle the object or creature to be launched---which must be in the centre of of the \materialref{ritual-circle}---with the feather.
	It is immediately launched 10 metres directly upwards.
	It falls again afterwards, unless it can do something to prevent it.
	The object or creature you launch cannot weight more than 10 kilograms.
}

\feat{Angular Launch}{ritual-launch-direction}{15}{
	\featref{ritual-launch}
}{
	A slight adjustment to the \materialref{ritual-circle} used in the \featref{ritual-launch}---essentially an arrow pointing outwards---allows the rite to launch the target at an angle.
	It flies just as far upwards, but also off in one direction.
	On flat ground, it travels a horizontal distance equal to 4 times the height it is launched to.
	
	The direction of the launch is determined by the shape of the \materialref{ritual-circle} used.
	The shape of the \materialref{ritual-circle} even determines whether the launch is directional at all---you cannot launch straight upwards with a circle configured for directional launch.
	Adjusting the direction of launch doesn't require rescribing the entire circle, however.
	The adjustment can be made comfortably in a minute, and can easily be rushed faster, especially for smaller circles.
}

\feat{Large Launch}{ritual-launch-weight}{15}{
	\skillref[1]{ritual-magic},
	\featref{ritual-launch}
}{
	By using a \circleref{medium} in the \featref{ritual-launch}, you can launch an object or creature weighing up to 100 kilograms, including a human.
}

\feat{Immense Launch}{ritual-launch-weight-2}{15}{
	\skillref[2]{ritual-magic},
	\featref{ritual-launch-weight}
}{
	By using a \circleref{large} in the \featref{ritual-launch}, you can launch an object or creature weighing up to 10 tonnes---even a small \materialref{standing-stone}.
}

\feat{High Launch}{ritual-launch-distance}{10}{
	\featref{ritual-launch}
}{
	If you use a \circleref{medium} in the \featref{ritual-launch}, you can launch the object or creature to a height of 100 metres.
}

\feat{Sky-Scraping Launch}{ritual-launch-distance-2}{10}{
	\skillref[1]{ritual-magic},
	\featref{ritual-launch-distance}
}{
	If you use a \circleref{large} in the \featref{ritual-launch}, you can launch the object or creature to a height of 1 kilometre.
}

\feat{Launch Efficiency}{ritual-launch-smaller}{20}{
	\skillref[2]{ritual-magic},
	\featref{ritual-launch-weight} or \featref{ritual-launch-distance}
}{
	You can squeeze more power out of a small circle when using the \featref{ritual-launch}.
	When you use a \circleref{small}, you can gain the benefit of \featref{ritual-launch-weight} or \featref{ritual-launch-distance}, but not both.
	When you use a \circleref{medium}, you can gain the benefit of \featref{ritual-launch-weight-2} or \featref{ritual-launch-distance-2}, but not both.
	This never allows you to gain the benefit of a feat you do not have.
}

\feat{Circle of Cushioning}{circle-safe-fall}{15}{
	\featref{ritual-launch}
}{
	A passive, reactive variant of the \featref{ritual-launch}, this \materialref{ritual-circle} cushions anyone who falls upon it, protecting them.
	This needs no ritual to active it; the existence of the \materialref{ritual-circle} itself is enough.
	The \materialref{ritual-circle} may be any size; \circlerefbare{small} or larger.
	
	Any creature or object which lands upon the surface where the circle is scribed, within the circle, is protected from any harm as a result of the fall.
	Even a porcelain vase won't so much as chip.
	The circle, and the surface on which it is scribed, are also protected from falling objects.
	For example, even a massive boulder won't knock through floorboards if it lands within the circle.
}

\feat{Circle of Rebound}{circle-bounce}{10}{
	\skillref[1]{ritual-magic},
	\featref{circle-safe-fall}
}{
	Combining the \featref{circle-safe-fall} and the \featref{ritual-launch}, you have have created the \featref{circle-bounce}.
	Much like the \featref{circle-safe-fall}, this has no associated ritual---it works passively.
	It can be scribed as a ritual circle of any size; \circlerefbare{small} or larger.
	It provides all the benefits of a \featref{circle-safe-fall}; objects and creatures landing within suffer no harm, and harm neither the circle nor the ground.
	
	However, instead of stopping the fall with a safe landing, this circle rebounds the falling object or creature back into the air.
	It rebounds to the same height it fell from.
	Horizontal velocity is conserved, not reversed, so if it falls in sideways, it rebounds out still going in the same direction.
}

\feat{Cushioning Augment}{circle-safe-fall-augment}{10}{
	\featref{circle-safe-fall}
}{
	You may scribe the \featref{circle-safe-fall}---or the \featref{circle-bounce}, if you have it---as an {\augmentation} to another \materialref{ritual-circle}.
}

\feat{Rite of Transposition}{ritual-teleport}{15}{
	\noprereq
}{
	When even a broomstick isn't fast enough, there are yet faster ways to travel.
	It is possible to connect a pair of locations, many miles apart, and then to take a single step from one to the other.
	Joining two points like this requires them to be well-anchored, however, with a pair of structures to fix them properly in space.
	For a novice, only \materialrefplural{stone-circle} will suffice.
	
	\materials{A \materialref{stone-circle} as the point of departure, a \materialref{stone-circle} as the point of arrival, an iron rod driven halfway into the ground at the point of departure}
	
	The ritual takes an hour, and at its conclusion you and your familiar may step from the point of departure to the point of arrival.
	Anything you are wearing or carrying is transported with you, except other creatures.
	Upon arrival, the iron rod used in the ritual is buried halfway in the ground at the destination, now upside down.
	
	The ritual is performed at the departure point, but you must be able to picture the arrival point clearly.
	As such, you must have seen it at some point previously.
	The \materialref{stone-circle} at the arrival point must still be standing.
	If it is not, you receive no warning until you attempt the passage at the conclusion of the ritual.
	Then, you must make a {\test} to avoid the disastrous consequences of an incomplete transposition.
}

\feat{Rapid Transposition}{ritual-teleport-speed}{10}{
	\skillref[1]{ritual-magic},
	\featref{ritual-teleport}
}{
	You may perform the \featref{ritual-teleport} in just five minutes.
}

\feat{Transpose Company}{ritual-teleport-others}{15}{
	\skillref[1]{ritual-magic},
	\featref{ritual-teleport}
}{
	Travelling by yourself is lonely, but you've got the hang of bringing company.
	When you perform the \featref{ritual-teleport}, you may leave the connection open.
	You need not pass through yourself.
	For the next {\round}, others may take the step from the point of departure to the point of arrival, also bringing everything they carry.
	Objects must be carried by just one person; if someone passes through carrying just one end of an object, the object doesn't usually survive having its two ends several miles apart.
	
	The connection can be severed early by pulling the iron rod out of the ground, at either end.
	It winds up at whichever end it is pulled out from, although remains only at the arrival end if the connection closes naturally.
}

\feat{Portal}{ritual-teleport-portal}{10}{
	\skillref[2]{ritual-magic},
	\featref{ritual-teleport-others}
}{
	Bringing company on your trips is all well and good, but sometimes your luggage is too heavy to carry.
	When you perform the \featref{ritual-teleport}, you may open the connection as a portal.
	It appears as a shimmering hole in the air, with the other side visible but heavily distorted.
	Creatures can pass from one side to the other as normal, but large objects---anything that will fit in the \materialref{stone-circle}---may also be carried, pushed, or rolled through.
	The connection still only remains open for the usual length of time, however, and anything that is only halfway through when it closes suffers violent disassembly.
}

\feat{Round Trip}{ritual-teleport-return}{10}{
	\skillref[1]{ritual-magic},
	\featref{ritual-teleport-others}
}{
	When you leave open the connection in a \featref{ritual-teleport}, you may make it bidirectional.
	If you do so and the connection closes naturally, the iron rod is left in the ground at both ends.
	There is still only one, however, and pulling it out at one end causes the ground to swallow it at the other.
}

\feat{Extended Transposition}{ritual-teleport-duration}{10}{
	\skillref[2]{ritual-magic},
	\featref{ritual-teleport-others}
}{
	When you leave the connection open in a \featref{ritual-teleport}, you may leave it open for an hour.
	It still ends early if the rod is pulled out.
	Each \materialref{stone-circle}---or \materialref{standing-stone}, using \featref{ritual-teleport-no-circle-arrive} and \featref{ritual-teleport-no-circle-depart}---may only support one connection at a time.
}

\feat{Indefinite Transposition}{ritual-teleport-duration-2}{20}{
	\skillref[3]{ritual-magic},
	\featref{ritual-teleport-duration}
}{
	When you recklessly punch holes in the fabric of space, they stay punched.
	When you leave the connection open in a \featref{ritual-teleport}, you may leave it open indefinitely.
	It still ends early if the rod is pulled out.
}

\feat{Immediate Transposition}{ritual-teleport-speed-2}{15}{
	\skillref[3]{ritual-magic},
	\featref{ritual-teleport-speed}
}{
	You've figured out which parts of the \featref{ritual-teleport} are actually the most important, and you've reduced it to a mere few flicks of the wrist as you drive the rod into the ground.
	You may perform the \featref{ritual-teleport} as an {\action}.
	Furthermore, you may pass through and remove the rod from the ground again, or even \actionref{ready} to remove the rod from the ground, as part of the same {\action}.
}

\feat{Arrival Point}{ritual-teleport-no-circle-arrive}{20}{
	\skillref[2]{ritual-magic},
	\featref{ritual-teleport}
}{
	The \featref{ritual-teleport} needs solid structures at both ends to anchor the connection.
	One \materialref{standing-stone} doesn't have the mass that a ring of them does, but it's still significant enough to receive a connection.
	
	You may use a \materialref{standing-stone} instead of a \materialref{stone-circle} for the \emph{arrival} point of the \featref{ritual-teleport}.
	You can still create a bidirectional connection, if you have \featref{ritual-teleport-return}, but must create it from the \materialref{stone-circle}.
	You can use \materialrefplural{standing-stone} at both ends if you also have \featref{ritual-teleport-no-circle-depart}.
}

\feat{Departure Point}{ritual-teleport-no-circle-depart}{20}{
	\skillref[2]{ritual-magic},
	\featref{ritual-teleport}
}{
	Performing a ritual without a circle is pretty funny business, but it still seems to work if you run around a \materialref{standing-stone}.
	
	You may use a \materialref{standing-stone} instead of a \materialref{stone-circle} for the \emph{departure} point of the \featref{ritual-teleport}.
	You can still create a bidirectional connection, if you have \featref{ritual-teleport-return}, but must create it from the \materialref{standing-stone}.
	You can use \materialrefplural{standing-stone} at both ends if you also have \featref{ritual-teleport-no-circle-arrive}.
}

\feat{Rite of Invisibility}{ritual-hide}{10}{
	\noprereq
}{
	\materials{A \circleref{medium}, a cloak}
	
	Performing this rite takes a minute, and requires you to stand in the circle and don the cloak.
	At the conclusion, you---along with the cloak, and your hat---become invisible as long as you remain in the unbroken circle wearing the cloak.
	You may also conceal other objects beneath the cloak, but any objects outside remain visible.
	You can still see yourself and your equipment.
}

\feat{Bestow Invisibility}{ritual-hide-others}{10}{
	\featref{ritual-hide}
}{
	You may perform the \featref{ritual-hide} upon other people and creatures.
	This requires one cloak per person, but several people can share a circle.
	People affected by the ritual, within the same circle, can see each other just as they can see themselves.
}

\feat{Rite of Concealment}{ritual-hide-all}{10}{
	\skillref[1]{ritual-magic},
	\featref{ritual-hide-others}
}{
	\materials{A \circleref{medium} (no larger), a sheet}
	
	Performing this rite takes a minute, and requires throwing the sheet over something inside.
	The sheet itself, and anything covered by it, becomes invisible as long as it is within the unbroken circle.
	The ground beneath, including the \materialref{ritual-circle} itself, remains visible.
	
	Creatures and people underneath the sheet can still see as normal within it.
	However, their vision of anything outside is obstructed by the sheet itself, and lifting it to peek outside reveals them.
}

\feat{Self-Concealing Circle}{ritual-hide-circle}{20}{
	\skillref[1]{ritual-magic},
	\featref{ritual-hide}
}{
	Turning a person invisible is of rather limited use as long as they are standing in a perfectly visible magical circle.
	You've finally figured out a way around that.
	When you perform the \featref{ritual-hide} or \featref{ritual-hide-all}, you may also turn the \materialref{ritual-circle} itself invisible for as long as the ritual lasts.
}

\feat{Combined Concealment}{ritual-hide-combine}{10}{
	\skillref[1]{ritual-magic},
	\featref{ritual-hide-all}
}{
	You can leverage the similarity between the \featref{ritual-hide} and the \featref{ritual-hide-all}, and may use a \materialref{ritual-circle} scribed in either design for either rite.
	Notably, this lets you use the same \materialref{ritual-circle} for both rites simultaneously.
	This does not affect the size of the design you can use, only the design.
}

\feat{Total Concealment}{ritual-hide-other-senses}{10}{
	\featref{ritual-hide}
}{
	When you perform the \featref{ritual-hide} or \featref{ritual-hide-all}, you may extend the effect to senses other than sight.
	Smells and sounds emitted by the affected creatures and objects are not detectable to anyone they are invisible to.
	Other creatures under the effect of the rite, or inside the sheet, respectively, can still hear and smell them, just as they can see them.
	This does not extend to touch; everyone and everything still exists and there is nothing you can do to prevent people bumping into them.
	
	A creature under the effect of the \featref{ritual-hide} may intentionally circumvent this effect when they speak, sing, whistle, or the like.
}

\feat{Hiding Nook}{ritual-hide-small}{10}{
	\featref{ritual-hide}
}{
	You may use a \circleref{small} for the \featref{ritual-hide} and \featref{ritual-hide-all}.
	It is incredibly difficult to fit more than four people in a \circleref{small}, and even then they'd better be very close friends.
}

\feat{Hiding Houses}{ritual-hide-large}{15}{
	\skillref[1]{ritual-magic},
	\featref{ritual-hide-all}
}{
	You may perform the \featref{ritual-hide-all} using a larger \materialref{ritual-circle}, though you may soon find yourself needing a very large sheet.
}

\feat{Concealing Augmentation}{ritual-hide-augment}{20}{
	\skillref[1]{ritual-magic},
	\featref{ritual-hide-circle}
}{
	Applying what you've learned from the \featref{ritual-hide}, you've learned to conceal your other \materialrefplural{ritual-circle}.
	You have learned an {\augmentation} that hides the {\augmented} circle, and itself, from sight.
	
	The {\augmented} circle does not function while it is hidden in this way---visibility is important to \materialrefplural{ritual-circle}.
	But it is a simple matter to break the {\augmentation} with a scratch through one line, and then to reconnect the same line when necessary.
}

\feat{Rite of Reduction}{ritual-shrink}{15}{
	\noprereq
}{
	\materials{A \circleref{small} (no larger) and a \circleref{medium} (no larger), touching at the perimeter; the object to be shrunk}
	
	This ritual takes about 15 minutes.
	At the conclusion, the object---which must be entirely within the \circleref{medium}---disappears and reappears in the centre of the \circleref{small}.
	It is half the size in all dimensions, and an eighth of the weight.
	It remains shrunk for an hour, then expands back to its original size over the course of about a minute.
	
	This ritual cannot be used on the same object again until it has returned to its original size.
	Using the \featref{ritual-grow} upon the object cancels both rituals' effects, returning it to its original size.
}

\feat{Rite of Enlargement}{ritual-grow}{15}{
	\noprereq
}{
	\materials{A \circleref{small} (no larger) and a \circleref{medium} (no larger), touching at the perimeter; the object to be enlarged}
	
	This ritual takes about 15 minutes.
	At the conclusion, the object---which must be entirely within the \circleref{small}---disappears and reappears in the centre of the \circleref{medium}.
	It is twice the size in all dimensions, and eight times the weight.
	It remains enlarged for an hour, then shrinks back to its original size over the course of about a minute.
	
	This ritual cannot be used on the same object again until it has returned to its original size.
	Using the \featref{ritual-shrink} upon the object cancels both rituals' effects, returning it to its original size.
}

\feat{Larger Scaling}{ritual-size-larger}{10}{
	\featref{ritual-shrink} or \featref{ritual-grow}
}{
	The sizes of the \materialrefplural{ritual-circle} used in the \featref{ritual-shrink} and \featref{ritual-grow} limit the size of the objects that may be affected.
	Enlarging both circles, you may affect larger objects.
	You may substitute the \circleref{small} for a \circleref{medium} (no larger), and the \circleref{medium} for a \circleref{large} (no larger), in the \featref{ritual-shrink} or \featref{ritual-grow}.
	If you make one substitution, you must make both.
}

\feat{Drastic Scaling}{ritual-size-more}{15}{
	\skillref[2]{ritual-magic},
	\featref{ritual-size-larger}
}{
	Building on \featref{ritual-size-larger}, you may use a \circleref{small} (no larger) and a \circleref{large} (no larger) in the \featref{ritual-shrink} or \featref{ritual-grow}, and in doing so, double the ritual's effect.
	The target's size changes by a factor of 4 in all dimensions, and its weight changes by a factor of 64.
}

\feat{Longer Scaling}{ritual-size-duration}{15}{
	\skillref[1]{ritual-magic},
	\featref{ritual-shrink} or \featref{ritual-grow}
}{
	Altering the size of an object is inherently unstable magic, and the object will return to its natural state before long.
	But with care, you can make it hold on longer.
	When you perform the \featref{ritual-shrink} or \featref{ritual-grow}, you may make its effect last 24 hours.
	After this 24 hours, the object still takes about a minute to gradually revert to its original size.
}

\feat{Indefinite Scaling}{ritual-size-duration-2}{25}{
	\skillref[3]{ritual-magic},
	\featref{ritual-size-duration}
}{
	You are an unrivalled expert at size changing magic.
	When you perform the \featref{ritual-shrink} or \featref{ritual-grow}, you may make its effect last indefinitely.
	Remember that the same rite cannot be performed again on the object until it has returned to its original size---and it can only be returned to its original size by performing the opposite rite upon it.
}

\feat{Unified Scaling}{ritual-size-bidirectional}{10}{
	\featref{ritual-shrink},
	\featref{ritual-grow}
}{
	You have consolidated the designs of the \materialrefplural{ritual-circle} for the \featref{ritual-shrink} and \featref{ritual-grow}.
	You can now scribe one pair of circles in a design that can be used for either ritual.
}

\feat{Rite of Reflection}{ritual-reflect}{10}{
	\noprereq
}{
	\materials{A \circleref{small}, a mirror, the object to be reflected}
	
	This ritual takes a minute to perform, and reflects an object that fits within the \materialref{ritual-circle}.
	The object is physically transformed to be the opposite way around, reflected along its centre-line.
	Its right becomes its left, and vice versa.
}

\feat{Living Reflection}{ritual-reflect-creature}{10}{
	\featref{ritual-reflect}
}{
	You can use the \featref{ritual-reflect} upon creatures as well as objects.
	This allows you to effectively move wounds from one side of the body to the other, and to make a right-handed person left-handed, or vice versa.
	Everything worn or carried by the affected creature is reflected with them.
}

\feat{Mirror Circle}{circle-mirror}{15}{
	\noprereq
}{
	This \materialref{ritual-circle} simply acts as a mirror, covering the entire area within the perimeter of the circle.
	It needs no ritual to activate it; the existence of the \materialref{ritual-circle} is enough.
	It can be scribed in any size; \circlerefbare{small} or larger.
	
	Unlike most \materialrefplural{ritual-circle}, this circle can be scribed on non-horizontal surfaces, even vertical surfaces, or on a ceiling.
	The mirror appears perfectly smooth, regardless of irregularities in the underlying surface.
	However, the lines that form the \materialref{ritual-circle} are still faintly visible in the mirror, making its nature obvious.
	
	One of these circles can be used in magic that requires a mirror, such as \featref{scrying-mirror}.
}

\feat{Rite of Clear Skies}{ritual-weather-clear}{10}{
	\noprereq
}{
	\materials{A \materialref{stone-circle} with a view of the open sky}
	
	This ritual takes 1 hour to perform, over the course of which any inclement weather gradually vanishes.
	The skies are cleared of cloud, high winds are calmed, and the temperature becomes roughly normal for the season.
	The changes affect everywhere within a few kilometres, and typically last a few hours before normal weather conditions reassert themselves.
}

\feat{Rite of the Sun}{ritual-weather-sun}{15}{
	\skillref[1]{ritual-magic},
	\featref{ritual-weather-clear},
	\featref{ritual-fire}
}{
	\materials{A \materialref{stone-circle} with a view of the open sky, \herb{sunflower seeds}{2}}
	
	This ritual functions as the \featref{ritual-weather-clear}, except that it also causes the sun to shine more brightly over the area.
	It can only be performed during the day, but as long as the day lasts, the temperature in the affected region becomes sweltering.
	It isn't hot enough to be immediately dangerous, but can cause heat stroke for those with an inadequate water supply.
	If continued over multiple days, it can bring drought.
}

\feat{Rite of Rain}{ritual-weather-rain}{10}{
	\noprereq
}{
	\materials{A \materialref{stone-circle} with a view of the open sky, a bucketful of water (about 10 litres)}
	
	This ritual takes 1 hour to perform, over the course of which the skies are covered with cloud.
	At the conclusion of the ritual, rain begins to fall.
	Everywhere within a few kilometres is affected, and the rain typically lasts a few hours before normal weather conditions reassert themselves.
}

\feat{Rite of the Storm}{ritual-weather-storm}{15}{
	\skillref[1]{ritual-magic},
	\featref{ritual-weather-rain}
}{
	\materials{A \materialref{stone-circle} with a view of the open sky, a bucketful of water (about 10 litres), a copper rod}
	
	This ritual functions as the \featref{ritual-weather-rain}, except that it brings a thunderstorm along with the rain.
}

\feat{Rite of Snow}{ritual-weather-snow}{15}{
	\skillref[2]{ritual-magic},
	\featref{ritual-weather-rain},
	\featref{ritual-cold}
}{
	\materials{A \materialref{stone-circle} with a view of the open sky, a block of ice}
	
	This ritual functions as the \featref{ritual-weather-rain}, except that it brings snow instead of rain.
	The affected region becomes cold enough to allow this snow---not necessarily below freezing, so the snow might not settle, but very close to it.
	Note that you may need to use the \featref{ritual-cold} to create the ice required for this ritual.
}

\feat{Weather Scribing}{ritual-weather-circle}{10}{
	\featref{ritual-weather-clear} or \featref{ritual-weather-rain}
}{
	You may use a \circleref{large} for the \featref{ritual-weather-clear}, \featref{ritual-weather-sun}, \featref{ritual-weather-rain}, \featref{ritual-weather-storm}, or \featref{ritual-weather-snow}, in place of the \materialref{stone-circle}.
	However, it must still have a view of the open sky.
	Each rite requires a different design of \materialref{ritual-circle}.
}

\feat{Lasting Weather}{ritual-weather-duration}{20}{
	\skillref[2]{ritual-magic},
	\featref{ritual-weather-clear} or \featref{ritual-weather-rain}
}{
	When you perform the \featref{ritual-weather-clear}, \featref{ritual-weather-sun}, \featref{ritual-weather-rain}, \featref{ritual-weather-storm}, or \featref{ritual-weather-snow}, you may make the effect last indefinitely, instead of fading after a few hours.
	Recall that the \featref{ritual-weather-sun} only causes heat during the day, but it will continue to cause heat each day.
	Only one of these rituals can be in effect over a given area at a time, however, and subsequent invocations---by you or someone else---override previous ones.
	
	You can end the ritual at any time, as long as you can touch the \materialref{stone-circle}---or \materialref{ritual-circle}, with \featref{ritual-weather-circle}---used for the ritual.
	It also ends if the circle is broken.
}

\feat{Instant Weather}{ritual-weather-speed}{15}{
	\skillref[3]{ritual-magic},
	\featref{ritual-weather-clear} or \featref{ritual-weather-rain}
}{
	You can perform the \featref{ritual-weather-clear}, \featref{ritual-weather-sun}, \featref{ritual-weather-rain}, \featref{ritual-weather-storm}, or \featref{ritual-weather-snow} in only 1 minute, instead of 1 hour.
	If you do so, the change in the weather is instantaneous, rather than gradual.
	Clouds pop into existence from thin air, or vanish without a trace.
	It is utterly obvious to anyone watching the sky that this was a magical change.
}

\feat{Rite of Detection}{ritual-detect}{10}{
	\noprereq
}{
	\materials{A \circleref{small}, a bell}
	
	This ritual takes a minute to perform, and it lasts for 24 hours, or until the \materialref{ritual-circle} is broken.
	For that duration, you are aware of how many creatures are within the circle--though not their identities---regardless of your location.
	You are also aware when the effect ends, whether it expires or the circle is broken.
	
	A change in the number of creatures in the circle, or the end of the ritual's effect, is enough to wake you from natural sleep.
	You may choose to ignore this when you go to sleep, if you wish to rest uninterrupted.
	
	If you have multiple \materialrefplural{ritual-circle} under the effect of this ritual, you know the count for each circle individually.
	You know which count corresponds to which circle.
}

\feat{Indefinite Detection}{ritual-detect-duration}{10}{
	\skillref[1]{ritual-magic},
	\featref{ritual-detect}
}{
	When you begin the \featref{ritual-detect}, you may choose to make it last indefinitely, until the \materialref{ritual-circle} is broken.
}

\feat{Detection Augmentation}{ritual-detect-augment}{10}{
	\skillref[1]{ritual-magic},
	\featref{ritual-detect}
}{
	You may scribe the \materialref{ritual-circle} for the \featref{ritual-detect} as an {\augmentation} to an existing \materialref{ritual-circle}.
	If you do so, the \featref{ritual-detect} also makes you aware when the {\augmented} circle is broken.
}

\feat{Ritual Identification}{ritual-detect-identify}{5}{
	\featref{ritual-detect},
	\featref{divination-taglock-identify}
}{
	You may leverage the connection provided by the \featref{ritual-detect}, using your \discref{divination} against targets within the circle.
	While a creature is within your \featref{ritual-detect}, you may identify them as though using \featref{divination-taglock-identify} upon them.
	Unlike \featref{divination-taglock-identify}, you consequently learn that the target is still alive, and---as long as you can remember where you scribed the \materialref{ritual-circle}---their approximate location.
}

\feat{Taglock Simulation}{ritual-detect-taglock}{15}{
	\skillref[2]{ritual-magic},
	\featref{ritual-detect-identify}
}{
	The connection provided by your \featref{ritual-detect} becomes so strong as to be almost physical.
	While a creature is within your \featref{ritual-detect}, you may perform magic as though you had a \materialref{taglock} from that creature.
}
