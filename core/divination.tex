\chapter{Divination}
\disclabel{divination}{Seer}{Seers}

\section{Feats}

\feat{Perfect Positioning}{divination-self}{10}{
	None
}{
	While most divination deals in seeing the future, the past, or distant locations, there is one very useful trick that simply allows you to see the \emph{here} and \emph{now}.
	
	You always know where and when you are, and which way you are facing.
	You can't get lost, and always know which way to go to get to a place you've been before.
	You know what time it is, and how long you've been sleeping whenever you awake.
	
	However, this does not function in the {\mentalrealm}.
}

\feat{Taglock Identification}{divination-taglock-identify}{10}{
	None
}{
	You can touch a \materialref{taglock} and detect who it originates from.
	If you have met the target, you can identify them infallibly.
	
	If you have never met the target, you must make a \testtype{heed}{divination} Test, with higher results giving more information about the target.
	You can only get general information about the target this way, such as height, build, sex, appearance, and occupation.
	You can't get any information about their location, or even whether they are still alive.
}

\feat{Danger Sense}{divination-initiative}{20}{
	\skillref[1]{divination}
}{
	You can't quite see the future, as such, but you can tell when \emph{something} is about to happen.
	You might not know what it is, but it won't catch you entirely unawares.
	You may use your \skillref{divination} skill in place of any other skill when rolling {\initiative}, and may even use it when the {\initiative} Test would otherwise use no skill at all.
}

\feat{Perfect Prediction}{divination-dodge}{20}{
	\skillref[1]{divination},
	\featref{divination-initiative}
}{
	With your eyes closed and all your concentration turned to it, you can enhance your premonitions of danger with perfect clarity.
	Activating this effect requires an {\action}, and lasts until the beginning of your next {\turn}, or until you open your eyes.
	For the duration, you automatically evade any \actionref{attack} or other harmful effect that could reasonably be evaded, unless it hits you with a critical success.
	
	This only allows you to sense anything which may harm you, restrain you, or the like.
	With your eyes closed, you likely unaware of many other things.
	
	Furthermore, maintaining this for long periods is tiring.
	The GM may call for Tests to avoid {\exhaustion} after a minute or more of use.
}
