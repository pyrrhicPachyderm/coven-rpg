\discipline{Sigil Magic}{sigil-magic}{Scribe}{Scribes}

\section{Sigils}
\seclabel{sigils}

\capital{\sigils} are elaborate shapes, symbols, or glyphs that may be scribed upon an object, person, or animal in order to produce a magical effect.
A {\sigil} may be scribed in many different ways, for instance drawn in chalk, written in blood, carved into wood, chiseled into stone, or dyed upon the skin.
It must be scribed exactly, however, requiring a high degree of precision.
In calm situations, with ample time, this presents no problem.
When rushed, scribed on an unstable surface, or under otherwise problematic conditions, the GM may require a \testtype{grace}{sigil-magic} {\test}.

\capital{\sigils} are complicated, requiring several minutes to scribe without rushing.
This assumes a sigil drawn in chalk at a reasonable size: no smaller than a palm and no larger than the \practitionerpossessive{sigil-magic} armspan.
Sigils carved into a surface, or otherwise using some slower method, will take even longer.
Particularly large or small renditions of a sigil may also require additional time.

In all cases, the scribing of the {\sigil} is sufficient to activate its magic.
This is one of its main distinctions from \discref{ritual-magic}, which usually requires an accurately scribed \materialref{ritual-circle}, but additionally requires a ritual to produce the effect.
If a {\sigil} has an ongoing effect, this always lasts indefinitely, until the {\sigil} is broken, deformed, or destroyed.

\section{Marks}
\seclabel{marks}

\capital{\sigilmarks} are essentially the simpler version of sigils.
They are far less complicated, with far fewer lines and a lower requirement for accuracy in their scribing.
As such, they are faster and easier to scribe.
Scribed in chalk or some other quick method, they can be done in an {\action}, though slower methods or abnormally sized {\sigilmarks} may require longer.
Furthermore, they don't require a {\test} to scribe even when rushed or stressed, except in extreme scenarios, such as with your hands tied, or scribing them on an uncooperative target.

Apart from speed and difficulty of scribing, {\sigilmarks} follow all the same rules as {\sigils}.

\section{Feats}

\feat{Sigil of Destruction}{sigil-destruction}{15}{
	\noprereq
}{
	You have learned to scribe a {\sigil} that will destroy the object it is scribed upon.
	The object crumbles to pieces, and is reduced to dust.
	This can only affect non-living objects, not people, animals, or living plants.
	
	The {\sigil} must cover a substantial amount of the object's surface, and this limits the size of the object you can destroy.
	A {\sigil} large enough to destroy a typical door takes no additional time, but a {\sigil} to destroy a castle must cover most of one wall, several storeys high.
}

\feat{Mark of Destruction}{sigil-destruction-mark}{15}{
	\skillref[2]{sigil-magic},
	\featref{sigil-destruction}
}{
	You have simplified the \featref{sigil-destruction}, and can now scribe it as a {\sigilmark}.
}

\feat{Mark of Ignition}{sigil-fire}{15}{
	\noprereq
}{
	You have learned to scribe a {\sigilmark} that will ignite the object it is scribed upon.
	If the object is flammable, the marked surface erupts with flame, and the object cannot be extinguished while the {\sigilmark} remains.
	If the object is not flammable, the {\sigilmark} will have no effect.
	This can only affect non-living objects, not people, animals, or living plants.
	
	The flame begins at the size of the {\sigilmark}, but will spread from there across a flammable object.
	Depending upon how the {\sigilmark} has been scribed, it is likely to be quickly destroyed as the object burns.
	The flame will not vanish once the {\sigilmark} is destroyed, but can be extinguished normally at that point.
}
