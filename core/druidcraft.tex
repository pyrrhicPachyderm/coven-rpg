\discipline{Druidcraft}{druidcraft}{Druid}{Druids}

\section{The Wild and the Tame}
\seclabel{druidcraft-wild}

There is something of a divide in the practice of \discref{druidcraft}, a difference in philosophy that reflects itself in a \practitionerpossessive{druidcraft} magic.

Some \practitioners{druidcraft} seek to harness the power of nature.
They tame animals, and cultivate herbs.
Their abilities are often entirely non-magical, springing entirely from a deep understanding of the plants and animals they tend.

But some believe that \emph{harnessing} nature is anathema to the very soul of \discref{druidcraft}.
They prefer to \emph{unleash} nature's power; its wrath.
Some of them even would see civilisation fall; see the forests reclaim the world.
For these \practitioners{druidcraft}, the hand of civilisation---taming, or cultivation---inhibits their magic.
They can only work with {\wild} plants and animals.

Of course, there are also those \practitioners{druidcraft} who seek a balance.
They will tame and cultivate as necessary, but neither are they afraid to unleash the power of the wild.
They might learn magic that works against both {\wild} and {\nonwild} plants and animals.

As with so many things, whether or a plant or animal counts as {\wild} is not strictly defined.
Guidelines are presented below, but the GM is the final arbiter.
And, if the \practitioner{druidcraft} in question can make a good enough case, the GM might allow a {\test} to make her magic work.

It is of note that {\wild} magic can prove a little self-defeating, with both animals and plants.
The more that a \practitioner{druidcraft} uses her magic on a {\wild} animal or plant, the more accustomed to humans it becomes.
If she keeps tending to the same animal or plant, it might no longer be {\wild}.
She should be careful not to let this happen.

\subsection{Wild Plants}

For a plant, {\wildness} begins when its seed is planted.
If it was planted by \emph{intention}, typically by a human or another intelligent creature, then it's {\nonwild}.
But this can change.
A plant that is tended regularly, even if its seed originally fell naturally, loses its {\wildness}.
Conversely, a seed that was planted intentionally, but now hasn't been tended for years, might become {\wild}.

As an exception, an \herbtype{5} that is conscious or intelligent---one with a score in any non-physical attribute---is never considered {\wild} \emph{or} {\nonwild}.
It is its own creature: not free from intelligence, but not under the hand of humans either.

\subsection{Wild Animals}

Whether or not an animal is {\wild} is determined more by its upbringing and experiences than by the circumstances of its birth.
Mostly, it is determined by its reactions to humans---or other intelligent creatures.
An animal that retains its natural reaction to people---be that fear or predatory instinct---is still {\wild}.
But if it is accustomed to people, seeing them as masters, suppliers of food, or is even something to ignore, that animal is {\nonwild}.

An \featref{animal-companion}---even a \featref{animal-companion-feral}---is always a {\nonwild} animal.
A familiar is neither {\wild} or {\nonwild}---it is it's own, just like an intelligent \herbtype{5}.
An undead creature is no longer an animal at all.

\section{Transformation}
\seclabel{transformation}

Many abilities of \discref{druidcraft} allow a \practitioner{druidcraft} to {\transform} herself, or others, into other forms.
Typically, these are the forms of animals, but some druids have learned to take the shapes of plants, or even stranger things.
These {\transformations} adjust some of the target's attributes, abilities, and other statistics to match those of the form they {\transform} into, as described below.
The statistics of the form they are {\transforming} into can often be found in \partref{statblocks}, or can otherwise be invented by the GM.

The target takes the \attref{might}, \attref{grace}, \statref{resilience}, and \statref{speed} of the new form.
It keeps its other six attributes, all its own skills, and all its own feats, as well as its own memories and personality.
Calculate its \statref{shock-threshold} and \statref{dodge-rating} using its adjusted attributes.

It also gains most abilities of the new form, such a \creaturerefpossessive{dog} bite, or a \creaturerefpossessive{raptor} keen eyes.
However, it may not gain abilities that are clearly mental, rather than physical---this is left to the discretion of the GM.
Nor does it gain the skills of the new form.

A {\transformed} creature loses most of its ability to communicate.
It typically doesn't have the anatomy to communicate in its original fashion---a bear's mouth cannot form human speech, for example.
However, it doesn't gain any understanding of how to communicate with creatures of its new form, either.
That said, a witch and her familiar can communicate regardless of which forms they are in.
Furthermore, feats which allow communication in new ways work even while {\transformed}.
For example, a \practitioner{druidcraft} with \featref{animal-language-2} who {\transforms} into a bear can still communicate with other animals, just not with humans.

\capital{\transformation} does not remove any conditions, such as the effects of potions, upon the target.
It is still affected by any {\damage}, and if this reduces its new \statref{shock-threshold} to 0 or below, it dies.
Any wounds remain, staying in an appropriate location on the new form.
For example, a human with a wounded arm would have a wounded wing if it {\transformed} into a bird.

If a {\transformed} creature dies, the corpse reverts to its original form.

There are two kinds of {\transformations}: those that take a generic form, and those that imitate a specific creature.
If they take a generic form, for example, a person who {\transforms} into a bear would appear as a generic bear, not necessarily looking like the bear whose form they were taking.
However, they would take the same form each time that they {\transform} into a generic bear.
Furthermore, their bear form would have some characteristics that make it recognisable as them---the set of the face and colour of the eyes, for example---although these features might require very careful study to spot.

If a {\transformation} imitates a specific creature, they take on the appearance of that creature entirely.
Wounds and other conditions are not imitated, however, although old scars might be.

\section{Feats}

\feat{Animal Companion}{animal-companion}{25}{
	\skillref[1]{animals}
}{
	You have a highly-trained a loyal companion that will follow you anywhere.
	This might be a \creatureref{dog}, \creatureref{raptor}, \creatureref{horse}, or any other easily trainable animal the GM approves.
	If it is killed or otherwise lost, it takes several weeks to train another creature to the same level.
	
	You don't share any magical bond with this animal like you do with your familiar, only a bond established through training and friendship.
	You've trained it with enough commands to get it to do what you want under normal circumstances, as long as you continue to treat it well.
	
	Note that you do not need this feat to have a pet, or even a riding horse.
	This feat is only necessary to have an animal that will follow you unquestioningly into life-threatening scenarios.
}

\feat{Twin Companions}{animal-companion-2}{25}{
	\skillref[2]{animals},
	\featref{animal-companion}
}{
	You have trained an additional \featref{animal-companion}, allowing you to have two at once.
	They may be the same kind of animal, or different kinds.
}

\feat{Beastmaster}{animal-companion-3}{25}{
	\skillref[3]{animals},
	\featref{animal-companion-2}
}{
	A third \featref{animal-companion} rounds out your pack.
}

\feat{Feral Companion}{animal-companion-feral}{15}{
	\skillref[2]{animals},
	\featref{animal-companion}
}{
	Anyone can train a hound, a hawk, or a horse.
	But a bear?
	A goat?
	A \emph{cat}?
	It takes quite someone to tame such a beast.
	
	You may make an \featref{animal-companion} from even animals that cannot easily trained; any animal the GM approves.
	You may only have one such \featref{animal-companion}; keeping several playing nicely together and under control is still beyond you.
}

\feat{Feral Beastmaster}{animal-companion-feral-2}{15}{
	\skillref[3]{animals},
	\featref{animal-companion-feral},
	\featref{animal-companion-2}
}{
	A sloth of bears?
	An army of frogs?
	A herd of cats?
	No menagerie is beyond your ability to tame.
	
	You may make \emph{every} \featref{animal-companion} a \featref{animal-companion-feral}, if you wish.
}

\feat{Beast Whisperer}{animal-language}{15}{
	\skillref[1]{animals}
}{
	Through observations of you own interactions with your familiar, you've learned to establish a similar, albeit lesser, form of communication with other animals.
	However, this only works with animals that already have some familiarity with humans: {\nonwild} animals.
	
	The ``language'' this establishes is rather rudimentary.
	It's a mixture of gestures, sounds, and body language, and it takes quite some time to convey any complexity.
	A short, shouted command still works as normal, but anything longer takes at least twice as long to convey as it would through speech---possibly more than ten times as long for something rather complicated.
	It works both ways, allowing you to communicate to the animal, and also allowing the animal to convey information to you.
	
	This communication does nothing to improve the animal's intelligence.
	Many animals can follow a multi-stage series of instructions, and even relay information such as what they saw and where they've been, but abstract reasoning is beyond them.
	This feature also gives an animal no particular inclination to follow instructions from you, if it did not already have it.
}

\feat{Beast Tongue}{animal-language-2}{15}{
	\skillref[2]{animals},
	\featref{animal-language}
}{
	As you learn to think like an animal, you're becoming better at talking like one too.
	You may communicate with even {\wild} animals through the same ``language'' provided by \featref{animal-language}.
}

\feat{Critter-Chatter}{animal-language-speed}{15}{
	\skillref[2]{animals},
	\featref{animal-language}
}{
	With more expressive body language and better impressions of animal noises, you can speed up your communications.
	You can use the ``language'' provided by \featref{animal-language}---or \featref{animal-language-2}, if you have it---to convey information at about the same speed as speech.
}

\feat{Beast Call}{animal-call}{10}{
	\skillref[1]{animals}
}{
	\materials{A whistle, a \materialref{taglock} from the animal you wish to call}
	
	You may call an animal from any distance, as an {\action}.
	As you blow the whistle, only the target animal hears your call, and it hears it regardless of distance---even from hundreds of miles away.
	It knows your location, and that you want it to come to you.
	However, this call gives an animal no particular inclination to obey you---an \featref{animal-companion} will probably come, but most {\wild} animals are just as likely to avoid you.
}

\feat{Calling Order}{animal-call-language}{10}{
	\skillref[2]{animals},
	\featref{animal-call},
	\featref{animal-language}
}{
	When you use \featref{animal-call}, you may issue any order, or other short communication---not just a call.
	This must be a single, brief phrase, the sort you could sum up in a few words.
	For example ``bring the pack'', ``fetch my broom'', or ``come; I have meat''.
	The animal still learns your location, regardless of what you tell it.
	
	Just as with \featref{animal-call}, this gives the animal no particular inclination to obey you, although ``I have meat'' may tempt it.
	Furthermore, you can only use this on animals that you can communicate with using \featref{animal-language}---you need \featref{animal-language-2} to use this on {\wild} animals.
}

\feat{Irresistible Call}{animal-call-2}{15}{
	\skillref[3]{animals},
	\featref{animal-call}
}{
	\materials{A \circleref{small}, a whistle, a \materialref{taglock} from the animal you wish to call, appropriate food for the animal you wish to call}
	
	Weaving compulsion in with your call, you can force an animal to come.
	This works like \featref{animal-call}, except that the animal feels compelled to come.
	This is not an all-overriding compulsion; its survival instinct still takes priority.
	But it will always come, as soon as it can.
	This cannot be used in conjunction with \featref{animal-call-language}---it is always, and only, a call.
	
	This only works as long as you remain in the \materialref{ritual-circle}, with the food.
	If you or the food leave, the compulsion vanishes.
}

\feat{Clever Boy}{animal-intelligence}{20}{
	\skillref[2]{animals},
	\featref{animal-companion},
	\featref{animal-language}
}{
	A witch who treats her animals like people might come to find her animals behaving a little like people.
	An animal that you have trained---besides a familiar, which is already far smarter than any mere beast---seems more intelligent.
	It gains a \positive{1} bonus to its \attref{ken} and \attref{wit} scores.
	It becomes able to understand more complicated concepts, perhaps even rudimentary abstract reasoning, and its memory improves.
	This benefit applies automatically to an \featref{animal-companion}, but also applies to other animals that the witch has played a major part in training.
}

\feat{Walkies}{animal-speed}{10}{
	\skillref[1]{animals},
	\featref{animal-companion}
}{
	Regular walks and runs with your animals keeps the muscles trained.
	An animal you have trained---except your familiar---adds 2 to its \statref{speed}.
	This applies to walking, swimming, flying or any other method the animal possesses, although it requires that the animal is able to move in the first place, and cannot do more than double the existing speed.
	This benefit applies automatically to an \featref{animal-companion}, but also applies to other animals that the witch has played a major part in training.
}

\feat{Run Like the Wind}{animal-speed-2}{15}{
	\skillref[3]{animals},
	\featref{animal-speed}
}{
	Your beasts are champion runners, fliers or swimmers, capable of incredible turns of speed.
	When you apply the bonus of \featref{animal-speed}, the animal's \statref{speed} increases by 2, or half its original speed, whichever is higher.
}

\feat{Beast Shape}{animal-transform}{15}{
	\skillref[1]{animals}
}{
	As an {\action}, you can {\transform} into the shape of an animal you touch.
	You take on a generic form corresponding to that animal---for example, if you touch a bear, you take on a generic bear form.
	The {\transformation} lasts up to an hour, but you can revert to your original form as an {\action} at any time.
}

\feat{Beast Study}{animal-transform-2}{10}{
	\skillref[1]{animals}
}{
	You no longer need to be able to touch an animal to take its shape with \featref{animal-transform}---you only need to be able to see it.
}

\feat{Extended Transformation}{animal-transform-duration}{10}{
	\skillref[1]{animals}
}{
	When you use \featref{animal-transform}, the {\transformation} lasts up to 24 hours.
}

\feat{Indefinite Transformation}{animal-transform-duration-2}{10}{
	\skillref[3]{animals}
}{
	When you use \featref{animal-transform}, the {\transformation} lasts indefinitely.
}

\feat{Forestwalker}{plant-movement}{10}{
	\skillref[1]{botany}
}{
	You have an agreement of sorts with the {\wild} plants of the world.
	They won't hurt you if you don't hurt them.
	
	You are unaffected by {\difficultterrain} or other impediment of movement caused by {\wild} plants.
	This doesn't go quite so far as to let you walk through solid trees, but you can run through a bramble bush just fine.
	Furthermore, their natural defences, such as a bramble's thorns or a nettle's sting, don't affect you.
	Lastly, the fact that you don't disturb the plants can make you far harder to track when travelling in a forest.
	
	This only works as long as you refrain from damaging the plants---start hacking at a bramble patch and it will tear at you just as much as anyone else.
	The effects of plants you ingest, or use in \discref{brewing}, are similarly unchanged.
}

\feat{Treespeaker}{plant-speak}{10}{
	\skillref[1]{botany}
}{
	Speaking with plants isn't like speaking with animals.
	Plants have no minds; there's nothing to speak to.
	But with many plants, trees, bushes, and grasses bound by their roots, whispering together as the wind blows through their canopy---a forest as a whole forms a creature, of sorts.
	It feels things moving through it---rustling its leaves, snapping its twigs---and the disturbances ripple through the plants.
	And when a tree burns, the whole forest cries out in anguish.
	
	You may listen to {\wild} plants around you, reading the visible and audible signs they give.
	This allows you to make a \testtype{heed}{botany} {\test} to pick up on disturbances that have affected the plants: animals passing through, people camping, crash-landings, fires, and so on.
	A thorough search takes a couple of minutes, but obvious signs might be detected in an {\action}.
	
	The talk of the plants is affected by time and distance, making anything more than a kilometre away or more than an hour ago quite difficult to detect.
	It is also far easier to pick up on these signs where the vegetation is denser; a forest is easy, but grassland is incredibly difficult.
	Lastly, any significant damage to plants is far easier to pick up on.
	Someone who takes branches to make a shelter leaves clear evidence, and logging echoes far and wide.
	Fire is by far the easiest to detect; signs of a large conflagration might echo to the next forest over, and be seen even by a witch who isn't paying attention.
}

\feat{Wild Growth}{plant-grow}{15}{
	\skillref[1]{botany}
}{
	Touching a {\wild} plant, you can accelerate its growth.
	You can cause a day's worth of growth in an {\action}, accumulating a year's growth for each hour.
	It grows exactly as it normally would, and you have no control over this.
	Furthermore, the entire plant grows; you cannot grow just a part of it.
	This growth can accelerate a plant outside of its typical season, so you can grow leaves on a tree in winter, or blackberries on a bramble bush in spring.
}

\feat{Rampant Growth}{plant-grow-2}{15}{
	\skillref[2]{botany},
	\featref{plant-grow}
}{
	When you use \featref{plant-grow}, you can push more life into the plant, causing it to grow faster.
	You can cause a month's growth in an {\action}---a year's growth every two minutes.
}

\feat{Sudden Growth}{plant-grow-3}{15}{
	\skillref[3]{botany},
	\featref{plant-grow-2}
}{
	Pushing any more life into a plant requires you to sacrifice a little of your own.
	If you do so, however, you can cause an incredible burst of growth.
	
	When you use \featref{plant-grow}, you can cause up to 20 years of growth in one {\action}.
	This growth can even occur fast enough to harm people who are standing in the way.
	Taking this {\action}, however, causes a level of {\exhaustion} affecting \attref{might} and \attref{grace}.
}

\feat{Partial Growth}{plant-grow-control}{10}{
	\skillref[2]{botany},
	\featref{plant-grow}
}{
	You can funnel life into a particular part of a plant, growing it faster than the rest.
	When you use \featref{plant-grow}, you can cause only parts of the plant to grow.
	For example, you might cause one branch to grow, or cause only the flowers and fruits to grow, without developing the branches.
	You still can't control \emph{how} it grows; if you grow a particular branch, that branch still grows in its normal direction.
	Nor does this affect the rate of growth; a branch grows at the same speed whether you grow just that branch, or the whole plant.
}

\feat{Directed Growth}{plant-grow-control-2}{20}{
	\skillref[3]{botany},
	\featref{plant-grow-control}
}{
	You can control a plant as you fuel its growth.
	When you use \featref{plant-grow}, you can control the nature and direction of the plant's growth.
	For example, you could cause a branch to grow upwards, downwards, or sideways, or even cause it to fork at the tip and grow in two directions.
	You cannot use this to increase the rate of growth; you need \featref{plant-grow-2} or \featref{plant-grow-3} for that.
}

\feat{Druidic Grove}{plant-grove}{15}{
	\skillref[1]{botany}
}{
	\materials{A \materialref{stone-circle} with plants growing within}
	
	You can pour your power into the soil, creating a grove where plants flourish.
	The ritual to do so takes an hour, and the effect lasts for one lunar month.
	
	For the duration, plants within the \materialref{stone-circle} grow and thrive without any tending.
	The area can be used as a {\garden}, without requiring any time tending the \materialrefplural{herb} that grow within.
	Only \materialrefplural{herb} that you could normally grow using your \skillref{botany} skill are affected.
	If you have \skillref[3]{botany}, however, even the needs of \herbtypeplural{5} are fulfilled by the grove.
}

\feat{Grove Tender}{plant-grow-grove}{15}{
	\skillref[3]{botany},
	\featref{plant-grow-3},
	\featref{plant-grove}
}{
	Inside a \featref{plant-grove}, you can draw power from the soil instead of yourself.
	You can use \featref{plant-grow-3} upon plants within a \featref{plant-grove} without suffering {\exhaustion}.
	Note that \featref{plant-grow-3} still requires {\wild} plants, so this is not particularly sustainable.
}

\feat{A Stand of Trees}{tree-standing-stone}{15}{
	\skillref[2]{botany},
	\skillref[1]{ritual-magic},
	\featref{plant-movement}
}{
	The trees will be stone for you, if you ask them nicely enough.
	You may treat any {\wild} tree taller than a man as a \materialref{standing-stone}, without a {\test}.
	Similarly, a ring of trees, or a clearing ringed by trees, acts as a \materialref{stone-circle}.
	Treating a forest without a clearing as a \materialref{stone-circle} still requires a {\test}---it's not obviously a circle, as such.
}

\section{Familiar Feats}

Just as \emph{you} can learn and improve, so can your familiar.
These feats represent that ability, and are also how you can increase your familiar's attributes and skills.

You still purchase these feats from your normal XP, so helping your familiar to develop comes at some cost to your own improvement.
The price is often well worth it, however; your familiar is a constant and loyal companion, and can contribute greatly in many situations.
Besides, you might learn a thing or two yourself while teaching your familiar.

Some of these feats require a particular kind of familiar, while others require your familiar to have particular skills.
This might affect your choice of familiar, so it can be worth reading these feats before selecting a familiar.
However, you will sometimes be able to train an unskilled familiar, in order to acquire a particular feat, so this needn't entirely dictate your choice.

\feat{Familiar Attribute}{familiar-attribute}{15}{
	\noprereq
}{
	Through regular training with your familiar, you've managed to improve its natural abilities.
	Increase one of your familiar's attributes by 1.
	You may take this feat multiple times, but you can only increase each attribute once.
}

\feat{Familiar Skill}{familiar-general-skill}{15}{
	\noprereq
}{
	You've taught your familiar a new skill, or helped it to improve an existing one.
	It gains 1 rank in a {\generalskill}.
	You may take this feat multiple times, but you can only increase each skill once.
}

\feat{Familiar Speciality}{familiar-speciality-skill}{15}{
	\noprereq
}{
	Some animals have a natural ability with a particular craft or vocation; an ability that is often carried to a familiar of that type.
	You've nurtured and developed that talent in your familiar.
	It gains 1 rank in a {\specialityskill}, in which it already had at least 1 rank.
	You may take this feat multiple times, but you can only increase each skill once.
}

\feat{Familiar Discipline}{familiar-discipline-skill}{25}{
	\noprereq
}{
	Not many animals carry a natural talent for magic, but your familiar does.
	You've honed this talent, bringing it to heights that some witches never even reach.
	Your familiar gains 1 rank in a {\disciplineskill}, in which it already had at least 1 rank.
	You may take this feat multiple times, but you can only increase each skill once.
}

\feat{Familiar Familiarity}{familiar-language}{5}{
	\noprereq
}{
	Each ``language'' that a witch shares with her familiar is unique, relying on the bond between their souls.
	However, all these ``languages'' share something in common, and, with a bit of work, your familiar has picked up some of this.
	
	Your familiar can now communicate with any other witch's familiar, just as quickly and easily as it can communicate with you.
	This works in both directions, with your familiar both ``speaking'' and ``listening''.
}

\feat{Familiar Witchspeak}{familiar-language-2}{5}{
	\featref{familiar-language}
}{
	A few more language lessons between you and your familiar have taught your familiar to communicate with other witches; and you to communicate with other familiars.
	This works just as quickly and easily as communication between you and your own familiar, and in both directions.
	
	Your familiar can only communicate with other witches who have---or once had---a familiar of their own.
	However, it can also communicate anyone who has somehow had a familiar, but is not a witch.
}

\feat{Familiar Layspeak}{familiar-language-3}{15}{
	\skillreffamiliar[1]{socialising},
	\featref{familiar-language-2}
}{
	Communicating with layfolk is \emph{far} harder than communicating with witches; most people aren't even expecting an animal to talk to them!
	But, as long as your familiar can draw their attention, it can make them understand.
	This works just like communication between \emph{you} and your familiar.
	However, it can sometimes leave layfolk a little confused as to \emph{why} they can understand this animal's gestures; they just can.
}

\feat{Familiar Beastspeak}{familiar-language-animals}{10}{
	\skillref[1]{animals},
	\skillreffamiliar[1]{animals},
	\featref{familiar-language},
	\featref{animal-language}
}{
	Just as you have learned to communicate with animals, you have taught your familiar to.
	It ought to be easier---it's nearly an animal itself, after all!
	
	Your familiar gains the benefit of \featref{animal-language}.
	If you also have \featref{animal-language-2} or \featref{animal-language-speed}, your familiar gains the benefit of those as well.
}
