\chapter{The Broad of It}
\chaplabel{general-rules}

\dropcap{This} chapter covers rules essential to day-to-day play.
Players and GMs alike should be familiar with at least the major points in here in order to play.
More specific rules, pertaining to various disciplines of magic, can be found in the appropriate chapters of \partref{disciplines}.

It is important to remember that this book cannot cover every situation that may arise during play.
The role of the GM includes adjudicating such scenarios, and the following section should contain guidelines to assist in that.
Furthermore, it is often helpful to do the same when the players simply cannot remember a rule, to avoid slowing down play while someone looks it up.
And lastly, remember that all the rules contained in this book are guidelines and suggestions.
Feel free to change them all that you want!
The most important thing is that everyone is having fun.

\section{Rounding Fractions}

In general, round down whenever you get a fraction, even if the fractional part is one half or greater.

\section{Tests}
\seclabel{tests}

\capital{\tests} are the dice rolls used to determine the outcome of an action when there is element of chance and risk involved.
Several of the rules in this chapter and others will specify the appropriate {\test} to make with a particular action, but the GM should be calling for other kinds of {\tests} whenever appropriate as well.

A {\test} is typically made with a skill and an attribute, although having no applicable skill is not uncommon.
Often, the rule that required the {\test} specifies these.
Otherwise, the GM chooses as appropriate.
The character's skill determines how many dice she rolls for a {\test}.
If there is no skill applicable to the test, or if the character has no ranks in the applicable skill, she rolls 3 dice.
Each rank in the skill gives an additional rolled die, to a maximum of 6 with all three ranks.
Total together the highest 3 of the rolled dice and add the character's relevant attribute to this total.
The final total is compared against a {\targetnumber} ({\tn}) set by the GM: if it meets or exceeds the {\tn} the {\test} succeeds; otherwise it fails.

A {\test} where every die shows a 1 or 2 is a critical failure, and a {\test} where all 3 kept dice show a 6 is a critical success.
In addition to the {\test} automatically succeeding or failing, the GM is encouraged to apply an additional drawback or benefit to the result of the {\test}.
Critical failures on {\tests} involving dangerous magic can be especially catastrophic.

\subsection{Dice Notation}

A variant of standard RPG dice notation is used for {\tests}.
The size of the dice and the fact that only three are kept is omitted, as these are constants.
For example, \dice{4} indicates a 4 die {\test} with no bonus, and \dice[2]{3} indicates a 3 die {\test} with an attribute bonus of 2.

\subsection{An Example Test}

As an example, suppose Mistress Talbot is peering out of her window and attempting to identify which manner of undead dog has just shambled into her garden.
The GM declares this to be a \testtype{ken}{necromancy} {\test}, as she is attempting to recall information about the undead.
Mistress Talbot dabbled in \discref{necromancy} as a youth, and has one rank in the skill, so she rolls 4 dice.
However, her memory has begun to fade with age, so she has only 1 \attref{ken}.
The four dice show 4, 6, 2 and 3.
Her player totals the three highest dice, the 6, 4 and 3, for 13.
Then she adds Mistress Talbot's \attref{ken}, 1, for a grand total of 14.
Her player announces the total to the table.

The GM knows that the dog is a simple zombie, the most common variety of undead, but it was killed and raised only yesterday so the characteristic rot hasn't properly set in yet.
In light of this, she assigns a {\targetnumber} of 12: not too easy, but not particularly difficult either.
Hearing Mistress Talbot's total of 14, the GM knows that she has met the {\tn} of 12: the {\test} has succeeded.
She announces that Mistress Talbot, by the creature's glassy eyes and stumbling gait, realises the midnight intruder is merely a zombie.
Reassured---she'd been fearing a ghoul or a hellhound---Mistress Talbot heads outside to see what the beast wants.
Though not without grabbing the poker from beside the fireplace, just in case.

\subsection{Target Numbers}
\seclabel{target-numbers}

A {\targetnumber} ({\tn}) represents the difficulty of the action that requires a {\test}.
The more difficult the action, the higher the {\targetnumber}, and the less likely the {\test} is to succeed.
In some situations, the same rule that requires a {\test} will specify its {\tn}.
In other situations, the GM should select a {\tn} she feels is appropriate.

Typical {\tns} range from approximately 9 to 21.
A {\test} with a {\tn} lower than 9 is not normally worth it: a character with no skill and an average score in the relevant attribute will succeed more than \SI{95}{\percent} of the time.
Similarly, a {\test} with a {\tn} higher than 21 is not normally worth it: a character needs a 5 in the relevant attribute to succeed without a critical success.
The following table shows a brief summary of the sorts of task particular {\tns} are suited to.

\begin{simpletable}{rX}
	\toprule
	{\tn} & Task Difficulty\\
	\midrule
	9 & Easy: An average, unskilled person would normally manage this.\\
	12 & Moderate: An average, unskilled person would manage this about half the time.\\
	15 & Challenging: It takes skill to pull this off consistently.\\
	18 & Difficult: Even a skilled person is unlikely to achieve this consistently.\\
	21 & Legendary: This takes great skill, ability and good luck to perform.\\
	\bottomrule
\end{simpletable}

Instead of assigning a simple pass-or-fail {\tn}, the GM may also employ graded success.
This is when a higher roll gives a higher level of success.
For instance, a higher roll on a {\test} to recall knowledge might mean that the character recalls more knowledge about the situation, while a higher roll on a check to influence a crowd might influence a greater proportion of the crowd.
This can also be used to apply success at a cost, where an intermediate roll, neither particularly high nor particularly low, means that the character succeeds at their task but incurs some drawback in doing so.
For example, a coven might try to intimidate a guard to allow them into the castle.
Failure could indicate the guard calls for backup and resists, while a very high result on the {\test} would mean he is cowed and allows them to pass.
an intermediate result might mean that he allows the coven to pass, but sneaks off to find reinforcements and confront them later, while they are inside the castle.

\subsection{Opposed Tests}
\seclabel{opposed-tests}

When two characters---be it two player characters or a player and a non-player character---act in direct opposition to one another, the GM may request an {\opposedtest}.
Such {\tests} have no fixed {\tn}.
Instead, roll for both characters, and whichever has the higher total succeeds.
If both have the same totals, the situation remains as it was before the {\test}, so far as possible.

\capital{\opposedtests} need not use the same attribute and skill for both characters.
For example, a character trying to hide might roll \testtype{grace}{stealth}, while the character searching for them rolls \testtype{heed}{perception}.

For a normal {\test}, the GM may alter the difficult by adjusting the {\tn}.
This is not possible in the event of an {\opposedtest}, so the GM may grant a bonus or penalty to a character at an advantage or disadvantage.
As a guideline, grant a \positive{3} bonus for a clear but not overwhelming advantage, or a \positive{6} bonus for a very major advantage.
If an even greater bonus would be appropriate, it is typically fair to simply award win to the advantaged character without a {\test}.

\subsection{Using Tests}

Be careful not to call for a {\test} when it's not necessary.
If the action is a simple one that the character should be able to routinely perform, such as walking through a door or ransacking a room for something that isn't hidden, it doesn't require a {\test}.
(However, what is routine for one character might not be for another; a closed door can present a serious obstacle to many familiars.)
If the action is impossible, such as jumping over the moon or convincing the King to give up his crown without solid leverage, the player shouldn't make a {\test}.
If the character wouldn't succeed even with a critical success, a {\test} should never be rolled.
Lastly, if there is no penalty for failure, there is no need for a {\test}.
If the character will keep on trying until she succeeds, there's no need to make the player keep rolling {\tests}.

\subsection{Rolling Fewer Than Three Dice}

Some effects will modify the number of dice a character rolls for a {\test}, and this can bring the number of rolled dice below three.
In this case, all the rolled dice are added to the total as normal, but the maximum total that can be reached is obviously reduced.
Additionally, critical success is no longer possible, as this require three dice showing 6.
Critical failure, however, becomes far more likely, as it only requires that all dice show 1 or 2.

If the number of dice rolled for a {\test} would be reduced to zero, the {\test} cannot be performed.
If it is unavoidable, it is automatically treated as a critical failure.

\section{The Flow of Time}

\subsection{Narrative Time}
\seclabel{narrative-time}

During normal play, the exact timing and duration of characters' actions are unimportant, and not carefully tracked.
It is enough to know whether something took a matter of seconds or minutes, an hour or two, or a couple of days.
This is {\narrativetime}, and the GM is free to be as accurate or as loose as necessary with time periods.

\subsection{Structured Time}
\seclabel{structured-time}

In tense situations with two opposing parties, exact timings and durations become important to track.
For this purpose, and to aid tactical thinking in such scenarios, the GM can move the game into {\structuredtime}.
Direct combat is perhaps the most common application of this, but chase scenes may also use them.
With the correct magic, some of the participants might even be many miles apart.

\capital{\structuredtime} is divided into {\rounds} and {\turns}.
Every character participating in the scene gets one {\turn} each {\round}.
Although the {\turns} are resolved in some order, all characters are assumed to be acting simultaneously and continuously.
If it becomes particularly relevant for some reason, assume each {\round} takes approximately 10 seconds.

On each {\turn}, a character may move a number of metres equal to their \statref{speed} and take one {\action}.
An {\action} is something that requires most of the character's effort during their {\turn}, such as attacking someone, performing a brief bit of magic, knocking a hole in a wall or quaffing a potion.
They may also take a reasonable number of minor actions that shouldn't require their full concentration, such as opening or slamming a door, drawing a sword, pointing at something or speaking a short sentence.
Not everything can be accomplished in one {\action}.
For example, winching a drawbridge closed may take several {\actions}, as might even one of the faster magical rites.
Some of the {\actions} available to a character are given in the section \secref{combat-actions}, but the GM is free adjudicate anything the characters try as one or more {\actions}.
In fact, it is likely that most of a character's {\actions} are improvised {\actions}, not appearing in that list.

\subsection{Initiative}
\seclabel{initiative}

When the GM determines that the game should move into {\structuredtime}, {\initiative} {\tests} are used to determine the order in which participants take their {\turns}.
{\initiative} determines how quickly characters notice the situation and are ready to act.

\capital{\initiative} {\tests} can use any attribute and skill appropriate to the situation, as determined by the GM.
For example, an argument that boils over into a brawl might prompt {\initiative} {\tests} using \testtype{heed}{insight}, favouring characters who noticed tensions rising and fists clenching.
Combat that begins as characters race to the source of a scream might use \testtype{grace}{athletics}, favouring characters who arrive fastest.
If nothing in particular seems appropriate, default to a \attref{grace} or \attref{heed} {\test} with no applicable skill.

The GM may even assign different {\tests} to different characters.
Bonuses and penalties may be assigned, as for {\opposedtests}.
For example, suppose a group of bandits ambush for a group of travellers.
The bandits roll \testtype{grace}{stealth} to spring from hiding, with a \positive{6} bonus as the ambushers.
The travellers roll \testtype{heed}{perception} to notice the bandits attacking.
The GM may assign a greater or lesser bonus to a better-laid ambush, or one staged in a suboptimal location.

\capital{\initiative} {\tests} are not made against a particular {\tn}.
Rather, all characters are ranked in order.
This is the {\initiative} order, and remains the same on subsequent {\rounds}.
The character with the highest result takes their {\turn} first, and subsequent {\turns} proceed down the {\initiative} order.
Once all characters have taken a {\turn}, return to the top of the {\initiative} order for the next {\round}.

To save time, the GM may make a single {\test} for a group of similar NPCs, such that they all get the same result and take their {\turns} at the same time.
Similarly, a witch's familiar and all other creatures associated with her (such as a horse she is riding, or her golems and undead) use the witch's {\initiative} result and take their {\turns} at the same time as her.

\section{Movement}

Each {\turn} in {\structuredtime}, a character can move a number of metres equal to her \statref{speed}, as well as taking an {\action}.
If she takes the \actionref{dash} {\action}, she may move a total number equal to twice her \statref{speed}.
This assumes that she is moving on foot over smooth ground.
This speed represents urgent movement over a short period.
A character trying to maintain this pace for more than a couple of minutes typically requires an \testtype{might}{athletics} {\test} to avoid tiring.
%TODO: Reference exhaustion rules here?

%TODO: Long-range travel times.

%TODO: Climbing and swimming.

\subsection{Difficult Terrain}
\seclabel{difficult-terrain}

\capital{\difficultterrain}, such as dense forest or a bog, slows characters trying to move through it.
As a simple default, movement through it is halved; it costs 2 metres from a character's \statref{speed} to move through 1 metre of {\difficultterrain}.
The GM is free to impose a lesser or greater penalty for more or less severe terrain.

For some kinds of {\difficultterrain}, the GM may offer players the option to ignore the movement penalty at an alternative cost.
For example, a character pushing through brambles may move at full speed, but be subjected to a {\damagetest} for doing so.
A character moving on slick ice or along a narrow ledge may move at full speed, but must succeed on a \testtype{grace}{athletics} {\test} to avoid falling over, or off the edge{\dots}

\subsection{Jumping}

As part of moving, you may make a jump.
Clearing a wide gap or a tall obstacle as part of this jump may require a {\test}.
This {\test} usually uses \testtype{might}{athletics}, though the GM might request \testtype{grace}{athletics} if finesse is more important than pure height or length.

As a guideline, the {\tn} to clear a horizontal obstacle is 3 times its length in metres, and the {\tn} to clear a vertical obstacle is 9 times its height in metres.

\section{Injury}

Witchcraft is a dangerous business.
Between mad spirits, evil demons, foul undead, and disgruntled mobs of villagers, injury is inevitable.
And it's not only her own injuries that a witch has to deal with.
One of a witch's duties is to tend to the injuries of her neighbours, nursing them back to health after an accident or disease has laid them low.
Or, when they are beyond her help, easing their final moments.

A character's resistance to injury is determined by two statistics: \statref{resilience} and \statref{shock-threshold}.
Most creatures of flesh and blood, including humans and familiars, have 3 \statref{resilience}.
Other creatures, such as golems, may be more or less resilient.
A character's \statref{shock-threshold} is equal to 12, plus their \attref{might}, plus their \attref{will}.

\subsection{Damage Tests}
\seclabel{damage-tests}

A {\damagetest} is a special type of {\test} used to determine how much an effect hurts a character.
It is made like a normal {\test}, by rolling some number of dice and adding the highest 3 together, with a flat bonus.
In the case of an attack by one character upon another, the number of dice are determined by the weapon used and the flat bonus by the wielder's strength.
In other cases, the GM or the rules of the damaging effect assign the number of dice and the bonus.
For small effects, this can often be fewer than 3 dice.

The following table provides examples of the number of dice and the bonus for {\damagetests}.

\begin{simpletable}{Xl}
	\toprule
	Effect & Damage\\
	\midrule
	Touching a hot cauldron & \dice{1}\\
	Crawling through brambles & \dice{2}\\
	Wave-tossed against a boulder & \dice{3}\\
	Hit by a falling brick & \dice{4}\\
	Falling on a sword & \dice{5}\\ %TODO: Evaluate and expand.
	Hit by a falling tree & \dice[4]{5}\\
	\bottomrule
\end{simpletable}

Additionally, a {\damagetest} is not made against a particular {\tn} like most {\tests}.
Instead, it applies two effects to the target, {\shock} and {\damage}.
\capital{\shock} is always tested for before {\damage} is applied.

Critical failure on a {\damagetest} means no effect is applied at all; the blow was glancing and won't do more than bruise slightly.
Critical success on a {\damagetest} may immediately kill the target or leave them with a lasting injury, at the GM's option, and always applies {\shock}.

\subsection{Shock}
\seclabel{shock}

If a {\damagetest} meets or exceeds the target's \statref{shock-threshold}, or critically succeeds, the target goes into {\shock}.
A character in {\shock} falls unconscious and cannot be roused while they remain in {\shock}.
If a character in {\shock} would go into {\shock} again due to another {\damagetest}, they die.

Additionally, at the start of each of the {\shocked} character's {\turns}, roll a special {\test} against them.
This {\test} applies no flat bonus, and uses the same number of dice as the {\damagetest} that sent the character into {\shock}: a character is more likely to bleed out from a sword wound than a punch.
If it meets or exceeds the {\shocked} character's \statref{shock-threshold}, they die.
This {\test} is not considered to be a {\test} made by any character.

If the {\test} made every {\turn} ever totals 9 or less, unless it also meets or exceed their \statref{shock-threshold}, the character is no longer in {\shock}.
However, the character remains unconscious and cannot be naturally roused for at least a few minutes---longer if their injuries are severe.

A character can also be brought out of {\shock} by another character tending to them.
This requires an {\action} and a successful \testtype{ken}{healing} {\test}.
The {\tn} for this {\test} is 3 times the number of dice that would be rolled against the {\shocked} character each round.

\subsection{Damage}
\seclabel{damage}

After {\shock} has been tested for, whether or not it occurs, the {\damagetest} causes {\damage}.
To calculate {\damage}, divide the result of the {\damagetest} by the target's \statref{resilience}.
For example, if the result of the {\damagetest} is 13 and the damaged creature has 3 \statref{resilience}, they suffer 4 {\damage}.
{\damage} accumulates: a character who has previously suffered 3 {\damage} and suffers an additional 2 is now suffering from 5 {\damage}.

\capital{\damage} has two effects.
Firstly, a character subtracts their current {\damage} from their \statref{shock-threshold}.

Secondly, if a character's \statref{shock-threshold} ever reaches zero, they die immediately.
This is unlikely to happen through repeated {\damage}, as an earlier blow would send them into {\shock}, but can occur if a lot of painkillers wear off all at once.

\subsection{Healing \& Recovery}
\seclabel{healing}

\capital{\damage} heals naturally over time, but it's a slow process.
Once per day, with a decent meal and at least about six hours of sleep, a character may recover from 1 point of {\damage}.
If the character takes an entire {\dayofrest}, they may heal 1 additional point of {\damage}, for a total of 2.
For a lightly wounded character, taking a stroll would be acceptable without disturbing a {\dayofrest}.
For a character with more serious wounds, they shouldn't move around too much, and may even require complete bed rest.

Tending by a healer can hasten the natural recovery process, but only provides any benefit if the character is taking an entire {\dayofrest}.
For each rank their physician has in the \skillref{healing} skill, a character taking a {\dayofrest} may heal 1 additional point of {\damage}.
A single healer can tend many patients in a day, up to about a dozen.
They may tend themselves, but only if their activities tending others do not prevent them from taking a {\dayofrest} themselves.

\subsection{Fire}
\seclabel{fire}

Although most civilised regions have come to respect witches, even if not to like them, there are still places where witch-burning is a time-honoured tradition.
And most towns still consider it appropriate punishment for a witch who goes to the bad.

A character who is on fire tracks the progression of the flames in stages, represented by a number of dice.
Each {\round}, at the beginning of their {\turn}, they suffer a {\damagetest} using this many dice.
\begin{itemize}
	\item \dice{1} represents a tiny flame, just licking at one edge of a piece of clothing.
		It can normally be extinguished very easily, without requiring an {\action}.
	\item \dice{2} represents a larger flame, perhaps a burning glove.
		It requires an {\action} to pat it out.
	\item \dice{3} represents a large portion of the character's clothing burning.
		Extinguishing it requires the character's entire {\turn}, as they stop, drop, and roll.
	\item \dice{4} represents the character largely aflame.
		Extinguishing it requires not only the character's {\turn}, but a {\test}.
	\item \dice{5} represents the character engulfed in an inferno.
		Extinguishing it requires extensive outside assistance, or a pond large enough to jump in.
\end{itemize}
The number of dice increases by 1, to a maximum of 5, each {\turn} that the character leaves their burning unattended.

A character need not begin the track at the beginning, depending on what ignites them.
Running through a burning building might begin them at \dice{3}, while being doused in oil before ignition would begin them at \dice{5}.

\subsection{Exhaustion}
\seclabel{exhaustion}

Besides injury, an active witch runs the risk of {\exhaustion}.
From late night vigils to running after tricksy spirits, many things can leave a witch tired and longing for her bed.

When a character performs something exhausting, or goes a day without at least 6 hours of sleep, the GM may apply a level of {\exhaustion}, or call for a {\test} (typically \attref{might} or \attref{will}) to avoid one.
Each level of {\exhaustion} reduces two of a character's attributes by 1.
The GM selects appropriate attributes depending on the type of {\exhaustion}.
For example, {\exhaustion} as a result of a long foot chase might decrease \attref{might} and \attref{grace}.
Sleep deprivation might decrease \attref{wit} and \attref{heed}.
A long day of socialising, rushing from meeting to meeting, might even reduce \attref{charm} and \attref{presence}.

Multiple levels of {\exhaustion} may reduce the same attribute, leading to a total reduction of 2 or more.
Whenever a character suffers a second level of {\exhaustion} affecting the same attribute, the GM may call for a {\test} to avoid passing out until they can sleep it off.
Reaching a total reduction of 5 on a single attribute can prove lethal.

A character may reduce their {\exhaustion} by 1 level when they get a good night's sleep: about eight hours.
An {\dayofrest} reduces {\exhaustion} by another level.
The player may choose which attributes to recover when they reduce their {\exhaustion}.

\section{Combat}
\seclabel{combat}

\subsection{Actions in Combat}
\seclabel{combat-actions}

\actiontype{Attack}{attack}{
	You attack a creature or object, with a \seclink{weapon}{weapons} or {\unarmed}.
	You must be adjacent to the target to attack with a melee weapon, or within the listed range of a ranged weapon.
	Make a {\test} using a number of dice determined, as normal, by your \skillref{weaponry} skill, and a flat bonus determined by your weapon's accuracy.
	The {\test} is made against a {\tn} equal to the target's \statref{dodge-rating}: 8, plus their \attref{grace}, plus their \attref{heed}.
	
	If you succeed in your {\test}, you hit.
	Make a {\damagetest} against the target, rolling dice as determined by your weapon and adding your \attref{might}.
}

\actiontype{Dash}{dash}{
	You may move an additional number of metres equal to your \statref{speed} this {\turn}.
}

\actiontype{Ready}{ready}{
	You don't act immediately, but prepare to take an {\action} later.
	Decide what {\action} you will take, and which circumstances trigger it.
	When those circumstances come around, you may choose to take the readied {\action} or not.
	If your next {\turn} comes around without you taking the readied {\action}, you lose the benefits of readying.
	You must \actionref{ready} again if you want to continue to wait.
}

\subsection{Combat Modifiers}

Some {\tests} in combat are made against fixed {\tns}, such as a \skillref{weaponry} {\test} against the target's \statref{dodge-rating}.
Instead of modifying these {\tns}, the GM may make such {\tests} easier or harder using the same modifiers as would be applied to {\opposedtests}.

For example, attacking a character who has fallen on the ground may grant a \positive{3} bonus.
Attacking after falling on the ground yourself, or with your feet stuck in thick mud, might impose a \negative{3} penalty.
Successfully sneaking up right behind someone might grant a \positive{6} bonus, or even allow an automatic hit if they're not aware they are in danger at all.

A helpless character---such as one who is in {\shock}, asleep, or whose mind has left their body---is automatically hit by any attack aimed at them.
Furthermore, a character with a melee weapon, who is not themselves being harassed by a nearby enemy, can simply kill a helpless character as an {\action}: slitting their throat with a knife, or bashing their head in with a rock.

\section{Magic}

Magic consists of too many diverse disciplines and effects to be effectively summarised in this section; indeed this is the entire topic of \partref{disciplines}.
However, a few general guidelines apply.

It is generally assumed that any witch who knows a spell, rite or technique has the knowledge and practice to pull it off consistently; doing so does not require a {\test} unless specified otherwise.
However, this practice only applies under normal conditions, with adequate time and materials.
A witch may attempt to rush her magic, perform it using whatever she has to hand, or to perform it in difficult conditions, and each of these requires a {\test}.
Such {\tests} typically use \attref{wit} and the relevant skill for the discipline of magic, but not always.
More formulaic disciplines such as \discref{brewing} and \discref{ritual-magic} often use \attref{ken}, while other disciplines, such as \discref{willing} and \discref{golemancy}, rely primarily upon a witch's pure force of \attref{will}.
Furthermore, drawing a chalk circle hurriedly might use \attref{grace}, and grinding a poultice while riding a broomstick might use \skillref{flying}.

\capital{\tns} for rushing or improvising magic are ultimately left up the GM, but some guidelines are provided below.

\subsection{Rushing Magic}

Generally, magic that would normally take at least an {\action} in combat cannot be performed in less than that time.
Exceptions may be made where the magic is used as part of the {\action} already being taken, to aid it or improve its effect, but the GM should still be careful allowing such things.
Otherwise, common sense may apply a limit to the minimum time magic can be performed in.
For example, if a potion requires boiling water, a witch needs some way to bring water to the boil in the time they want to brew their potion.

Where magic can be rushed, guideline {\tns} for doing so are given in the following table.

\begin{simpletable}{rX}
	\toprule
	\capital{\tn} & Example Task\\
	\midrule
	9 & Performing a simple rite in half the normal time.\\
	12 & Performing a complex rite in half the normal time.\\
	15 & Performing a simple rite in a tenth the normal time.\\
	18 & Performing a complex rite in a tenth the normal time.\\
	21 & Performing a simple 5 minute rite in one {\action}.\\
	\bottomrule
\end{simpletable}

\subsection{Improvising Materials}

This applies to both the tools used to conduct magic and the ingredients consumed by it, and works equally well in brewing and rites.
The most important part is that the witch can justify any substitution to herself.
From a gameplay perspective, this also means that the player should justify such improvisations to the GM.
This can be as simple as using a pool of water in place of a mirror, because both are reflective, or more extreme, such as using a fresh egg in place of blood, as both are the fluids of life.

\begin{simpletable}{rX}
	\toprule
	\capital{\tn} & Example Task\\
	\midrule
	9 & An unusual component that still meets the specifications, e.g.\ a ritual circle scratched in the dirt instead of drawn in chalk.\\
	12 & A component that retains the fundamental property, e.g.\ scrying through a pool of still water instead of a mirror.\\
	15 & A component that is close, but violates a specification, e.g.\ pig blood instead of human blood.\\
	18 & A component with a reasonable justification for relatedness, e.g.\ a fresh egg in place of blood.\\
	21 & A component with a weak justification for relatedness, e.g.\ apple juice in place of blood.\\
	\bottomrule
\end{simpletable}

\subsection{Consequences}

Magic is dangerous, especially when rushed or improvised.
The GM should feel free to reflect this in the consequences of failure on a magic {\test}, even when it is not a critical failure.
Failure on a magic {\test} need not indicate that nothing occurred, but might indicate that something unwanted or something rather tangential has occurred, or that the magic has succeeded, but with side effects.

For example, suppose a witch is attempting to brew a potion for hair regrowth, but has substituted several of the ingredients for similar ones they hoped would work.
A failure on the {\test} might mean that the potion successfully causes hair regrowth, but that the hair is the wrong colour, or grows in more places that desired.

Other magics can have even more dangerous consequences.
A witch trying to scry through a puddle instead of a mirror might, on a narrow failure, only get an unclear image as the puddle is disturbed by wind.
But a more dire failure could mean that the target instead sees the witch herself through any nearby reflective surfaces, or that the imperfect scrying draws the attention of \emph{things} from other dimensions that look, reach or even climb out of the puddle.
Rituals to summon demons and the like can obviously have some of the most dangerous consequences of all, should they go wrong.
