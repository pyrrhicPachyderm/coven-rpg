\chapter{Brewing}
\chaplabel{brewing}

\section{Creation and Application}

The \skillref{brewing} skill and \discref{brewing} discipline don't just cover potions brewed in a cauldron.
Potions, poultices, poisons, tinctures, salves, ointments, even beer, mead, wine and spirits.
Witches have many ways of turning \seclink{Herbs}{herbs}, and even other things, into more useful forms.

Each feat that allows a witch to prepare such a mixture lists the method of preparation and delivery.
The rules of such methods are presented below.

\subsection{Brewing and Chewing}

Different methods of preparation require different equipment, and take different periods of time.

\mixcreation{Cauldron}{cauldron}
Most potions are brewed in cauldrons, filled with water and brought to boil.
This requires, obviously, a cauldron, as well as a fire to heat it.
A smaller kettle might do in a pinch, but requires a Test.
A full cauldron will typically yield several doses.
Brewing in a cauldron requires around 15 minutes to bring the water to the boil, and another minute to mix the potion.

\mixcreation{Poultice}{poultice}
A poultice doesn't need to be brewed at all; the ingredients are simply chewed into a paste.
Some of the more dangerous poultices should definitely be ground with a mortar and pestle, however, rather than allowed anywhere near the mouth.
Creating a poultice requires less than a minute.

\mixcreation{Still}{still}
Some potions, or spirits, need to be distilled.
This requires quite a lot of dedicated equipment, a carefully maintained heat source, and several hours.

\subsection{Method of Delivery}

\section{Feats}

\feat{Numbing Painkiller}{painkiller-grace}{15}{
	None
}{
	\materials{\herbref[willow bark]{2}}
	
	The drinker may ignore 1 point of \secrefraw{damage} for a few hours, but loses one point of \attref{grace} for the same duration.
	Two doses may be effective simultaneously.
	Further doses cause paralysis, and possibly organ failure.
}

\feat{Dimming Painkiller}{painkiller-wit}{15}{
	None
}{
	\materials{\herbref[poppy seed]{2}}
	
	The drinker may ignore 1 point of \secrefraw{damage} for a few hours, but loses one point of \attref{wit} for the same duration.
	Two doses may be effective simultaneously.
	Further doses cause unconsciousness, and possibly cessation of breathing.
}

\feat{The Hard Stuff}{brewing-booze}{20}{
	None
}{
	You know how to make a drink that'll really put hairs on a man's chest.
	Or a woman's, at that.
	This potion is made in a still, instead of a cauldron.
	
	\materials{Alcohol, \herbref[apple]{2}}
	
	The drinker gains 1 \attref{might} for a few hours, and loses 2 \attref{wit} for the same duration.
	A second dose will render the drinker unconscious, and further doses are dangerously poisonous.
}

\feat{Healing Salves}{brewing-healing}{15}{
	\skillref[1]{brewing}
}{
	You know a wide range of minor poultices, salves and remedies for cuts, bruises and other physical injuries.
	As long as you have access to a reasonable supply of various \herbrefplural{2}, and time to chew up poultices, you may use your \skillref{brewing} skill in place of your \skillref{healing} skill on Tests to heal people and creatures of most physical injuries.
	Setting broken bones and performing surgery still requires \skillref{healing}.
	
	Similarly, you may use your \skillref{brewing} rank in place of your \skillref{healing} rank when determining the \secrefraw{damage} healed by a patient during a day of rest.
}
