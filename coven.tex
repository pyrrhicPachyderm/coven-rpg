\documentclass[a4paper,10pt,twocolumn]{book}
\usepackage[all]{nowidow}
\usepackage{amsmath}
\usepackage{siunitx}
\usepackage{enumerate}

%\usepackage[margin=0.9in]{geometry}

\usepackage{titling}
%\setlength{\droptitle}{-0.5in} %Adjust margin above title

\usepackage{titlesec}
\titleformat{\chapter}[hang]{\normalfont\huge\bfseries}{\chaptertitlename\ \thechapter:}{1em}{} %Chapter number and name on same line
%TODO: This justifies split lines awfully, fix it.

\makeatletter
\renewcommand{\@seccntformat}[1]{} %Remove section numbers
\makeatother

\usepackage{tocloft}
\makeatletter
\renewcommand{\cftsecpresnum}{\begin{lrbox}{\@tempboxa}} %Removes section numbers from the table of contents.
\renewcommand{\cftsecaftersnum}{\end{lrbox}} %Latter half of the above.
\makeatother
\setlength{\cftsecnumwidth}{0pt} %Removes the space reserved for the section numbers in the table of contents.

\usepackage{hyperref}
\usepackage{nameref}

\usepackage{graphicx}
\usepackage{float} %Provides the float option [H] for a non-floating float.
\graphicspath{{./imgs/}}

\usepackage{hyperxmp} %Recommended for doclicense
\usepackage[
	type={CC},
	modifier={by-nc-sa},
	version={4.0},
]{doclicense}

\usepackage{booktabs}
\usepackage{tabu}
\tabulinesep=1.2mm %Space tables more nicely.
\usepackage[table]{xcolor}
\newenvironment{simpletable}[1]{
	\begin{center}
	\rowcolors{2}{}{gray!20}
	\begin{tabu}{#1}
}{
	\end{tabu}
	\end{center}
}


\newcommand\partlabel[1]{\label{part:#1}}
\newcommand\chaplabel[1]{\label{chap:#1}}
\newcommand\seclabel[1]{\label{sec:#1}}
\newcommand\partref[1]{Part~\ref{part:#1}}
\newcommand\chapref[1]{Chapter~\ref{chap:#1}}
\newcommand\chapnameref[1]{\nameref{chap:#1}}
\newcommand\secref[1]{the \nameref{sec:#1} section}
\newcommand\seclink[2]{\hyperref[sec:#2]{#1}}

\newcommand\feat[4]{ %Arguments: title, label, prerequisites, text.
	\subsection{#1}\label{feat:#2}
	\textbf{Prerequisites:} {#3}
	
	{#4}
} %TODO: XP cost, tags such as 'first circle'?
\newcommand\featref[1]{\nameref{feat:#1}}
%TODO: Make feat references add the chapter in brackets afterwards, if the feat is in a different chapter.

\newcommand\skilllabel[1]{\label{skill:#1}}
\newcommand\skillref[2][0]{%Takes two arguments, the first of which is optional and defaults to zero.
	\nameref{skill:#2}%
	\ifnum #1=0%
		%
	\else%
		~#1%
	\fi%
}
\newcommand\skillrefspecialty[3][0]{%Takes three arguments, the first of which is optional and defaults to zero.
	\nameref{skill:#2} (#3)%
	\ifnum #1=0%
		%
	\else%
		~#1%
	\fi%
}

\usepackage{amsfonts} %Gives the stuff we use to build \shortminus
\DeclareMathSymbol{\shortminus}{\mathbin}{AMSa}{"39}

\usepackage{xstring} %Gives \StrDel
\newcommand\dice[2][0]{%Takes two arguments, the first of which is optional and defaults to zero.
	#2d%
	\ifnum #1=0%
		%
	\else%
		\ifnum #1>0%
			$+$%
		\else%
			$\shortminus$%
		\fi%
		\StrDel{#1}{-}%Strips the minus from it, essentially giving absolute value.
	\fi%
}

\newcommand\negative[1]{$\shortminus$#1}

\newcommand\titleemph[1]{\emph{#1}} %For the titles of books and such, including Coven itself.

\newcommand\storybreak{\bigskip}

\title{Coven: An RPG of Witches}
\author{Christopher Brown}
\date{}

\begin{document}
\pagestyle{plain} %Fixes page numbers appearing in the top left on empty pages.

\maketitle

\include{edition-notice}

\setcounter{tocdepth}{1} %No subsections or deeper.
\tableofcontents

\chapter{Introduction}
\chaplabel{introduction}

%TODO: Missing \titleemph here, due to \dropcap.
\dropcap{Coven} is a role-playing game designed upon a simple premise: the player characters are witches and the party is a coven.
Every character shares the common tools of witchcraft: a familiar, a broomstick and, most importantly, a pointed hat.
However, that is often where the similarities end.
There are many different disciplines to witchcraft, and many different approaches even within a discipline.
From meticulous ritualists to soaring broom-riders, from shy girls to terrifying matriarchs, hunched over a cauldron or chatting with squirrels in the forest, a coven can be a diverse lot.

In light of this, they don't always get along.
Witches can be somewhat solitary creatures by nature, tending to their own villages, dealing with their own problems.
But they do tend to keep tabs on one another, and a good witch recognises when things are bit much to handle by herself.
When the great spirits of the land are threatened, when \emph{things} push through from other realms, or when one of their own begins to cackle: these are the times witches come together.
And these are the adventures the players have with them.

\section{The Craft}

The Craft, the Art, the Way.
Witchery, occultism, thaumaturgy.
There are many names for witchcraft.
Few things define it, however.
In truth, it is nothing but knowledge of the diverse disciplines of magic, and the skill to apply it.

Witchcraft is not like the enchantments of the faeries or the sorcery of warlocks.
It's not a power one is born with, nor one absorbed in a moment.
It is learned through years of training, grasped through decades of practice, and never truly mastered.
Anyone can pick it up, given enough patience and determination.
But few even have the inclination.

For the power of witchcraft comes with more responsibility than most.
The responsibility to care for one's neighbours, one's charges, one's village.
To see them through sickness and through strife, to see them into the world and back out of it.
The responsibility to take up arms and defend them from the horrors of the night, of other realms, even the ones they bring upon themselves.
To lay down one's own life in defence of others.
And finally, the responsibility to train a successor, that the Craft may continue to serve one's village after one dies.
Everything that goes on in a witch's realm is her responsibility, and that is too great a burden for many to bear.

Which brings us to the topic of the Black Craft.
Witchcraft is simply knowledge, to be used how it will.
Even possession, voodoo, and \discref{necromancy} are not evil acts in themselves, when turned to the purpose of good. %TODO: Link for voodoo
Evil begins when all the responsibility becomes too much for a witch.
When she wonders why she's doing so much for these people who never do anything for themselves.
When she believes that she is better than other people.
When she begins to cackle.
And so comes another responsibility of witches: to sit her down and give her a stern talking to.
Or, failing that, to show her the way out{\dots}

\section{The World}

%TODO: Assumptions of the setting.

\section{Tabletop Role Playing}

\section{Dice and Tests}

Like most tabletop role-playing games, \titleemph{Coven} uses dice to determine the result of certain actions.
\titleemph{Coven} uses only six-sided dice (d6s), which you should be able to pilfer in abundance from a few board games.

Whenever the action of an outcome is in doubt, the GM may call for a {\test} by the acting character, specifying an attribute and optionally a skill with which to make the {\test} (attributes and skills are explained in \chapref{attributes-and-skills}).
A {\test} is made by rolling a number of six-sided dice determined by the character's skill and adding the highest 3 of these dice together
The relevant attribute is then added to the total, which is compared to a {\targetnumber} ({\tn}) provided by the GM.
If the total meets or exceeds the {\tn}, the {\test} has succeeded.
Otherwise, the {\test} has failed.
Sometimes, two characters will make directly {\opposed} {\tests}.
Such a {\test} has no {\tn}, and the character with the higher total succeeds.
In the event of equal totals, the situation remains as it was before the {\test}, so far as possible.

The number of dice rolled for a {\test} is typically determined by a character's skill.
If a character is unskilled at a task, or there is no applicable skill for it, she rolls 3 dice (and hence keeps all of them).
If she has some relevant skill, she rolls 4.
If she is an expert in the applicable skill, she rolls 5.
If she is a veritable master of the skill, she rolls 6.
%TODO: Using numbers or names for skill ranks?

Sometimes, a character does more than simply succeed, she excels.
And sometimes, she fail catastrophically.
These are represented by critical successes and failures.
If every rolled die shows a 1 or a 2, the {\test} is a critical failure.
If the highest 3 dice all show 6, the {\test} is a critical success.
In addition to the {\test} automatically succeeding or failing, the GM is encouraged to apply an additional benefit or drawback to the result of the {\test}.
Critical failures on {\tests} involving dangerous magic can be especially catastrophic.

More details on {\tests}, including examples and prescribed tests for particular situations, can be found in \partref{rules}. %TODO: Ensure that's all there.


\part{Character Creation}
\partlabel{character-creation}

\chapter{Character Creation Guide}
\chaplabel{character-creation-guide}

\section{Step Zero: Character Concept}

The most important part of a character is the concept.
Who is your character, what does she do?
You can try to flesh this all out now, or fill it in as you work through character creation.
Make sure your character is somebody you will enjoy roleplaying.

Also make sure your witch fits into the coven; discuss this with the other players and your GM.
It can be quite painful for everyone involved playing an unscrupulous necromancer in a coven of saccharine healers, or vice versa.
The GM should provide some idea of the tone intended for the game, to avoid this sort of trouble.
Diversity can also be good: make sure you know what you're letting yourselves in for if everybody in the group wants to play a potion-brewer.

\section{Step One: Attributes}

Attributes are a witch's broad, innate capabilities.
Is she skinny and lithe or broad and well-muscled, quick-witted or bullheaded, domineering or silver-tongued?

At character creation, you have 15 points to spend on your character's attributes.
Spend these points on each of the six attributes, setting each attribute to between 0 and 4, inclusive.

\section{Step Two: Starting Experience}

Your GM will assign you an amount of experience (XP) to use during character creation.
By default, this is 15 XP, though the GM is free to adjust this to suit a different style of characters or campaign.

15 XP is suitable for witches who have just completed their apprenticeship and are taking over their own steading.
7 XP might be more suitable for witches who are still in an apprenticeship, and may have only just bound their familiar.
30 XP might be appropriate for witches with a few years of caring for a steading under their belt.
Even more experienced witches might require even more starting XP.
For fairness, the GM should probably give all characters the same starting XP, unless there is a good reason otherwise.

\section{Step Three: Skills}

A witch does not begin totally unskilled.
Select 1 \seclink{general skill}{general-skills} (a skill with no governing discipline); you begin with 2 ranks in this skill.
Select an additional 3 general skills; you begin with 1 rank in each of these.
Lastly, select 1 \seclink{discipline skill}{discipline-skills} (a skill with a governing discipline); you begin with 1 rank in this skill.

The GM is also free to adjust the number of skills and ranks granted to starting characters.
General skills represent general life experience, while discipline skills represent experience with magic and witchcraft.
However, acquiring more than 1 rank in a discipline skill without learning magic from the associated discipline is very rare; such ranks ought to be acquired through XP rather than granted at character creation.

\section{Derived Statistics}

\begin{simpletable}{ll}
	\toprule
	Statistic & Derivation\\
	\midrule
	Resilience & $(5 + \text{\attref{might}}) \div 2$\\
	Shock Threshold & $12 + \text{\attref{will}}$\\
	Speed & $8 + \text{\attref{might}} + \text{\attref{grace}}$\\
	\bottomrule
\end{simpletable}

\section{Steading}

Most witches have a steading.
This is the area a witch watches over, a region she defends and protects the inhabitants of.
The duties a witch has to her steading are numerous and varied, but typically involve healing the inhabitants and protecting them from threats of a magical nature.
Some witches also perform midwifing, care for the land itself, or even take it upon themselves to deal with non-magical threats, such as invading armies.
A witch's responsibilities are not limited to her steading, and nothing stops her from responding to threats outside it.
But inside it, everything is certainly her responsibility.

Decide whether your witch has a steading.
How big is it?
One village, several, or an entire kingdom?
What duties does she perform within it?
Do the inhabitants appreciate what she does for them?

Also discuss this with your GM, and the other players.
Has the GM already described a village that could be your steading?
It is not unheard of for witches to share a steading, although this can obviously lead to disagreements.
Do you share a steading with your coven, or have you carved the local region into one steading each?


\chapter{Attributes and Skills}
\chaplabel{attributes-and-skills}

\section{Attributes}

\section{Skills}

%TODO: Intro fluff paragraph about the skill of a witch.

%TODO: Recap how skills affect dice rolled for Tests, and how not every Test has an applicable skill.

Each skill is governed by one of the disciplines of witchcraft, and a witch typically improves her skill alongside the relevant discipline.
Some skills are nonetheless general skills, pertaining to things any witch might find herself doing.
Others are of little use to a witch who does not practice such a discipline, although they can often be used to identify, and sometimes to counteract, the effects from it.

A list of the skills available to a witch, alongside their governing discipline and examples of their use, is provided below.
Details on increasing skills can be found in the chapters on each discipline in \partref{disciplines}.

\newcommand\govdisc[1]{Governing discipline: #1} %TODO: Make these link to the relevant chapter, and insert the title of that chapter automatically.

\subsection{Botany}
\govdisc{Botany}

Used to raise crops and herbs in a witch's garden, find them out in the forest, or identify a fishy-looking leaf.

\subsection{Broomcraft}
\govdisc{Broomcraft}

Used by a witch on a broomstick, whether she's settling in for a cross-country flight, showing off with a barrel roll, or pulling a stalled stick out of a deep dive.

\subsection{Deceive}
\govdisc{Headology} %TODO: Rename

Many witches make it a rule not to lie.
That doesn't mean they always need to tell the whole truth, so this can still be a useful skill for them.

\subsection{Perception}
\govdisc{Divination}

Used by the uninitiated to see their present surroundings: to spot things out of place or to pick out details at a distance.
Also used to see the past and future, and places many miles away, for those who know how.

\subsection{Necromancy}
\govdisc{Necromancy}

Used to pervert the natural order and bring the dead back to life.
Also used to send them on again, if hitting them over the head with a big stick won't suffice.

\subsection{Weaponry}
\govdisc{} %TODO

Used for everything from stabbing people with a concealed knife to clonking them over the head with a hefty staff, or even slugging them with a mean right hook.


\chapter{Familiars}
\chaplabel{familiars}

\dropcap{A} wizened old woman leans back in her rocking chair, eyes closed.
A white cat lies curled in her lap, its own eyes also shut, purring as she rubs its chin.

A handsome, tanned woman stands on the peak of a grassy hill, arm held aloft.
A falcon dives from above, alighting on her thick leather glove.
It casts its eyes north-west, then knowingly back at the witch.
With a sly grin, she throws the bird back into the air and strides downhill after it.

A bright-eyed girl, no more than thirteen, stands beside a bubbling cauldron, carefully teasing the seeds from a pine cone with a small knife.
``Sage leaf next, Harold?''
She looks up at the frog on the kitchen bench, as it croaks and nudges one of the piles of herbs that surrounds it.
``Ohh, right. Rosemary. Of course{\dots}''
The girl shakes her head and tuts to herself as she counts out seven leaves into her hand and drops them in the cauldron.
Harold peers over from the bench, keeping a close eye on the brew as it slowly turns a deep blue.

\section{No Mere Beast}

A witch's familiar is no mere animal.
It is a fusion of a tamed beast, and a tiny sliver of soul from the witch herself.
Obtaining a familiar is one of the first steps for any witch-in-training, and the familiar often aids in the witch in her subsequent learning.

Familiars are intelligent creatures, in some cases even more intelligent than the witches they are bound to.
They understand language, though the limits of animal form mean that most are incapable of speech.
Despite this, the bond that a witch shares with her familiar allow them to communicate.
With simple looks and gestures a familiar can communicate great meaning to its witch, communicating as effectively as if through speech.
This ability does not extend to other witches, and especially not to layfolk, who may require a {\test} to interpret a familiar's communication.
Pointing and beckoning are typically fairly unambiguous, however.

A witch's communication with her familiar allows her to lean on its expertise when her own is lacking.
A witch may use her familiar's ranks in a skill in place of her own, as long as the {\test} takes at least a minute, and she can confer with her familiar through the duration.

\section{Binding a Familiar}

Binding a familiar takes place in a simple ritual: achievable by even the most novice witch, though often performed under direct tutelage.
It requires a \circleref{small}, and takes an hour to perform.
The animal to become the familiar must be tamed by the witch beforehand, at least enough that it willingly remains by her side throughout the ritual.
Many witches find this to be the hardest part of binding their familiar, and it means that some animals make for rather rare familiars.
Lastly, the witch must offer up a sliver of her own {\soul}, to seal the bond.
She does so by feeding the familiar animal a drop of her own blood.

Upon completion of the ritual, the animal and the {\soul}-sliver are fused to form a new entity, the familiar.
It retains many traits from the animal, but becomes decidedly more human in personality.
Slight changes to its physical form often manifest, such as a coat that always remains strangely glossy, a slight chill to the touch, or sharper, whiter teeth.
Changes in eye colour are especially common.
Lastly, the sliver of the witch's {\soul} included in the creation of a familiar also influences its personality.
It ensures that, although a witch and her familiar may not always get along, and may certainly disagree on the best way to go about something, they will always have each other's best interests at heart.

\section{Creating a Familiar}

From the perspective of character creation, there are many things to bear in mind when creating a familiar.
While the familiar is unlikely to take the foreground as much as the witch herself, they are still a character in their own right, and should be designed as such.

The most important decision is the form the familiar will take, the animal they were created from.
This determines the familiar's attributes, skills and abilities.
Note that familiars, as non-human characters, may have attributes below the human 0 to 5 range.

Beside its game statistics, it is also important to get an idea of your familiar as a character.
Try answering some of the following questions.

\begin{itemize}
	\item What is your familiar's name?
	\item Is your familiar male or female?
		Do you not know?
	\item At what stage in her life, and her training, did your witch bind her familiar?
	\item Do your witch and her familiar get along?
		Do they engage in playful banter?
		Philosophical debate?
	\item Does your familiar have any quirks, physical or mental?
\end{itemize}

Lastly, it is important to decide whether each familiar will be played by the player or the GM.
Both are valid, but if the GM is playing familiars they should typically act in their witch's best interests.

\section{Familiar Injury and Death}
\seclabel{familiar-injury-death}

Familiars suffer {\damage}, {\shock} and death just like other characters.
A witch is always aware when her familiar dies, feeling it as a searing pain in her very soul.
It is even worse the other way around, however.
If a witch dies and her {\soul} departs the world, it tears free the shard of {\soul} bound in the familiar, killing it.

It is possible to recover a deceased familiar.
This requires the witch to tame another animal of the same kind, and repeat the original binding ritual upon it.
The familiar's {\soul} and personality are entirely restored, and it may take either the new animal's appearance or its original one.

Repeating the ritual takes another sliver of the witch's {\soul}, provided through another drop of blood.
As such, recovering a deceased familiar costs 10 XP every time.

Familiars tend to age more slowly than a normal animal of their kind would, growing old and dying at the same time as their witch does.

\section{Familiar Fighting}

Some witches will be inclined to have their familiars fight for them.
This is perfectly valid; some familiars will even make better fighters than their witches.
In {\structuredtime}, familiars take their {\turn} at the same time as their witch, using their witch's {\initiative} score.

Familiars cannot normally use weapons, what with lacking hands.
Some can make {\unarmed} attacks, and the number of dice they roll for their {\damagetests} are given in their list of abilities.
If this is not given, they have no effective attacks.

``No effective attacks'' does not mean that they cannot fight at all---a \familiarref{rat} still has teeth.
It simply means that any fight between one of these familiars and one \emph{with} effective attacks is a foregone conclusion.
If two creatures without effective attacks end up in a fight, it may be prudent to reduce several minutes of fighting to one opposed {\test}, or a handful of them.

\section{Familiar Animals}

A list of the types of animal available as familiars is presented below, along with the attributes, skills and abilities of the familiar.
Besides the abilities listed below, the players and GM are encouraged to apply common sense.
For instance, familiars lack thumbs and will struggle with door handles, and a weasel can squeeze through a smaller hole than a hound.

If you would like your familiar to be an animal not presented on the list below, discuss your option with your GM.
It might be possible to design a new familiar for you to use, or to use the statistics of a familiar presented here to represent something else.
Note that familiars are fairly small animals; the exclusion of anything larger than a medium-sized dog is intentional.

Many types of familiar---more powerful ones---come with an associated XP cost.
This is deducted from the witch's starting XP.
Some types of familiar also come with options which may be purchased for an additional XP cost.
These represent inherent differences in the animal used and must be purchased at the same time your familiar is created.
You may only select one option; they are mutually exclusive.

Lastly, bear in mind that some feats that can be purchased later depend upon particular types of familiar, and your familiar's later development is limited by its form.
As such, it can be worth taking a quick look at other feats you may be interested in taking when selecting your familiar.
%TODO: If those are all in a discipline chapter on familiars, direct people there.

\familiar{Bat}{Bats}{bat}{15}{
	\atttable{\negative 2}{2}{1}{2}{2}{2}{0}{0}
}{
	\speed{2}, \flyspeed{15}
}{
	\skillref[1]{divination}, \skillref[1]{flying}, \skillref[1]{perception}, \skillref[1]{stealth}
}{
	Lurkers in darkness, and nocturnal fliers, \familiarrefplural{bat} fit right in with a certain type of witch.
	Their echolocation not only makes them excellent scouts in dark caves, but also lends them a slight natural talent with \discref{divination}.
}{
	\ability{Echolocation}{
		The \familiarref{bat} can sense perfectly in darkness, or even when blinded, using its echolocation.
		This works within about 50 metres.
		The sounds it produces are beyond the hearing range of humans, birds, fish, and amphibians, but can be detected by smaller mammals, such as cats, dogs and rats, as well as some insects.
	}
	
	\familiaroption{Vampire Bat}{5}{
		The \familiarref{bat} gains \skillref[1]{weaponry}, and it rolls 1 die for {\unarmed} {\damagetests}.
		Its bite is painless, and can go unnoticed by the victim.
		A full feed (one dose of \materialref{blood}) takes about 30 minutes, but it draws enough \materialref{blood} to use as a \materialref{taglock} in just one {\action}---even if it deals no {\damage}.
		It can regurgitate any \materialref{blood} it drinks at any time within the next few hours.
	}
}

\familiar{Cat}{Cats}{cat}{20}{
	\atttable{\negative 2}{3}{2}{2}{2}{2}{3}{1}
}{
	\speed{10}
}{
	\skillref[1]{athletics}, \skillref[1]{deception}, \skillref[1]{perception}, \skillref[2]{socialising}, \skillref[2]{stealth}
}{
	Graceful and charming on the outside, \familiarrefplural{cat} can be incredibly sly and manipulative underneath.
	Just like many witches.
}{
	\ability{Natural Acrobat}{
		The \familiarref{cat} rolls an extra die on {\tests} to jump, retain its balance, land on its feet, or avoid {\damage} from falling.
		%TODO: Reduce fall damage in some more definite fashion?
	}
	
	\ability{Cat's Eyes}{
		The \familiarref{cat} can see excellently in the dark.
		It suffers no penalties in low-light conditions, though it is as blind as anyone in complete darkness.
	}
	
	\ability{Claws}{
		The \familiarref{cat} rolls 2 dice for {\unarmed} {\damagetests}.
	}
}

\familiar[Crow/Raven/Magpie]{Crow}{Crows}{crow}{10}{
	\atttable{\negative 2}{2}{2}{3}{2}{2}{1}{1}
}{
	\speed{2}, \flyspeed{15}
}{
	\skillref[1]{divination}, \skillref[1]{flying}, \skillref[1]{necromancy}, \skillref[1]{perception}
}{
	\capital{\familiarrefplural{crow}} and other corvids are the smartest birds, and among the smartest animals of all.
	Many people see their appearance as an omen, typically of ill fortune, or death.
	If they're followed by a witch, this might even be the case.
}{
	\ability{Thief of Glitter}{
		The \familiarref{crow} rolls an extra die on \skillref{perception} {\tests} to spot shiny objects.
	}
}

\familiar{Dog}{Dogs}{dog}{20}{
	\atttable{1}{1}{1}{1}{3}{2}{1}{2}
}{
	\speed{12}
}{
	\skillref[2]{intimidation}, \skillref[2]{perception}, \skillref[1]{weaponry}
}{
	A man's best friend, and often a witch's too.
	\capital{\familiarrefplural{dog}} are a diverse lot, including hunting dogs, sheepdogs, sled dogs and more.
}{
	\ability{Bite}{
		The \familiarref{dog} rolls 5 dice for {\unarmed} {\damagetests}.
	}
	
	\familiaroption{Scenthound}{5}{
		The \familiarref{dog} rolls an extra die on \skillref{perception} {\tests} relying on smell.
	}
	
	\familiaroption{Greyhound}{5}{
		The \familiarref{dog} has a \statref{speed} of \speed{20}.
	}
	
	\familiaroption{Sheepdog}{5}{
		The \familiarref{dog} has \skillref[2]{animals}.
	}
}

\familiar[Ferret/Stoat/Weasel]{Ferret}{Ferrets}{ferret}{15}{
	\atttable{\negative 2}{3}{1}{1}{2}{2}{1}{0}
}{
	\speed{8}
}{
	\skillref[1]{athletics}, \skillref[2]{stealth}, \skillref[1]{weaponry}
}{
	A \familiarref{ferret}, stoat, weasel, polecat, ermine, mink, or marten.
	Despite their small size, these creatures are ferocious predators.
	Their long, narrow bodies allow them to invade the burrows of much smaller animals, or the trousers of their witch's unfortunate foes.
}{
	\ability{Bite}{
		The \familiarref{ferret} rolls 2 dice for {\unarmed} {\damagetests}.
	}
	
	\ability{Slippery}{
		The \familiarrefpossessive{ferret} \statref{dodge-rating} is increased by 2.
	}
}

\familiar[Frog/Toad]{Frog}{Frogs}{frog}{0}{
	\atttable{\negative 3}{\negative 1}{1}{1}{2}{0}{\negative 1}{\negative 1}
}{
	\speed{4}, \swimspeed{4}
}{
	\skillref[1]{brewing}
}{
	\capital{\familiarrefplural{frog}} and toads make excellent companions to brewing witches, due to their natural affinity with water.
	Particularly with some of the stuff that gets into the murkier ponds around{\dots}
	
	It is important to try and keep their skin moist, but maybe refrain from dropping them in the cauldron.
}{
	\ability{Amphibian}{
		The \familiarref{frog} can breathe underwater.
	}
	
	\ability{Leapfrog}{
		The \familiarref{frog} can jump at least 3 metres from a standing start.
		It rolls an additional die on {\tests} made to jump.
	}
	
	%TODO: Something about keeping the skin moist? Maybe when there are exhaustion/fatigue rules.
}

\familiar{Owl}{Owls}{owl}{15}{
	\atttable{\negative 2}{2}{3}{2}{2}{3}{1}{1}
}{
	\speed{2}, \flyspeed{12}
}{
	\skillref[1]{flying}, \skillref[2]{perception}, \skillref[1]{stealth}
}{
	The great wisdom of \familiarrefplural{owl} makes them well suited to those witches who enjoy a spot of intellectual conversation, and can't find any other humans who seem to be up to it.
	They also make excellent nocturnal scouts, and even hunters.
}{
	\ability{Night Eyes}{
		The \familiarref{owl} can see excellently in the dark.
		It suffers no penalties in low-light conditions, though it is as blind as anyone in complete darkness.
	}
	
	\ability{Swooping Talons}{
		The \familiarref{owl} rolls 2 dice for {\unarmed} {\damagetests}, or 3 dice when striking from a dive.
	}
}

\familiar[Raptor (Eagle/Falcon/Hawk)]{Raptor}{Raptors}{raptor}{25}{
	\atttable{\negative 1}{3}{1}{2}{2}{3}{\negative 1}{2}
}{
	\speed{2}, \flyspeed{20}
}{
	\skillref[2]{flying}, \skillref[2]{perception}, \skillref[1]{weaponry}
}{
	\capital{\familiarrefplural{raptor}} include buzzards, eagles, falcons, harriers, hawks, kites, and osprey; birds of prey.
	They are excellent fliers, have keen eyesight, and nobody would want to tangle with their wicked beak and talons.
	Among the nobility, falconry is largely a status symbol, but a witch with a \familiarref{raptor} for a familiar has herself a great asset.
}{
	\ability{Eagle Eyes}{
		The \familiarref{raptor} rolls an extra die on \skillref{perception} {\tests} to see things at a long distance.
	}
	
	\ability{Beak \& Talons}{
		The \familiarref{raptor} rolls 3 dice for {\unarmed} {\damagetests}, or 4 dice when striking from a dive.
	}
}

\familiar[Rat/Mouse]{Rat}{Rats}{rat}{0}{
	\atttable{\negative 3}{1}{1}{1}{2}{1}{\negative 2}{\negative 1}
}{
	\speed{8}
}{
	\skillref[1]{stealth}
}{
	The \familiarref{rat} is a rather widely reviled animal, but it's certainly easy for a new witch looking for a familiar to find one.
	And it can get into smaller places than a \familiarref{cat} or \familiarref{owl}, which often proves helpful.
}{
	\ability{Filth-Liver}{
		The \familiarref{rat} rolls an extra die on {\tests} to resist poison or disease.
	}
	
	\ability{Keen Smell}{
		The \familiarref{rat} rolls an extra die on \skillref{perception} {\tests} relying on smell.
	}
}

\familiar{Songbird}{Songbirds}{songbird}{5}{
	\atttable{\negative 3}{2}{1}{2}{2}{1}{2}{0}
}{
	\speed{2}, \flyspeed{15}
}{
	\skillref[1]{flying}, \skillrefspeciality[2]{performance}{Singing}
}{
	\capital{\familiarrefplural{songbird}} include sparrows, larks, robins, wrens, thrushes, warblers, nightingales, and countless other types of bird.
	They fill the forests and woods, and a witch seeking one needs only to follow the sound of singing.
}{
	\ability{Songspeak}{
		The \familiarref{songbird} may use its song---which can carry for a few hundred metres---to communicate with you, just as effectively as through speech.
		If it learns to communicate with any other creatures, such as through \featref{familiar-language}, \featref{familiar-language-2}, \featref{familiar-language-3}, or \featref{familiar-language-animals}, it can use its song for that too.
	}
}

\familiar{Spider}{Spiders}{spider}{5}{
	\atttable{\negative 5}{2}{1}{1}{1}{2}{\negative 2}{\negative 1}
}{
	\speed{2}
}{
	\skillref[1]{intimidation}, \skillref[2]{stealth}
}{
	Crawling up walls, hanging from ceilings, and so tiny as to avoid notice, \familiarrefplural{spider} make excellent spies.
	They also terrify some people, which can often prove handy.
}{
	\ability{Web Spinner}{
		Given about an hour, the \familiarref{spider} can spin a cobweb.
		This does nothing to creatures of much size, but traps insects and the like on contact.
		Particularly dense webbing can also obstruct vision.
	}
	
	\ability{Spider Climb}{
		The \familiarref{spider} can move at full speed over walls and ceilings, at no risk of falling.
		The \familiarref{spider} can also hang from a surface on a single strand of web silk.
		It spools out this strand, or climbs up it, using its usual \statref{speed}.
	}
	
	\familiaroption{Venomous}{5}{
		The \familiarref{spider} may inject venom with a bite.
		The bite is not immediately harmful, and in some cases, may go unnoticed.
		Intense pain at the site of the bite begins after about 5 minutes.
		Sickness, including nausea, vomiting, and weakness, develops over the next hour, and may last several days.
		Victims must make a \attref{might} {\test} to determine the severity and duration of symptoms.
		Critical failure on this {\test} leads to death.
	}
	
	\familiaroption{Silk Sailor}{5}{
		The \familiarref{spider} is light enough to be carried on the wind, ballooning on a single thread of gossamer.
		It gains \skillref[1]{flying}.
		As an {\action}, it can spool out this gossamer thread, and take off.
		It will only fly in moderate or strong winds, and cannot control the direction of its flight; it goes where the wind blows.
		However, it can travel many hundreds of miles this way, moving quite quickly in strong enough winds.
	}
}

%TODO:
%Beaver
%Chicken
%Chipmunk
%Dove
%Gecko
%Goose
%Lemming
%Lizard
%Mole
%Otter
%Parrot
%Pigeon
%Praying Mantis
%Rabbit/Hare
%Red Panda
%Salamander/Newt
%Seabird
%Swan
%Tortoise
%Vole/Shrew/Gopher
%Vulture
%Woodpecker


\chapter{Tools of the Craft}
\chaplabel{equipment}

\section{The Hat}

A witch's pointed hat is the most important of her tools, in many regards.
There are no particular rules about the hat; its effects are left up to the GM.
But it always has an effect on people.
It may make them angry, reverent, reassured or afraid, but most importantly it makes sure they know that they are in the presence of a witch.

A witch's hat says a lot about her, particularly to other witches.
When you create your character, you can answer the following questions about your hat.

\begin{itemize}
	\item Did you make it yourself?
	\item How tall is it?
	\item Is it the traditional black, or some other colour?
	\item How long have you had it?
		Is it visibly worn?
		Well cared for?
	\item Is it plain, tastefully decorated, or covered in stars and sequins?
	\item Does it have any useful accessories?
		Pockets?
\end{itemize}

Many witches accompany their hats by a black cloak or other such attire.
Opinions on occult jewellery are mixed: some witches wear masses, others frown on it heavily.



\section{Broomsticks}

Sometimes, walking from one village to another just takes too long.
A lot of witches---to maintain their mystique or simply because the townsfolk wouldn't be happy otherwise---even choose to live quite a way from the nearest village.
Such circumstances make a broomstick an essential accessory for any witch.

Broomstick flight is no mean feat and while every witch picks up the rudiments, most can use it for nothing more than getting from A to B.
The broom needs a running start, has to be ridden sidesaddle, and has a turning circle several hundred metres across.
Detailed rules for flying a broomstick can be found in \chapref{broomcraft}.

Before it can be used, a broomstick needs to be trained to to fly.
This requires someone to fly it around on another broomstick so that it can learn its craft from one of its fellows.
It must be held parallel to the broom being ridden, to ensure it learns to fly in the correct direction.
The process takes about eight hours.
These hours need not be consecutive, but should all be done within a couple of weeks.
Once trained, a broom retains its flight skill for a long time.
Taking it out for a few hours each year is enough to keep its hand in.

At character creation, every witch is assumed to own a trained broomstick one way or another.
It was probably trained using the broom of whoever taught her witchcraft, at least if she's still using their first broom.
It might feel like an old friend at this point, the witch familiar with every knot and notch in its handle.
A more careless witch might have gone through a few brooms during her career.



\section{Common Magical Components}

The various rites and magics of the various disciplines of witchcraft require too many different materials to enumerate here.
However, there a few components that make a regular appearance.
Some details of their acquisition, construction and use are given here.

\subsection{Ritual Circles}

A ritual circle describes any large arrangement of symbols or shapes required by a rite.
They are traditionally drawn on the floor in chalk, but other methods are far from uncommon; the visibility and accuracy are the only important aspects for most rites.
Some witches use paint for permanence, or even chisel their circles into stone.
Many a witch in a hurry has scratched their circles into the dirt with the toe of their boot.
Some witches even embroider their circles upon sheets of fabric that can be rolled up and laid down where needed.
However, a roll bearing even the smallest of circles is most of the height of a man.

Each rite requires a ritual circle of a particular design, different for every rite, but the same each time the rite is performed.
This means that scribing a circle just once and using it for many performances of the rite is a common practice.
Ritual circles are not even universally circular, although it is the most common shape and almost all have some sort of symmetry.
Squares, triangles and hexagons are not uncommon, and pentagrams are particularly common in certain disciplines.

Ritual circles are classified primarily by their size.
\begin{itemize}
	\item A small ritual circle can be scribed entirely in arm's reach while standing in one spot.
		It can comfortably be drawn in a couple of minutes.
	\item A medium sized circle is a few paces across.
		Most houses should have a room large enough to draw one in, if the furniture is moved.
		It can comfortably be drawn in a quarter of an hour.
	\item A large ritual circle is at least two dozen paces across.
		A ballroom or village hall is probably the only place one could be drawn indoors, so most are drawn outside.
		At least a couple of hours are required to draw such a circle without haste.
\end{itemize}

\subsection{Megalithic Circles}

Some rites require a circle of standing stones, called a megalithic circle.
Such a circle must be at least the size of a large ritual circle, with at least a dozen stones each taller than a man.
The arrangement and shape of the stones is unimportant, as long as it is recognisable as a ring of standing stones, and so the same circle can be used for all rites that require one.
Constructing a megalithic circle is no easy task, typically requiring weeks of work by much of a village, even if the site is quite close to a stone quarry.

\subsection{Taglocks}

A taglock is any part of a person's body, such as a piece of flesh, a strand of hair, a nail clipping, a drop of blood, or a gob of saliva.
It is often used to bind a spell to a particular target.
It can always be picked off a person---although taking a hair without being noticed might be difficult---but people often leave taglocks behind them, especially in places they frequent.
Finding a taglock in a place you suspect someone might have left on, such as their house or a bed they've slept in, typically uses Perception.

\subsection{Poppets}

A poppet is an abstract representation of a person, although not a particular person.
Voodoo dolls are a typical example.
A poppet can be crafted from cloth, wood, clay, wax or other suitable material.
It should be recognisable as a human, bearing four appropriately-arranged limbs, a head, and two eyes.
However, if it is to be used in sympathetic magic affecting a non-human creature, it should resemble whichever creature the magic is intended to affect.
A poppet should be at least a handspan tall, though can be much larger.

\subsection{Effigy}

An effigy is much like a poppet and follows all the rules for those, except that it represents a particular person and must be crafted in their likeness.
Ideally, an effigy should be recognisable to even passing acquaintances of the person it is supposed to represent.
Less recognisable effigies will require an appropriate Test to be used for magic.



\section{Improvised Tools}



\section{Weapons}

Weapons are divided into several broad categories.
Players are free to describe their character's weapons how they wish, within the bounds of reason, placing them in one of the categories.
Anything a character might find at hand and hit people with can also be placed into a category.

A weapon's accuracy is added a flat bonus to rolls to hit, in place of an attribute.
A weapon's damage determines the number of dice rolled upon hitting.
The highest 3 dice are kept, as always, but the number of dice rolled are determined by the weapon instead of the wielder's skill.
The wielder's Might is added to the damage roll for melee or thrown weapons, but not for bows.

\begin{simpletable}{X[2.4]XXX[1.3]}
	\toprule
	Weapon & Accuracy & Damage & Range (metres)\\
	\midrule
	Fist & +2 & 3 & Melee\\
	Club & +4 & 4 & Melee\\
	Knife & +2 & 5 & Melee\\
	Hand Weapon & +4 & 5 & Melee\\
	Thrown Rock & +0 & 3 & $5\times\text{Might}$\\
	Thrown Weapon & +0 & 5 & $5\times\text{Might}$\\
	Bow & +2 & 5 & 100\\
	\bottomrule
\end{simpletable}

\subsubsection{Fist}
A punch, a kick, or a headbutt.
Covers any attack you make without any weapon at all.

\subsubsection{Club}
A club, a walking stick, a chair, or a cauldron.
A club is just about anything you pick up and hit someone with.

\subsubsection{Knife}
A knife or dagger.
Easily concealed, and a staple of blood witches.
The short blade costs the wielder reach, but can do as much damage as a sword if you get the enemy in the tender parts.

\subsubsection{Hand Weapon}
A sword, an axe, a mace, a spear, a pike.
This category covers most things actually designed as a weapon and larger than a knife.

\subsubsection{Thrown Rock}
A genuine rock, but also a teapot, a boot or a frog.
Anything you might pick up and throw.
This includes weapons that aren't designed to be thrown.

\subsubsection{Thrown Weapon}
A spear, a knife, a hatchet.
Any weapon you can throw that was actually designed for the purpose.
Rocks from slingshots fall in this category too.

\subsection{Bow}
A bow and arrow.
Also covers crossbows, if the setting includes them.


\part{Playing the Game}
\partlabel{rules}

\chapter{The Broad of It}
\chaplabel{general-rules}

This chapter covers rules essential to day-to-day play.
Players and GMs alike should be familiar with at least the major points in here in order to play.
More specific rules, pertaining to various disciplines of magic, can be found in the appropriate chapters of \partref{disciplines}.

It is important to remember that this book cannot cover every situation that may arise during play.
The role of the GM includes adjudicating such scenarios, and the following section should contain guidelines to assist in that.
Furthermore, it is often helpful to do the same when the players simply cannot remember a rule, to avoid slowing down play while someone looks it up.
And lastly, remember that all the rules contained in this book are guidelines and suggestions.
Feel free to change them all that you want!
The most important thing is that everyone is having fun.

\section{Tests}

Tests are the dice rolls used to determine the outcome of an action when there is element of chance and risk involved.
Several of the rules in this chapter and others will specify the appropriate Test to make with a particular action, but the GM should be calling for other kinds of Tests whenever appropriate as well.


\section{Combat}

\section{Injury}

\section{Movement}

\section{Magic}


\part{Disciplines of Witchcraft}
\partlabel{disciplines}

\chapter{Brewing}
\chaplabel{brewing}

\section{Creation and Application}

The \skillref{brewing} skill and \discref{brewing} discipline don't just cover potions brewed in a cauldron.
Potions, poultices, poisons, tinctures, salves, ointments, even beer, mead, wine and spirits.
Witches have many ways of turning \seclink{Herbs}{herbs}, and even other things, into more useful forms.

Each feat that allows a witch to prepare such a mixture lists the method of preparation and delivery.
The rules of such methods are presented below.

\subsection{Brewing and Chewing}

Different methods of preparation require different equipment, and take different periods of time.

\mixcreation{Cauldron}{cauldron}
Most potions are brewed in cauldrons, filled with water and brought to boil.
This requires, obviously, a cauldron, as well as a fire to heat it.
A smaller kettle might do in a pinch, but requires a Test.
A full cauldron will typically yield several doses.
Brewing in a cauldron requires around 15 minutes to bring the water to the boil, and another minute to mix the potion.

\mixcreation{Poultice}{poultice}
A poultice doesn't need to be brewed at all; the ingredients are simply chewed into a paste.
Some of the more dangerous poultices should definitely be ground with a mortar and pestle, however, rather than allowed anywhere near the mouth.
Creating a poultice requires less than a minute.

\mixcreation{Still}{still}
Some potions, or spirits, need to be distilled.
This requires quite a lot of dedicated equipment, a carefully maintained heat source, and several hours.

\subsection{Method of Delivery}

\mixdelivery{Drink}{drink}
A drink is about half a litre of liquid that must be drunk to take effect.
It can be quaffed as an Action, and takes effect immediately unless specified otherwise.

\mixdelivery{Spike}{spike}
A spike is a much smaller quantity of liquid than a potion, little enough that it could be slipped into a glass of wine without noticeably changing the volume.
It can be drunk willingly, but typically isn't.
It takes effect immediately, unless specified otherwise.

\mixdelivery{Topical}{topical}
A topical mixture is applied to the skin.
It typically requires more than an Action to smear it on, or bind a wad in place.
It generally only takes effect after a few minutes, but can kick in a little faster if applied to a wound or a mucous membrane.
Some need to be applied to the correct part of the body.
For example, if it is to treat a wound, it should be applied to the wound, and if it is to enhance the eyesight it should be applied to the eyes.

\mixdelivery{Injury}{injury}
These mixtures, typically harmful ones, must be delivered into the bloodstream via an injury.
The most expedient way to do this is to smear it on an arrow or an edged weapon, requiring an Action.
It's good for one cut, but otherwise remains on the weapon until rubbed off or washed away.
Beware rain.
It takes effect immediately, unless specified otherwise.

%TODO: Gaseous? Needs to be stored in an air-tight bottle?
%TODO: Incense? Needs to be burned, evaporated?

\section{Feats}

\feat{Numbing Painkiller}{painkiller-grace}{15}{
	None
}{
	\materials{\herbref[willow bark]{2}}
	
	The drinker may ignore 1 point of \secrefraw{damage} for a few hours, but loses one point of \attref{grace} for the same duration.
	Two doses may be effective simultaneously.
	Further doses cause paralysis, and possibly organ failure.
}

\feat{Dimming Painkiller}{painkiller-wit}{15}{
	None
}{
	\materials{\herbref[poppy seed]{2}}
	
	The drinker may ignore 1 point of \secrefraw{damage} for a few hours, but loses one point of \attref{wit} for the same duration.
	Two doses may be effective simultaneously.
	Further doses cause unconsciousness, and possibly cessation of breathing.
}

\feat{The Hard Stuff}{brewing-booze}{20}{
	None
}{
	You know how to make a drink that'll really put hairs on a man's chest.
	Or a woman's, at that.
	This potion is made in a still, instead of a cauldron.
	
	\materials{Alcohol, \herbref[apple]{2}}
	
	The drinker gains 1 \attref{might} for a few hours, and loses 2 \attref{wit} for the same duration.
	A second dose will render the drinker unconscious, and further doses are dangerously poisonous.
}

\feat{Healing Salves}{brewing-healing}{15}{
	\skillref[1]{brewing}
}{
	You know a wide range of minor poultices, salves and remedies for cuts, bruises and other physical injuries.
	As long as you have access to a reasonable supply of various \herbrefplural{2}, and time to chew up poultices, you may use your \skillref{brewing} skill in place of your \skillref{healing} skill on Tests to heal people and creatures of most physical injuries.
	Setting broken bones and performing surgery still requires \skillref{healing}.
	
	Similarly, you may use your \skillref{brewing} rank in place of your \skillref{healing} rank when determining the \secrefraw{damage} healed by a patient during a day of rest.
}


Elle Weerstrom looked up from her parsley patch as the air swooshed overhead.
Black fabric flapped.

``Evenin' Linda.
Didn't expect to see you today.
What brings you up 'ere?''

A navy-lined cloak fluttered as the younger witch pulled her broomstick short and dropped to the ground.
``It's young Barnie, Elle.
He's got a mob together, marching on Buckle Hollow.
Says Musgrave's been sleeping with his wife.''

Elle brushed her gloves together, knocking dirt onto the lawn.
``Well, has he?''

``No!
I mean, they might've kissed a bit but{\dots}
They've got torches, Elle!
Pitchforks and torches!
C'mon, grab your broom.
We've got to stop them.''

Elle looked down at the ground, then up at the sky.
She sighed.
``Alright, we'll go.
But we're walkin'; there's a storm brewing.''

Linda looked up.
A single wisp of cloud drifted lazily across the azure sky.
``Looks alright to me.''

``It's on its way, mark my words.
Wouldn't want to be flyin' home in it.''
Elle strode towards her cottage.
``I'm goin' to get my coat.''

\storybreak

Sure enough, the sky was grey when the mob got to Buckle Hollow.
A fine drizzle filled the air.
The farm gate stood open, a figure between the posts in its place.
Her parka was pulled up against the rain, pointed hat tall above her crown.
The mob stopped in its tracks as a crack of lightning cast her silhouette upon them.

``Fine weather for arson, innit?''
Her voice seemed to carry further than it should in the damp air, reaching the ears of all present.
They shuffled their feet in the thickening mud.
``Yer a disappointment, the lot o' yer.''
More feet shuffled.
A voice rose in dissent, but Elle continued over it.

``Now, I know Musgrave ain't the finest man you've all met.
An' I ain't quite sure what he's been up to that's got you all riled up.
But I \emph{am} sure that it ain't nothin' worse than half o' you've done to yer own wives!
Honestly, torches lads?''
The rain intensified and the torches guttered.
One spluttered out.
``What were you goin' to burn?
The barn?
His house?
\emph{Him}?
Put 'em away, men.''

There was another shuffling of feet, and a few torches wobbled noncommittally.
A sudden gust of wind drove the rain sideways for a moment.
Every torch went out with a pathetic cough.
``Get home to yer own wives, an' stop worryin' about other people's.''

With a quiet mumble, a general grumble and a mutter of ``Soddin' linen's gonna be soaked\dots'' the mob turned around and began to trudge the other way.

``An' Barnie!''
The mob stopped in its tracks again.
One man turned around, a few others craned their necks to see.
``She mightn't be kissin' other blokes if you spent as much time in yer own bed as in the gutter out back o' the Head.''
A muffled chuckle ran through the mob before another peal of thunder cut it short.
Collectively, they slank off through the mud.

\chapter{Willing}
\disclabel{willing}{Willer}{Willers}

\discref{willing} is the most raw and versatile application of a witch's magic.
Known to many layfolk as sorcery or spellcraft, it is the art of making something true simply by willing it hard enough.
Most \discref{willing} is performed without any of the accoutrements that accompany other forms of magic, and it doesn't follow the prescribed formulae of rites and brews.
This makes it the weakest form of magic in some ways, but its flexibility and ease of access more than make up for it.
So much so that most witches know at least the basics.

Like any witchcraft, \discref{willing} is something anyone can do if they know how.
But there is a knack to it.
It requires that the witch not only \emph{want} something to be the case, but \emph{believe} that it already is.
That she outright refuses to accept any possibility that it might not, in fact, be the case.
It involves willfully deceiving not only oneself, but also the very universe.
Most people would never even think to try it, but it is among the first things that most aspiring witches learn.

The line between \discref{willing} and \discref{headology} can be a little blurred, at times.
Both have the ability to make things true by making people believe them.
Many \practitioners{willing}	 say that the difference is that \discref{willing} affects the real world, while \discref{headology} only affects other people's minds.
The \practitioners{headology} point out that other people are just as much a part of the real world as any old rock is.
Some \practitioners{headology} say that the difference is that \discref{headology} is about convincing other people, while \discref{willing} is about convincing yourself.
The \practitioners{willing} point out that it's about more than convincing yourself, it's about convincing the world.
And that includes other people.
A few say that there's no real difference at all, that it's just two ways of thinking about the same thing.
These tend to be the witches who are obnoxiously good at both, and everyone else pointedly ignores them.

One interesting property of \discref{willing} is that it cannot affect living people, plants, or animals, although you can affect yourself.
It takes more than force of will to convince someone that they're a different shape; usually this entails talking to them.
This doesn't stop people getting knocked off their feet by a gust of wind, or crushed by a falling tree, however.
Witches interested in affecting people more directly are encouraged to pursue \discref{headology}, while plants and animals fall under the purview \discref{druidcraft}.

\section{Willing Tests}

\discref{willing} is not about memorising rites or recipes, nor about complexities and intricacies; \attref{ken} and \attref{wit} are unimportant to most applications of \discref{willing}.
Rather, \discref{willing} is about shunting your own stubbornness and conviction up against the fabric of reality until it gives; it depends upon raw force of \attref{will}.

Similarly, skill in \skillref{willing} represents very little in the way of knowledge, making it even more useless to a non-practitioner than most discipline skills.
Rather, the skill primarily represents a witch's ability to convince herself of things that are not yet true, in order that she may make them so.
An unskilled witch has difficulty with this, and it takes some effort to achieve even a broad, imprecise effect.
The best \practitioners{willing}, however, can effortlessly visualise even the faintest detail of their desired reality.

As such, the \skillref{willing} skill represents not only a witch's ability to rush or stretch her magic, but her accuracy with it.
She might use it to thread a needle, form an intricate shape, or change something subtly to avoid notice.
She may also use her \skillref{willing} skill in place of \skillref{weaponry} when attacking with an object she is controlling through \discref{willing}.

\section{Feats}

\feat{Basic Willing}{willing-basic}{10}{
	None
}{
	You can perform very basic acts of \discref{willing} upon things you can touch, given a bit of time to focus your mind and an obvious physical cue.
	Examples include:
	\begin{itemize}
		\item Lighting tinder or a candle without a spark, by cupping your hands around it and blowing on it.
		%\item Colouring or mildly flavouring a small pot of water by stirring it.
		\item Scratching writing into stone using just a fingernail.
		\item Rubbing stains out of clothing using your bare hands.
		\item Combing your hair with just your fingers.
	\end{itemize}
	The amount of time required to produce an effect depends on the desired outcome, but should be more than an {\action} without a Test.
	This ability cannot produce a lasting effect by itself.
	You can light a fire, because that sustains itself once ignited, but you cannot create, destroy or melt a pebble.
}

\feat{Kindling}{willing-fire}{15}{
	\featref{willing-basic}
}{
	You've practiced \discref{willing} a fire to life, and it's getting a lot easier for you.
	You can now ignite a fire within a dozen metres as an {\action}, with nothing more than a quick glare.
	You no longer require tinder, but still need something a fire can catch on fairly easily, such as twigs, cloth or dry leaves.
	Lighting a log or floorboards is still beyond you.
	
	The fire begins small, so will be extinguished by rain or a moderate wind before it can catch.
	A person whose clothes are ignited with this begins at \dice{1} {\fire}.
}

\feat{Fan the Flames}{willing-fire-2}{15}{
	\skillref[1]{willing},
	\featref{willing-fire}
}{
	You can use your will as a bellows, blowing a fire hotter and brighter.
	As an {\action}, you may double the size of an existing fire within a dozen metres of you.
	However, this is less effective on large fires: you can ignite more than about a cubic metre of a material in one {\action}.
	Using this against a person who is on {\fire} increases their {\fire} by 1 die.
	
	Additionally, through continuous concentration, you may double the heat and brightness of an existing fire (up to a cubic metre of it).
	This does double the rate at which it consumes fuel, however.
	A campfire affected in such a way is hot enough to forge iron with.
}

\feat{Firestarter}{willing-fire-3}{20}{
	\skillref[2]{willing},
	\featref{willing-fire-2}
}{
	By fanning the flames of a fire as you light it, you can burn bigger things.
	When you use \featref{willing-fire}, you can ignite the fire immediately on an object such as a log, or floorboards.
	The flame begins larger, enough to withstand drizzling rain or a moderate wind.
	A person whose clothes are ignited with this begins at \dice{2} {\fire}.
}

\feat{Extinguish}{willing-extinguish}{15}{
	\skillref[1]{willing},
	\featref{willing-fire}
}{
	Your experience working with fire allows you to extinguish them as easily as you light them.
	As an {\action}, you can extinguish up to a cubic metre of burning material within a dozen metres.
	The embers are still left hot to the touch, but not particularly dangerous.
}

\feat{Flamewalker}{willing-extinguish-2}{25}{
	\skillref[2]{willing},
	\featref{willing-extinguish}
}{
	You can extinguish fire near you---very near you---with only a modicum of concentration.
	As long as you are conscious, you are immune to the detrimental effects of heat and fire.
	This effect extends to your clothes, and most stuff you're carrying as long as it's not too large and doesn't extend too far from you.
}

\feat{Heatsink}{willing-extinguish-3}{15}{
	\skillref[3]{willing},
	\featref{willing-extinguish-2}
}{
	You can suck the heat from the air far and wide around you.
	As an {\action}, you can extinguish all fire within a dozen metres of you.
}

\feat{A Tool for the Job}{willing-tools-improvise}{20}{
	\featref{willing-basic}
}{
	Sometimes, the easiest way to convince someone of something is the hit them with a big stick until they agree with you.
	The world itself is no different.
	You've learned to make \discref{willing} easier using physical tools, even if they aren't the \emph{right} tools.
	
	Most simply, this means axes and knives cut just as well as ever in your hands, even if they've lost their edge.
	But you can take it even further, cutting carrots or trees with nothing more than an appropriately shaped stick.
	You can make any similarly-shaped object behave as the appropriate tool for a job.
	For a worse approximation, this may require a Test.
	A solid branch with a flat, sort of axe head shaped bit on the end will do a fine job of cutting down a tree.
	A solid branch without such an attachment would require a Test.
	A limp reed is going to be a real stretch.
	
	Such tools still obey the usual rules of \discref{willing}, and are of no additional use as weapons against people and animals.
	See \featref{headology-weapons-improvise} if you want weapons too.
}

\feat{A Hefty Tool}{willing-tools-effective}{15}{
	\featref{willing-basic}
}{
	You can make an appropriate tool more effective when you use it.
	Or an inappropriate tool, with \featref{willing-tools-improvise}.
	
	Tools are several times more effective when you use them.
	For example, when you use a spade it always lifts clumps of dirt several times the size of the spade's head.
	You can bring down a tree that you can barely wrap your arms around with only 4 or 5 swings of an axe.
	You can bail out a rowboat with only a few scoops of a bucket.
	
	This only works as long as you are still using the tool.
	For instance, you cannot store any more water than normal in a bucket unless you are carrying it.
}

\feat{Bubbling Brook}{willing-water}{10}{
	\featref{willing-basic}
}{
	Water is considered by many to be an element of change.
	You've certainly figured out how to change it.
	While touching water, you can move it around with your mind.
	You can make it flow, swirl, form into fairly elaborate shapes, or even float into the air.
	
	You can only affect the water while it remains one contiguous mass, which you must be touching.
	Afterwards, it flows normally again.
	You can only affect a couple of buckets-full at a time, and can't stretch it out over more than a couple of metres.
	You also can't move the water fast enough to hurt anybody.
	You can move other liquids if they are primarily water, such as wine, blood, or most potions.
	As always with \discref{willing}, you cannot affect liquids inside a living person, plant, or animal.
	
	Idly carrying around a bubble of water requires little concentration, but shaping it or doing much besides just carrying it is an {\action}.
}

\feat{Water Walk}{water-walk}{20}{
	\skillref[1]{willing},
	\featref{willing-water}
}{
	You can walk on water, or any other liquid you could affect with \featref{willing-water}.
	This takes great concentration, and you cannot take an {\action} and move on the water's surface in the same {\turn}.
	You may take an {\action} if you stand still on the water, however.
	
	If the water is flowing, you will be carried with it.
	Staying upright on fast flowing or turbulent water may require a Test, and the effect requires you to stay on your feet; falling prone will cause you to fall into the water.
	You may take use an entire {\turn} to clamber onto the water, if you are swimming at the surface.
}

\feat{River Run}{water-walk-2}{15}{
	\skillref[2]{willing},
	\featref{water-walk}
}{
	Walking on water has become second nature to you.
	You may take {\actions} while moving.
	Additionally, flow and turbulence pose you no threat.
	You may treat water you are standing on as though it were not flowing, and you can remain on the water's surface even when prone.
	Lastly, climbing upright onto the water while swimming at the surface is treated as though you are merely standing from being prone.
}

\feat{Condensation}{willing-water-vapour}{10}{
	\featref{willing-water}
}{
	The air is filled with water, and the skilled may draw it out to form liquid.
	You can draw it out within a couple of metres, into a container, spilling it on the ground, or holding it using \featref{willing-water}.
	
	Under normal conditions, you can produce about a litre a minute this way.
	However, this will be faster in a swamp or slower in dry air.
	In some situations, such as a desert or a burning building, the GM may call for a Test to gather enough water to be useful at all.
	
	You may also evaporate liquid water into the air in the same way, at the same rate.
	As such, you can perform \discref{brewing} that would normally require a \mixcreationref{still} by hand, although it still takes a couple of hours.
}

\feat{Illuminuous}{willing-light}{10}{
	\featref{willing-basic}
}{
	Your bright or gloomy moods are more literal than most, as you Will your surroundings brighter or darker.
	This effects a region centered on you and extends up to a few metres, enough to fill a room in a cottage.
	You can change the light level by about the amount that would be emitted by a few candles, just enough to read by.
	Light you cast this way has no apparent source; it simply suffuses the area.
	Maintaining this effect requires a minimum of concentration, and does not impede your other activities.
}

\feat{Rope Dance}{willing-rope}{15}{
	\featref{willing-basic}
}{
	Anyone can move a whip with their hand, but you can move one with your mind.
	That said, you still need to hold it{\dots}
	
	While holding a whip, rope, string, or thread up to 2 metres long, you can move it with your mind.
	You can barely lift any weight other than the rope itself; even a small knife tied to the end is a struggle.
	You can still yank on the rope or reel something in with your hands, of course.
	You do have very fine control, however, comparable to your manual dexterity.
	
	%TODO: Weapon damage? Add whips to the main table?
}

\feat{Know the Ropes}{willing-rope-2}{15}{
	\skillref[1]{willing},
	\featref{willing-rope}
}{
	As the finest rope-whipper in the West (or near enough), and you can thread a needle at twenty paces.
	
	When using \featref{willing-rope}, you can affect up to 20 metres of rope at a time.
	Additionally, you may divide this length between as many ropes as you can hold.
	This doesn't, however, grant you any extra ability to multitask, so this is about as practical as trying to use two whips at once.
}

\feat{Gust}{willing-wind}{20}{
	\featref{willing-basic}
}{
	Your mind can stir the air around you.
	You can create gusts of air within a dozen metres.
	Very minor and imprecise effects, like blowing hair or a cloak, don't require much effort, and can be done without an {\action}.
	If you concentrate as an {\action}, you can produce enough wind to send dishes flying, or to set a rock slowly rolling.
	With a Test, you might even produce enough of a gust to knock a person down.
	The gust must be fairly localised; you can't shift enough air to blow at anything larger than a person.
}

\feat{Breath}{willing-breathing}{20}{
	\skillref[1]{willing},
	\featref{willing-wind}
}{
	You hold the wind within you.
	You are immune to suffocation and drowning.
}

\feat{Updraft}{willing-wind-self}{15}{
	\skillref[1]{willing},
	\featref{willing-wind}
}{
	By surrounding yourself in a localised updraft, you can leap higher and slow yourself as you fall.
	You may leap 3 times as far or high.
	As long as you are conscious, you may fall up to 5 metres safely, and subtract 5 metres from the distance fallen when suffering {\damage} as a result.
	%TODO: Check this after writing the falling rules.
}

\feat{Cushion of Air}{willing-wind-self-2}{20}{
	\skillref[2]{willing},
	\featref{willing-wind-self}
}{
	You can cushion your fall with the wind.
	As long as you are conscious, you do not suffer any {\damage} from falling.
}

\feat{Team Lift}{willing-wind-others}{15}{
	\skillref[2]{willing},
	\featref{willing-wind-self},
	\featref{willing-wind-aoe}
}{
	You may extend your own updraft to surround those around you.
	All creatures of your choice within a dozen metres may benefit from \featref{willing-wind-self}, and from \featref{willing-wind-self-2}, if you have it.
}

\feat{Breeze}{willing-wind-aoe}{15}{
	\skillref[1]{willing},
	\featref{willing-wind}
}{
	\featref{willing-wind} produces only small gusts, to blow at one person or object.
	You can now affect a larger area, altering the direction and strength of the wind everywhere within a dozen metres of you.
	You can't create more than a moderate wind with this effect; enough to pick up light objects and roll them away, and enough to be uncomfortable, but not enough to knock people down.
	You can also counteract a wind of up to the same strength, creating a region of dead calm.
	
	This effect acts in very broad strokes.
	It always affects a roughly spherical region around you---although you may reduce the radius---and affects the whole area in the same way.
	The effect requires minimal effort to maintain; you may begin or alter it on your {\turn} without requiring an {\action}.
}

\feat{Long-Winded}{willing-wind-range}{20}{
	\skillref[1]{willing},
	\featref{willing-wind}
}{
	The air stretches everywhere, always.
	The wind does not stop short after a few metres!
	There is no reason your wind should be so limited, either!
	
	You may ignore the range limitation for \featref{willing-wind} and \featref{willing-wind-others}.
	You may use these feats, if you have them, on anything you can see.
}

\feat{Wind}{willing-wind-aoe-2}{15}{
	\skillref[2]{willing},
	\featref{willing-wind-aoe},
	\featref{willing-wind-range}
}{
	You can affect the speed and direction of the wind over a large region.
	This follows the same rules as \featref{willing-wind-aoe}, however the effect may extend for many kilometres around you.
	
	Shifting such large volumes of air can take quite some time.
	A major change, such as reversing the direction, or changing a strong breeze to a dead calm, may require several {\rounds} to take effect.
}

\feat{Cloud}{willing-weather}{10}{
	\skillref[1]{willing},
	\featref{willing-wind-range},
	\featref{willing-water-vapour}
}{
	You can affect the weather, in small ways.
	You cannot change general weather patterns over an area, but you can create or disperse the odd cloud, and cause it to hold or drop its rain.
	These changes can be performed in the background, without requiring {\actions}.
	
	Your control is fine enough to pass a cloud in front of the sun at a certain moment, or conversely to break a hole in the cloud for the sun to shine through.
	You can create a dry patch in light rain, or a patch of light rain on an overcast day.
	Even in torrential rain, you can at least ease it slightly for a small group of people.
}

\feat{Weather}{willing-weather-2}{20}{
	\skillref[2]{willing},
	\featref{willing-weather},
	\featref{willing-wind-aoe-2}
}{
	You can alter the weather, even causing meteorological changes over whole regions.
	You can turn a clear day overcast, or even bring rain.
	You can banish the rain and cloud to bring the bright sun.
	You can turn a chilly day warm, or a balmy day parky.
	
	The weather must be seasonally appropriate; you can't make it snow outside of winter or baking hot outside of summer.
	You cannot create winds stronger than those achievable using \featref{willing-wind-aoe-2}.
	Thunderstorms prove particularly difficult to create; you might manage it if conditions are already relatively close, but you cannot create one from a clear or even merely overcast day.
	
	Changes also take a while to take effect.
	If there isn't a cloud in the sky, turning the day overcast might take fifteen minutes, and the rain mightn't start for half an hour.
	These changes can be performed in the background, without requiring {\actions}.
}

\feat{Storm}{willing-weather-3}{20}{
	\skillref[3]{willing},
	\featref{willing-weather-2}
}{
	You have become even more proficient at influencing the wind and weather.
	You can produce a thunderstorm from even the clearest of days inside half an hour.
	You can also create wind strong enough to blow over less robust trees, heavily impede walking and running, and possibly damage poorly-built buildings.
	However, you create this wind as a windstorm, and while you control its general direction you have very little influence over the finer details.
}

\feat{Lightning}{willing-lightning}{10}{
	\skillref[2]{willing},
	\featref{willing-weather}
}{
	When a thunderstorm brews, when the pressure aches in your bones, when the electricity can almost be heard humming in the air, then it takes the barest spark to send lightning racing from cloud to ground.
	
	As an {\action} in a thunderstorm, you can cause a lightning strike somewhere within several kilometres.
	You have only the barest control over the location of the lightning: a margin of error of a kilometre or two.
	Essentially, you have just enough control to determine whether the flash illuminates your face or casts you in silhouette.
	
	There is a limited potential within a thunderstorm's clouds; creating lightning flashes more closely spaced than a minute is difficult.
}

\feat{Smite}{willing-lightning-2}{10}{
	\skillref[3]{willing},
	\featref{willing-lightning}
}{
	When you use \featref{willing-lightning}, you may aim your lightning with perfect precision as long as you are aiming at the highest point for a kilometre in any direction.
}

\feat{Amplify}{willing-voice}{15}{
	\featref{willing-wind}
}{
	You can Will your voice to travel louder and further through the air around you.
	When you speak, sing, whistle or so on, you may make it up to twice as loud.
	However, you may also make it carry far further---up to several kilometres---even if you choose not to make it louder.
}

\feat{Ventriloquism}{willing-voice-2}{20}{
	\skillref[1]{willing},
	\featref{willing-voice}
}{
	You have developed enough control over the air that you can speak directly with your mind, without involving your mouth.
	Consequently, you need not speak from yourself, but can speak from anywhere you can see.
	In fact, you can even speak from several places simultaneously, and may whisper directly into several people's ears without the risk of anybody else hearing.
}

\feat{Silence}{willing-silence}{20}{
	\skillref[1]{willing},
	\featref{willing-wind}
}{
	Much as you can move the air around you, you can also still it, creating silence.
	You may prevent any sound escaping from yourself, silencing your voice, your breathing, you footsteps and so on.
	This only prevents sounds originating from right near you; you can silence a twig snapping under your boot, but not a vase you've knocked onto the floor.
}

\feat{Dead Air}{willing-silence-2}{15}{
	\skillref[1]{willing},
	\featref{willing-silence},
	\featref{willing-wind-aoe}
}{
	You may extend your bubble of silence to cover the area you may affect with \featref{willing-wind-aoe}.
	No sound can originate within, enter, or pass through this region.
}

\feat{Guided Missile}{willing-wind-weapons}{15}{
	\skillref[1]{willing},
	\featref{willing-wind}
}{
	You can use the air to guide you projectiles in flight.
	You may use your \skillref{willing} skill in place of your \skillref{weaponry} skill when attacking with a \weaponref{thrown-rock}, \weaponref{thrown-weapon}, or \weaponref{bow}.
}

\feat{Dig}{willing-earth}{10}{
	\featref{willing-basic}
}{
	You can dig far faster with your mind than with your hands.
	You can move earth, but not rock, while you are touching it.
	You can affect up to a cubic metre at a time, in one connected chunk that reaches no more than 1 metre from you.
	If you are digging, you can excavate one cubic metre per {\action}---two {\actions} to dig a grave, for reference.
	
	Earth is an element of force, not precision.
	You can grab chunks and move them, but even forming crude shapes is difficult.
	You also lack the ability to shore up a roof, and so cannot create tunnels with this ability.
	
	You can move chunks of earth with enough force to use them as \weaponrefplural{club}, and may even draw them from the ground in the same {\action} as you \actionref{attack}.
	However, because you can only affect the earth while you touch it, you cannot throw it very far; no more than a few metres.
}

\feat{Burrow}{willing-earth-tunnel}{15}{
	\featref{willing-earth}
}{
	When using \featref{willing-earth}, you may shore up a roof and hence create a tunnel.
	The tunnel is only stable with a diameter of up to 1 metre, and hence acts as {\difficultterrain} for a human.
	You can extend the tunnel by 1 metre as an {\action}, while touching the face.
	You may compact the excavated earth into the walls of the tunnel, leaving minimal spoil.
	
	You may burrow only though earth---tunneling through rock requires \featref{willing-destruction-earth}.
}

\feat{Rockmover}{willing-rock}{10}{
	\skillref[1]{willing},
	\featref{willing-earth}
}{
	Rock is harder than mere dirt, and more difficult to affect.
	But in a way, it is simpler, being formed from one big piece instead of many small ones.
	
	You may use \featref{willing-earth} upon rocks, allowing you to lift boulders up to a cubic metre in size (2--3 tonnes) almost effortlessly.
	If the boulder is any bigger, you'll have to break it first.
	As such, this feat is insufficient to carry \materialrefplural{standing-stone} alone, but may help in their transportation.
	
	A sufficiently large rock counts as a \weaponref{hand-weapon}.
}

\feat{Erosion}{willing-destruction}{15}{
	\featref{willing-basic}
}{
	Time, wind, and water destroy everything eventually.
	Even the mountains themselves are ephemeral in the face of eternity.
	You can accelerate this process, however, delivering years of decay in an instant.
	Wood rots and crumbles, iron rusts and flakes away, even rock wears to dust.
	
	You must be touching whatever you are affecting, which must be one connected mass.
	You can affect about a cubic metre at a time, stretching no more than 1 metre from you.
	You can deliver about a decade of decay per minute.
	Flesh vanishes in one {\action}, while bone or lumber takes a minute.
	An iron bar should be rusted through in about ten minutes, although a thicker chunk may take longer.
	Stone takes many hours, often more.
	
	As usual, you cannot affect living people, plants or animals with this.
}

\feat{Shatter}{willing-destruction-earth}{15}{
	\skillref[2]{willing},
	\featref{willing-destruction},
	\featref{willing-rock}
}{
	The erosion of rock is a slow process, but shattering it can go very quickly, when you know how.
	As an {\action}, you can destroy up to a cubic metre of rock that you touch, in one connected mass that stretches no more than 1 metre from you.
	You reduce the rock to shards, or even to dirt.
	
	You can carve a chunk out of a rock face, but you have very little precision in the shape you destroy.
	Unless you also have \featref{willing-earth-tunnel}, the shattering has a tendency to collapse large overhangs or roofs, making this useless for tunneling.
	With \featref{willing-earth-tunnel}, however, this allows you to tunnel through solid rock, at the usual rate of 1 metre per {\action}.
}


\chapter{Headology}
\chaplabel{headology}

\section{Feats}

\feat{Mind over Magic}{foil-magic}{
	None
}{
	For all the magic circles and burning incense, magic ultimately comes from the mind.
	Not only do you know this, but you know \emph{how to exploit it}.
	
	If you can convince a practitioner of magic that their magic won't work, then it won't.
}


Today was not going well for Linda Greene.
It had started out alright.
A brisk walk in the frosty air at sunrise, a quick trip up to the castle to drop off a couple of poultices for the servants there.
The cook had even given her a big side of braised ham for her help.
But things had gone downhill pretty quickly when the warty old crone had strolled in and started turning people into frogs.

Now here she was, speeding over the mountaintops, hair and cloak whipped back by the frozen wind, and a crazy old hag hot on her tail.
The crone had a wicked-looking knife clutched between her teeth.
Been screaming that she was going to gut the king with it, or somthing unpleasant like that.
Well, the king was safe for now, even if he was croaking rather indignantly.
Linda had stuffed him down her blouse so his now-cold-blooded majesty wouldn't freeze in the mountain air.
It did explain the indignancy, perhaps, but Linda had bigger problems on her mind.
The hag was gaining on her.

Linda leant right forwards and threw the stick into a dive.
She picked up speed as she shed altitude, but the hag quickly followed suit.
Her feet brushed the snow as she skimmed down the far side of the mountain

\dots %TODO: Write the middle of the chase.

She wasn't the best flyer in the world, she knew that.
She wasn't even the best in the kingdom; young Wren up Salwich way could fly circles around her.
And this hag, too, was clearly better than her.
But the problem with being the best was that there were some things you didn't actually get to practice that much.
Some things that Linda, who was the first to admit that her reach often exceeded her grasp, got to practice all too often.

%TODO: Stall the sticks, and have Linda recover. The hag falls into a snowdrift.

\chapter{Broomcraft}
\chaplabel{broomcraft}

\section{Feats}

\feat{Broom Whisperer}{untrained-broom}{
	Flying 1
}{
	You've got the knack of flying for yourself now, and don't need a broom to be trained to fly it.
	You can even train a broom this way, although without one of its own to learn from the process takes about 24 hours of flight time.
}


\chapter{Sympathetic Magic}
\disclabel{sympathetic-magic}{Sympathist}{Sympathists}

\section{Sympathetic Links \& Symbols}
\seclabel{sympathetic-links}

Central to the practice of \discref{sympathetic-magic} is the creation and manipulation of {\symbols}.
A {\symbol} is a representation of a creature or object, and by affecting the {\symbol} a witch may cause a mirroring effect upon the target.
Not every \materialref{poppet} or \materialref{effigy} is automatically a {\symbol}.
It must by magically bound to the target by a {\symlink}.

A novice witch can only maintain one {\symlink} at a time.
It's not that maintaining one is particularly arduous; once established, a {\symlink} remains in place indefinitely, as long as the target is not resisting it.
Rather, two {\symlinks} tend to tangle themselves up, like pieces of string left together in a drawer.
Soon enough, both are totally useless and they have to be cut to separate them.

A {\symlink} by itself does nothing, but a \practitioner{sympathetic-magic} soon learns to use it to transmit numerous things: sensations, physical effects and more.
A {\symlink} doesn't always transmit everything it is capable of transmitting: only what the witch who established it wants it to.
The witch can change what the link transmits at any point she chooses, regardless of proximity to the {\symbol} or the target.
However, she has no particular sense of what is being transmitted by the link, and must watch the {\symbol} or the target if she wants to know.
As such, leaving {\symbols} lying around is a slightly dangerous proposition.

\subsection{Establishing a Sympathetic Link}

The simplest method for establishing a {\symlink} actually relies upon a trick of \discref{headology}.
The target must be \emph{expecting} the link, allowing the witch the opportunity to fasten it in place.
As such, at first, the witch can only establish {\symlinks} with people as the target, using a \materialref{poppet} or \materialref{effigy} as the {\symbol}.

Establishing the link requires an {\action}.
The target must see the {\symbol}, and the witch must declare to the target that she is binding them together.
Many witches adopt a standard incantation for this, often some piece of mumbo jumbo that suits the mystique they wish to cultivate.
The important thing is that the target understands the intent---that they are \emph{convinced} by it is not so important as in ``true'' \discref{headology}.

The target's expectation provides a hook that the witch may fasten the {\symlink} to.
If the target welcomes the {\symlink}, this is easy---it is established automatically and remains in place indefinitely.
Otherwise, establishing the link requires a \testtype{wit}{sympathetic-magic} Test {\opposed} by the target's \testtype{will}{sympathetic-magic}.

\subsection{Severing a Sympathetic Link}
\seclabel{break-sympathetic-link}

A witch can sever any {\symlink} she has established as an {\action}, or as part of establishing any new {\symlink}.
Additionally, a {\symlink} is severed if the {\symbol} or target are destroyed, or die.

Otherwise, a {\symlink} to an object, a willing creature, or a creature who is unaware they are the target of a {\symlink} at all, will persist indefinitely.
However, a {\symlink} to a creature that knows it is the target of a link, and does not wish to be, will be dislodged over time.
It automatically breaks after a minute, but can be broken sooner if it is {\stressed}.
This applies even if the creature previously accepted the link, but now wants rid of it.

Some uses of a {\symlink} will cause it considerable {\stress}, giving an unwilling creature another change to dislodge the link.
In this case, repeat the Test used to establish a link---your \testtype{wit}{sympathetic-magic} {\opposed} by the target's \testtype{will}{sympathetic-magic}.
If the target wins the Test, the {\symlink} is broken.
Actions that {\stress} a link will say so in their relevant feats.

\section{Feats}

\feat{Taglock Tracing}{identify-taglock}{10}{
	None
}{
	You can touch a \materialref{taglock} and detect who it originates from.
	If you have met the target, you can identify them infallibly.
	
	If you have never met the target, you must make a \testtype{heed}{sympathetic-magic} Test, with higher results giving more information about the target.
	You can only get general information about the target this way, such as height, build, sex, appearance, and occupation.
	You can't get any information about their location, or even whether they are still alive.
}

\feat{Stable Sympathy}{symlink-stable}{20}{
	\skillref[1]{sympathetic-magic}
}{
	An unwilling target will soon throw off a {\symlink}, but you've learned to stabilise your links against this, leaving them fastened strong in the face of adversity.
	However, this requires some preparation.
	
	By using an \materialref{effigy} in the likeness of the target as the {\symbol}, the {\symlink} does not expire over time, even when resisted.
	However, this does not allow it to resist {\stress}.
	This still requires the usual Test to establish the link in the first place.
}

\feat{Twin Links}{symlink-extra}{20}{
	\skillref[1]{sympathetic-magic}
}{
	You may maintain two {\symlinks} simultaneously.
}

\feat{Sympathetic Jerk}{sympathetic-puppet}{15}{
	None
}{
	An expert \practitioner{sympathetic-magic} can make their target dance on the puppet strings of their {\symlink}.
	You aren't there yet, but you've taken the first step.
	
	You cannot control your target's movements, but you can \emph{disrupt} them by jerking the {\symbolpossessive} limb the wrong way at the opportune time.
	If the target is just walking and talking normally, this doesn't do more than faintly disturb them.
	But if they are performing something highly physical or precise---running, jumping, aiming a weapon, or sewing, for example---it can severely disrupt them.
	You must know what the target is doing, or at least be able to take a very good guess, in order to jerk the correct limb at the correct time.
	Normally, this means you should be able to see them.
	
	Typically, you can use this by taking the \actionref{ready} {\action} in order to disrupt the target's next {\action}, while holding their {\symbol}.
	Common disruptions include making them miss an \actionref{attack}, or making them trip and fall prone when jumping or taking the \actionref{dash} {\action}.
	The GM ultimately decides the result of any disruption.
	Disruptions like those listed above do not require a Test, but if the outcome is in doubt, the GM may call for an {\opposedtest}.
	This typically uses \testtype{wit}{sympathetic-magic} for the witch, and might use something like \testtype{grace}{athletics} or \testtype{grace}{weaponry} for the target.
}

\feat{Sympathetic Destruction}{sympathetic-damage}{20}{
	None
}{
	When a {\symbol} is destroyed, you can send its death throes lashing along the {\symlink}, tearing at its target.
	Roll a {\damagetest} against the target, using \testtype{wit}{sympathetic-magic}.
	This works against objects, as well as creatures.
	The destruction of the {\symbol} obviously terminates the {\symlink}.
}

\feat{Sympathetic Buoyancy}{sympathetic-weight}{10}{
	None
}{
	The mass of a {\symbol} affects the mass of its target: a stone or iron \materialref{poppet} will make a person heavier while a wood or cloth one will make them lighter.
	Not hugely so---no more than about \SI{25}{\percent}---but enough to make a person easily float or sink, and to aid or hinder jumping and climbing.
	%TODO: Mechanical effects on jumping, etc.
	
	This effect can be used on objects as well as creatures, making them easier or harder to lift and carry.
}

\feat{Sympathetic Sleep}{sympathetic-sleep}{10}{
	None
}{
	A {\symbol} can rest in place of its target, allowing the target to work through most of the night.
	The rest, the {\symbol} needs to be tucked into a small bed, with soft bedding, a pillow, and sheets.
	It needs to be in a quiet, dim location, and generally to be in conditions where a person could easily sleep.
	The {\symbol} cannot be used for any other \discref{sympathetic-magic} while it is resting.
	
	As long as the {\symbol} rests for at least 12 hours each day, the target can get by on only 1 hour of sleep each day without any ill effects.
	However, the target does not recover from {\damage} and {\exhaustion} as a result of this rest.
}

\feat{Sympathetic Insomnia}{sympathic-sleep-deprive}{15}{
	\skillref[1]{sympathetic-magic},
	\featref{sympathetic-sleep},
	\featref{symlink-stable}
}{
	By keeping a {\symbol} awake, you can deprive its target of restful sleep.
	If the {\symbol} is subjected to loud noises, bright lights, stony bedding, or other significant discomforts while the target sleeps, the sleep will be fitful and restless.
	The sleep does not help them recover from {\damage} or {\exhaustion} (although they may still benefit from a day of rest).
	If this goes on for several nights, they may begin suffering {\exhaustion} due to sleep deprivation.
}

\feat{Sympathetic Warmth}{sympathetic-heat}{10}{
	None
}{
	The temperature of a {\symbol} affects the temperature of its target.
	Uncomfortable temperatures remain comfortable as long as the {\symbol} is at a comfortable temperature, and comfortable temperatures become uncomfortable if the {\symbol} is warmed or chilled.
	This effect cannot create dangerous temperatures---hot enough to cause heat stroke or cold enough to cause hypothermia---but can counteract them if the {\symbol} is inversely heated or cooled.
	Temperatures sufficiently extreme to cause {\damage}, such as fire or anything that would directly freeze the flesh, are outside the reach of this effect.
}

\feat{Sympathetic Combustion}{sympathetic-fire}{15}{
	\skillref[1]{sympathetic-magic},
	\featref{sympathetic-damage},
	\featref{sympathetic-heat}
}{
	When you burn someone in effigy, they really burn.
	If you destroy a {\symbol} with fire, and use \featref{sympathetic-damage}, the target also catches fire.
	A person ignited this way begins at \dice{3} {\fire}.
}

\feat{Sympathetic Malady}{sympathetic-attribute-reduce}{10}{
	None
}{
	You may afflict a target with various maladies by though a {\symlink}.
	You may reduce one of their attributes by 1 point by causing some appropriate affliction to the {\symbol}.
	For instance, you could reduce the target's \attref{grace} by binding their {\symbolpossessive} arms and legs, their \attref{heed} by blindfolding their {\symbol}, or their \attref{charm} by giving their {\symbol} some obvious disfigurement.
	A target may only be subject to one of these effects at a time, per witch who is affecting them.
}

\feat{Sympathetic Communication}{sympathetic-speak}{20}{
	\skillref[1]{sympathetic-magic}
}{
	You can send sounds along a {\symlink}, like a string telephone.
	A creature can hear sounds that originate near its {\symbol}, as long as it is conscious and not deafened.
	It can avoid this by plugging its ears, although this obviously leaves it deaf to its own surroundings as well.
	The {\symbol} has a very short range of hearing; speaking through it essentially requires picking it up and holding it near the mouth.
}

\feat{Sympathetic Pestering}{sympathetic-speak-2}{15}{
	\skillref[1]{sympathetic-magic},
	\featref{sympathetic-speak},
	\featref{sympathic-sleep-deprive}
}{
	When sending sounds along a {\symlink} using \featref{sympathetic-speak}, you may send them directly into the target's mind, bypassing its ears.
	The target hears them even if it is deaf, or has its ears plugged.
	You may even be able to wake the target up with loud enough sounds, if it is asleep.
}

\feat{Sympathetic Knot}{sympathetic-knot}{15}{
	\skillref[1]{sympathetic-magic},
	\featref{symlink-extra}
}{
	Normally when {\symlinks} get tangled, it renders both useless.
	However, if you knot them together intentionally, carefully, you can take advantage of it.
	
	You can knot together two or more {\symlinks} of the same kind---to creatures or to objects---as an {\action}.
	This requires that you are touching at least one end of each {\symlink} to be involved in the knot.
	For example, knotting together two {\symlinks} from \materialrefplural{poppet} to people requires you to be touching both \materialrefplural{poppet}, both people, or the \materialref{poppet} from one link and the person from the other.
	
	You can also undo a knot as an {\action}, but again you must be touching at least one end of every {\symlink} in the knot---you can only undo knots in their entirety, and not remove just one {\symlink}.
	Similarly, severing any {\symlink} in the knot severs all of them.
	You can only knot or unknot your own {\symlinks}.
	
	While two {\symlinks} are knotted, anything transmitted by any {\symbol} in the knot affects every target in the knot.
	You may still control what each {\symbol} transmits, but it always transmits to all targets.
}

\feat{Unbarred Sympathy}{sympathetic-ignore-barrier}{15}{
	\skillref[2]{sympathetic-magic}
}{
	Most barriers that interfere with magical effects don't break a {\symlink}, they just prevent it transmitting.
	But a finger on a string doesn't stop it from vibrating; it just restricts it.
	You can circumvent it if you know how.
	
	Barriers created by a \featref{circle-contain}, \featref{circle-exclude}, \featref{circle-contain-exclude}, or the like no longer impede transmission by your {\symlinks}.
	You still can't establish a {\symlink} that would be blocked by such a barrier, however.
}

\feat{Threading the Barrier}{sympathetic-ignore-barrier-2}{10}{
	\skillref[3]{sympathetic-magic},
	\featref{sympathetic-ignore-barrier}
}{
	If air can pass a magical barrier, why not a {\symlink}.
	It's just like threading a needle: it takes a bit of dexterity and your eyesight better be good, but it's hardly \emph{impossible}.
	
	You may establish a {\symlink} even through the barrier created by a \featref{circle-contain}, \featref{circle-exclude}, \featref{circle-contain-exclude}, or the like.
	%You can't always do it first time, however, and the GM may require a Test if you are in a hurry.
}


\discipline{Golemancy}{golemancy}{Golemancer}{Golemancers}

\section{Animating a Golem}

A golem must be animated as part of the creation of its body, and the witch doing the animation must be involved in its creation, even if she is not the primary craftswoman.
To animate a golem, a witch simply touches it and wills it life; many consider \discref{golemancy} to be a particularly specialised application of \discref{willing}.
Animating a golem always requires at least a minute, even if the golem's body can be crafted faster than that.

A novice \practitioner{golemancy}---one who can create a golem---has enough animating force to maintain one, and only one, golem.
If she animates a new golem, the previous golem immediately becomes inanimate.
A witch may also withdraw the animating force from a golem she has animated at any time, though if this is to be done urgently (perhaps the golem has gone rogue), the GM may require a Test.
Lastly, all a witch's golems become inanimate if she dies.

The crafting and animation, although strongly interlinked, are separate processes.
Tests related to the craftsmanship use \skillref{crafting} and an appropriate attribute.
Tests related to the golem's animation, such as giving it instructions, use \testtype{will}{golemancy}.
A witch can only animate a particular material into a golem if she has taken the appropriate feat.

\section{A Golem's Instructions}

A witch just beginning to dabble in \discref{golemancy} only has the skill to make very simple-minded, single-purpose creatures, although she will learn more nuance later.
These golems are imbued with a single instruction at the moment of their creation.
They will follow this until its completion, whereupon they will simply stand still and await destruction.
The instruction must be very simple, and the golem has minimal ability to improvise around it.
It should not have any conditional aspects, and the golem is unable to respond to any form of communication.
Example instruction are given below.

\begin{itemize}
	\item Deliver this note to the castle.
	\item Fetch my broom.
	\item Kill that man.
	\item Sweep the floor every evening.
	\item Extinguish any fires you see.
\end{itemize}

Additional information necessary to the completion of the task, such as the location of (and directions to) the castle, or the identity of an intended victim, may be imparted with the instruction.
The golem will trust this information and cannot adapt if it is wrong, for example if the victim has been disguised.
Furthermore, such information must be quite explicit.
For example, a golem instructed to ``attack intruders'' has no mechanism for distinguishing between invited guests and intruders.

Giving instructions with nuance, or instructions with multiple linked parts (such as ``go to the castle and kill the King'') requires a Test, with a {\tn} set by the GM based on the complexity of the instruction.
A failure either prevents the golem from animating or, at the GM's option, corrupts the instructions.

\section{A Golem's Statistics}

A golem's physical statistics are determined by the material and method of its construction, and are specified in the appropriate feat.
These include its \attref{might}, \attref{grace}, \statref{speed}, \statref{resilience} and \statref{shock-threshold}.
A golem whose \statref{shock-threshold} is met or exceeded by a {\damagetest} is immediately destroyed, instead of going into {\shock}.
The GM is also advised to apply common sense to other consequences of a golem's construction: for example, a clay or fabric golem will sink in water, a wood or wax golem will float, and a gingerbread golem will go soggy and fall apart.
All golems are immune to poisons and diseases, and unaffected by potions, poultices, and the like.

As for its other attributes, a golem lacks \attref{ken}, \attref{wit}, \attref{will}, \attref{charm} and \attref{presence} entirely; it automatically fails Tests that would require them.
It has 0 \attref{heed}.
However, a golem is unrelenting and untiring, and it has no mind to affect.
As such, it may be considered to automatically succeed at many Tests that would require \attref{will}.
Lastly, a golem has no ranks in any skills.

Initially, a witch can only animate small golems: about a handspan in height.
She doesn't have enough animating force to manage anything bigger, and anything smaller can't hold the magic required.
These golems do not have any effective attacks, and are too small to use weapons.

A golem knows no languages; it cannot read, write, or comprehend speech.
It cannot speak, and furthermore cannot vocalise in any fashion.
The sounds it can make are limited to such things as clapping its hands and stamping its feet.

A golem has senses as good as a human, although only if its craftsmanship gives it the appropriate anatomy.
For example, a gingerbread golem with two currants for eyes can see, but if baked without the currants it will be blind.
A clay golem can only smell if a nose is sculpted upon its face.

\section{Feats}

\feat{Gingerbread Golem}{gingerbread-golem}{15}{
	\noprereq
}{
	The simplest golems are not baked of clay, but of dough.
	When you bake a humanoid figure from gingerbread, you may animate it as a gingerbread golem.
	The entire preparation and baking process takes approximately half an hour, although an entire batch of golems can be crafted at once if the size of the oven allows.
	
	\materials{Flour, sugar, eggs, butter, \herb{ginger}{3}}
	
	A gingerbread golem has \negative{2} \attref{might}, 1 \attref{grace} and 15 \statref{speed}.
	It has an effective \statref{shock-threshold} of 1; it is destroyed if it takes any {\damage}.
	
	Additionally, a gingerbread golem has a limited lifespan.
	After about a week, it grows stale and can no longer move.
	Moisture or water, even a couple of minutes in rain, will destroy it sooner.
}

\feat{Fabric Golem}{fabric-golem}{15}{
	\noprereq
}{
	You can make your dolls get up and move.
	When you stitch, knit, or crochet fabric or yarn to form a humanoid figure, and fill it with stuffing, you may animate it as a fabric golem.
	The crafting process typically takes longer than an hour---much longer if you knit it.
	
	A fabric golem has \negative{2} \attref{might}, 1 \attref{grace} and 10 \statref{speed}.
	It has 1 \statref{resilience} and a \statref{shock-threshold} of 10.
	It can be repaired with a needle and thread, requiring several minutes.
	
	Unlike a gingerbread golem, a fabric golem isn't \emph{destroyed} by water.
	But a waterlogged fabric golem can't move until it dries out.
}


\feat{Wood Golem}{wood-golem}{15}{
	\featref{gingerbread-golem} or \featref{fabric-golem}
}{
	Wood offers a far more robust golem than gingerbread or cloth.
	When you whittle or assemble a humanoid figure from wood, you may animate it as a golem.
	The time required to do this depends on the size of the golem and the piece of wood you are crafting from.
	Whittling a small golem from an approximately man-shaped piece of wood may take as little as ten minutes, but carving one from a solid chunk of log might take an hour or more.
	Carving a much larger one could take days, and it would likely be faster to assemble it from multiple pieces of wood.
	
	A wooden golem has \negative{1} \attref{might}, 1 \attref{grace} and 10 \statref{speed}.
	It has 4 \statref{resilience} and a \statref{shock-threshold} of 14.
	Damage to a wooden golem cannot be repaired.
}

\feat{Wax Golem}{wax-golem}{15}{
	\featref{gingerbread-golem} or \featref{fabric-golem}
}{
	Wax isn't quite so robust as wood, but it can be very quick to mould and repair.
	When you mould or cast a humanoid figure from tallow or beeswax, you may animate it as a golem.
	The wax or tallow needs to be warmed to be moulded.
	Leaving it in the sun on a warm summer's day provides about the temperature required, as does sitting near a fire.
	Once warmed, the golem can be moulded by hand in a couple of minutes.
	
	A wax golem has \negative{1} \attref{might}, 1 \attref{grace} and 10 \statref{speed}.
	It has 2 \statref{resilience} and a \statref{shock-threshold} of 12.
	Damage to the golem can be easily repaired by the application of a little more warm wax.
	
	Wax golems are susceptible to heat.
	A hot summer's day won't hurt, just make them a little softer, but coming too close to a fire or forge will leave them in a dribbly pool on the ground.
}

\feat{Clay Golem}{clay-golem}{15}{
	\skillref[1]{golemancy},
	\featref{wood-golem} or \featref{wax-golem}
}{
	Wood, wax, tallow and gingerbread contain at least traces of life, making them easier to animate.
	Clay has never known life at all, but you've finally figure out how to teach it.
	When you mould a humanoid figure from clay and fire it in a kiln, you may animate it as a golem.
	The firing process takes at least ten hours.
	
	A clay golem has 0 \attref{might}, 1 \attref{grace} and 10 \statref{speed}.
	It has 14 \statref{resilience} and a \statref{shock-threshold} of 18.
	Damage to a clay golem can be repaired by filling the cracks with clay and refiring the golem.
	
	Clay golems are all but immune to heat.
	After all, they were fired to over \SI{1000}{\celsius} at their creation.
	Only rapid quenching from red-hot poses any threat at all.
}

\feat{Golem Programming}{golem-change-instructions}{15}{
	\featref{gingerbread-golem} or \featref{fabric-golem}
}{
	You can change the instruction imbued into one of your golems, allowing you to reuse the same golem for multiple tasks.
	Reprogramming a golem requires you to be touching it, and takes a minute of concentration.
	The normal restrictions apply to the new instruction, and the golem can only have one instruction at a time; adding a new instruction removes the previous one.
	You may only reprogram golems powered by your own animating force.
}

\feat{Delegated Programming}{golem-change-instructions-familiar}{10}{
	\featref{golem-change-instructions}
}{
	You've taught your familiar a few tricks of golemancy, and it may use the bond it shares with you to tap into your own animating force.
	Your familiar may reprogram golems, using the same rules as \featref{golem-change-instructions}.
	It reprograms your golems, however---it has no golems of its own.
	If imbuing the new instruction requires a Test, your familiar uses its own \testtype{will}{golemancy}, not yours.
}

\feat{Golem Reanimation}{golem-reanimate}{15}{
	\featref{golem-change-instructions}
}{
	Normally, a golem must be animated as its body is created.
	But, much as you've learned to change its instructions after its animation, you've figured out how to reanimate it after its creation.
	
	You can animate a golem's body at any time, even separately from its creation---although you must still have been involved in its creation.
	The body need not have been animated before, but can have been.
	Animating the golem takes about a minute, and requires you to touch it.
}

\feat{Golem of Another}{golem-reanimate-2}{10}{
	\skillref[1]{golemancy},
	\featref{golem-reanimate}
}{
	Animating a golem usually requires an intimate familiarity with its form, a familiarity that can only come from helping to craft its body.
	You've either learned to acquire that familiarity through inspection, or you can use brute will to animate it without that familiarity.
	
	You can animate a golem, using \featref{golem-reanimate}, even if you weren't involved in the creation of its body.
}

\feat{Advanced Instructions}{golem-advanced-instructions}{20}{
	\skillref[1]{golemancy},
	\featref{gingerbread-golem} or \featref{fabric-golem}
}{
	You can imbue your golems with more advanced instructions.
	The instructions can contain several steps, and conditional elements.
	On the whole, the golem can contain instructions that would take no more than a minute to convey by reasonably-paced speech.
	
	The golem still has next to no ability to improvise around the instructions, and will unreasoningly attempt to carry them out until it completes them or is destroyed.
	Information can still be imparted alongside the instructions, but it must still be explicitly.
	For example, an instruction to ``attack intruders'' will still fail, however the golem can now be instructed to ``attack anyone except me who enters this house'' or ``attack anyone who enters this house unless the door is unlocked with the key.''
	
	The golem can respond to some degree of communication, such as pointing and nodding, if explicitly instructed to.
	However, it still cannot \emph{comprehend} the communication.
	For example, it can follow an instruction to ``go where this man points,'' but not to ``follow this man's directions''.
}

\feat{Golem Language}{golem-understand-language}{20}{
	\skillref[1]{golemancy},
	\featref{golem-advanced-instructions},
	\featref{golem-change-instructions}
}{
	You imbue golems with your own understanding of language, allowing them to understand speech, as well as to read and write.
	This also includes some understanding of body language, though sarcasm, metaphor and the like continue to elude the golem.
	
	The golem still exclusively follows the instructions it has been imbued with, but you may now give it instructions such as ``follow my orders,'' and give further orders verbally.
	Verbal instructions must be just as explicit as imbued ones, however.
}

\feat{Golem Familiar Interpretation}{golem-understand-familiar}{10}{
	\skillref[1]{golemancy},
	\featref{golem-understand-language},
	\featref{golem-change-instructions-familiar}
}{
	Your golems gain the same ability to innately understand your familiar that you have, as effectively as though your familiar was speaking.
}

\feat{Golem Intelligence}{golem-intelligence}{20}{
	\skillref[1]{golemancy},
	\featref{golem-advanced-instructions},
	\featref{golem-change-instructions}
}{
	You may imbue your golems with some degree of intelligence.
	Although, to be honest, they're still a little dull.
	The golem gains \attref{ken}, \attref{wit}, \attref{charm} and \attref{presence} scores of 0, and may make Tests requiring these attributes.
	It can also perform a little improvisation around the best way to carry out its instructions.
	For example, if instructed to kill someone, it might pick up a club or sword instead of using its fists.
	It may finally be given instructions such as ``attack intruders,'' and will use its best judgement to determine whether someone is an intruder.
	However, the golem cannot disobey an instruction, even if doing so would be in your best interest.
}

\feat{Golem Speech}{golem-speak}{15}{
	\skillref[2]{golemancy},
	\featref{golem-understand-language},
	\featref{golem-intelligence}
}{
	If you create your golem's body with a mouth and a tongue, you may imbue it with the ability to speak.
	%And sing, though not necessarily well.
}

\feat{Twin Golems}{more-golems}{15}{
	\featref{gingerbread-golem} or \featref{fabric-golem}
}{
	You can muster enough animating force to maintain a second golem at the same time.
	If you try to animate a third golem, you may choose which existing golem becomes inanimate.
}

\feat{Golem Crew}{more-golems-2}{20}{
	\skillref[2]{golemancy},
	\featref{more-golems}
}{
	As you muster more animating force, your crew of golems grows.
	You can maintain three golems simultaneously.
}

\feat{Dwarf Golem}{medium-golem}{20}{
	\featref{wood-golem} or \featref{wax-golem}
}{
	You can muster enough animating force to create larger golems, about mid-thigh height.
	Such golems have their \attref{might} increased by 1, and their \attref{grace} decreased by 1.
	You can maintain only one golem of this size---or larger---regardless of how many golems you can maintain in total.
	
	Golems of this size can attack effectively.
	They use 2 dice for {\unarmed} {\damagetests}.
	They can even use weapons, if their instructions are sufficiently explicit about acquiring and using them.
	
	Obviously, crafting the body of a larger golem takes more material and normally more time.
	Acquiring an oven, kiln, or forge large enough can also present an obstacle for some kinds of golem.
}

\feat{Dwarf Crew}{more-medium-golems}{20}{
	\skillref[1]{golemancy},
	\featref{medium-golem},
	\featref{more-golems}
}{
	You're really getting the hang of animating a \featref{medium-golem} now.
	You are no longer limited to just one, and may maintain your extra golems at this size if you wish---the ones granted by \featref{more-golems}, and \featref{more-golems-2}, if you have it.
}

\feat{Life-Size Golem}{large-golem}{25}{
	\skillref[1]{golemancy},
	\featref{medium-golem}
}{
	It takes a lot of clay to make a golem the size of a man, but this pales in comparison to the force required to bring such a golem to life.
	You should know, because you can finally muster that much force.
	Such golems have their \attref{might} increased by 3, and their \attref{grace} decreased by 1.
	They can attack and use weapons, the same as a \featref{medium-golem}.
	
	You can only maintain one golem of this size, regardless of how many golems you can maintain in total.
	Furthermore, a golem of this size also counts as a \featref{medium-golem}, against the limit for golems of that size.
	So you cannot maintain both a \featref{large-golem} and a \featref{medium-golem} without the \featref{more-medium-golems} feat.
}

\feat{Twin Men}{more-large-golems}{15}{
	\skillref[2]{golemancy},
	\featref{large-golem},
	\featref{more-medium-golems}
}{
	It takes a lot of animating force to maintain a \featref{large-golem}, but you've finally mustered enough to maintain \emph{two}.
}

\feat{Golem Helpers}{more-small-golems}{15}{
	\skillref[1]{golemancy},
	\featref{more-golems}
}{
	You can maintain 2 hand-sized golems, in addition to however many golems you can otherwise maintain.
	These must always be hand-sized golems, however; they are unaffected by \featref{more-medium-golems}.
}

\feat{Helper Army}{more-small-golems-2}{15}{
	\skillref[2]{golemancy},
	\featref{more-small-golems}
}{
	You can maintain 5 hand-sized golems, in addition to however many golems you can otherwise maintain.
	These replace the additional 2 given by \featref{more-small-golems}.
}

\feat{Batch Baking}{more-gingerbread-golems}{20}{
	\skillref[2]{golemancy},
	\featref{gingerbread-golem},
	\featref{more-golems}
}{
	Nobody bakes just one gingerbread man, so why should you animate just one?
	You can maintain two gingerbread golems for the effort of one, at any size category.
	This even applies to extra golems granted by other feats---such as \featref{more-medium-golems}, \featref{more-large-golems}, or \featref{more-small-golems}.
}

\feat{Horse Golem}{golem-horse}{15}{
	\skillref[2]{golemancy},
	\featref{gingerbread-golem} or \featref{fabric-golem}
}{
	While a humanoid shape is certainly traditional for golems, other shapes are not impossible.
	Instead of making your golems humanoid, you may sculpt or otherwise form them in the shape of horses.
	A horse golem has the normal statistics for its size and material, except that its \statref{speed} is doubled.
	
	Horse golems can be made in every size that a \practitioner{golemancy} can make humanoid golems.
	A horse golem is of an appropriate nature to be ridden by a humanoid golem of the same size category.
	Therefore, the horse variety of a \featref{large-golem} can be ridden by a human, as long as it has at least 2 \attref{might}.
	
	A horse cannot understand language, and similarly a horse golem can never benefit from \featref{golem-understand-language} or \featref{golem-speak}.
}

\feat{Slender Golem}{golem-slender}{10}{
	\featref{gingerbread-golem} or \featref{fabric-golem}
}{
	You can animate more slender golems, with increased agility.
	You must make this decision when you craft the golem.
	A slender golem gains 1 \attref{grace}, but loses 1 \attref{might}.
	Furthermore, you can make it using slightly less material than a regular golem.
}

\feat{Bulky Golem}{golem-bulky}{10}{
	\featref{gingerbread-golem} or \featref{fabric-golem}
}{
	You can animate more bulky golems, with increased strength.
	You must make this decision when you craft the golem.
	A bulky golem gains 1 \attref{might}, but loses 1 \attref{grace}.
	However, it requires slightly more material than a regular golem.
}

\feat{Matryoshka Golems}{golem-nesting}{15}{
	\skillref[1]{golemancy},
	\featref{medium-golem},
	\featref{golem-reanimate}
}{
	You may nest golems inside one another.
	You can put a typical hand-sized golem inside a \featref{medium-golem}, or a \featref{medium-golem} inside a \featref{large-golem}.
	You may even nest all three levels.
	
	You must actually craft these nested golems, which obviously means the outer golems must be hollow.
	This is easy to do with clay or fabric, harder with wood, and very difficult with wax or gingerbread.
	The innermost golem does not need to be hollow, however, and does not even need to be the same material as the outer golem---it must be a material you can animate, however.
	However, it should be the same shape; you can only nest a \featref{golem-horse} inside another \featref{golem-horse}, for instance.
	
	Only the current outermost golem is animated---and only that one counts towards your animation limit.
	However, if it is destroyed, the next layer of golem down is automatically and immediately animated.
	It inherits the instructions and knowledge of the outer golem, and may immediately continue from where the outer golem left off.
}

\feat{Golem Self-Crafting}{golem-crafting-skill}{10}{
	\skillref[1]{golemancy},
	\skillrefspeciality[1]{crafting}{\anyspeciality},
	\featref{golem-intelligence}
}{
	When you animate a golem, you imbue it with your knowledge of the craft used to create it.
	It gains one \skillref{crafting} speciality appropriate to creating golems of its type.
	For example, you might give a gingerbread golem \skillrefspeciality{crafting}{Cook}, or a clay golem \skillrefspeciality{crafting}{Potter}.
	A wood golem might gain \skillrefspeciality{crafting}{Woodcarver} or \skillrefspeciality{crafting}{Carpenter}, depending on how it was constructed.
	
	The golem gains all the ranks that \emph{you} have in the relevant skill.
	You cannot use this feat if you do not have any ranks in a relevant skill, but the golem can gain up to 3 ranks, if you have that many.
	
	Some golems can ever repair themselves using this skill, if their instructions allow it and they have appropriate materials.
	Alternatively, you could set them to creating additional golem bodies, to animate with \featref{golem-reanimate-2}.
}

\feat{Remote Access}{golem-change-instructions-projection}{15}{
	\skillref[1]{golemancy},
	\featref{projection-golemancy},
	\featref{golem-change-instructions}
}{
	You can do more than just find your golems in the mental realm: you can alter them.
	You can reprogram a golem while you have an {\overtinterface} with it from the {\mentalrealm}.
	This still takes however long it would take you if you were physically touching the golem.
}


\end{document}
