\usepackage{amsfonts} %Gives the stuff we use to build \shortminus
\DeclareMathSymbol{\shortminus}{\mathbin}{AMSa}{"39}

\newcommand\negative[1]{$\shortminus$#1}
\newcommand\positive[1]{$+$#1}
\newcommand\stattimes[1]{$\times$#1}

\newcommand\nostat{---}

\newcommand\noskills{None}

\newcommand\anyspeciality{Any}

\newcommand\dice[2][0]{%Takes two arguments, the first of which is optional and defaults to zero.
	#2d%
	\ifnum #1=0%
		%
	\else%
		\ifnum #1>0%
			$+$%
		\else%
			$\shortminus$%
		\fi%
		\StrDel{#1}{-}%Strips the minus from it, essentially giving absolute value.
	\fi%
}

%%%%%%%
%Feats.
%%%%%%%

%Macros to set and get default discipline and tags for feats.
%The first argument to each of these either "discipline" or "tags"; the argument being dealt with.
\newcommand\definefeatdefault[1]{%Argument: argument to define.
	\providebool{isfeatdefaultvisible#1}%
}
\newcommand\setfeatdefault[3]{%Arguments: argument to set, value to set to, visibility (true or false).
	\csdef{featdefault#1}{#2}%
	\setbool{isfeatdefaultvisible#1}{#3}%
}
\newcommand\unsetfeatdefault[1]{%Argument: argument to set.
	\csdef{featdefault#1}{}%
}
\newcommand\evaluatefeatdefault[2]{%Arguments: argument to evaluate, value given to feat (possibly empty).
	\ifstrempty{#2}{%
		\ifcsempty{featdefault#1}{}{%
			\ifbool{isfeatdefaultvisible#1}{\csuse{featdefault#1}}{}%
		}%
	}{#2}%
}

\definefeatdefault{discipline}
\definefeatdefault{tags}

\newcommand\printfeattags[1]{%Argument: a list of tags, separated with commas and spaces after the commas.
	\capitalisewords{#1}%
}
\newcommandx\feat[7][3,5]{%Arguments: title, label, discipline, XP cost, tags, prerequisites, text.
	\edef\featdiscipline{\evaluatefeatdefault{discipline}{#3}}%
	\edef\feattags{\evaluatefeatdefault{tags}{#5}}%
	\subsection[#1]{#1 [#4 XP%
		\expandafter\ifstrempty\expandafter{\featdiscipline}{}{, \capital{\discref{\featdiscipline}}}%
		\expandafter\ifstrempty\expandafter{\feattags}{}{, \printfeattags{\feattags}}%
	]}%
	\labelname{#1}
	\label{feat:#2}
	\index{#1}
	\textbf{Prerequisites:} {#6}
	
	{#7}
} %TODO: Tags for feats such as 'first circle', instead of just talking about what level of the governing skill it requires?

\newcommand\featref[1]{\nameref{feat:#1}}
\newcommand\featlink[2]{\link{#1}{feat:#2}}
%TODO: Make feat references add the chapter in brackets afterwards, if the feat is in a different chapter.

\newcommand\noprereq{None}

\newcommand\anyfeatdiscipline[1]{any \discref{#1} feat}
\newcommand\Anyfeatdiscipline[1]{\emakefirstuc{\anyfeatdiscipline{#1}}}

%The option argument is the material's article, e.g. "a", "an".
\newcommand\anyfeatmaterial[2][]{any feat using #1 \materialref{#2}}
\newcommand\Anyfeatmaterial[2][]{Any feat using #1 \materialref{#2}} %I can't make \emakefirstuc work here.

\newcommand\prereqfamiliar[1]{\familiarref{#1} familiar}

%%%%%%%%
%Skills.
%%%%%%%%

\makeatletter
\newcommand\skill[3][\@nil]{%Takes three arguments, the first of which is optional and defaults to nothing.
	\def\govdisc{#1}
	%The first argument is the governing discipline, and is currently ignored.
	\subsubsection{#2}%
	\label{skill:#3}%
}
\makeatother

\makeatletter
\newcommand\skillrefinner[2][\@nil]{%Takes two arguments, the first of which is optional and defaults to nothing.
	\def\numrank{#1}%
	#2%
	\ifx\numrank\@nnil%
		%
	\else%
		~#1%
	\fi%
}
\newcommand\skillref[2][\@nil]{%
	\skillrefinner[#1]{\nameref{skill:#2}}%
}
\newcommand\skillrefspeciality[3][\@nil]{%
	\skillrefinner[#1]{\nameref{skill:#2} (#3)}%
}
\newcommand\skillreffamiliar[2][\@nil]{%
	\skillref[#1]{#2} (Familiar)%
}
\makeatother



%%%%%%%%%%%%
%Attributes.
%%%%%%%%%%%%

\newcommand\attribute[2]{%
	\subsection{#1}
	\label{attribute:#2}
}

\newcommand\attref[1]{%
	\nameref{attribute:#1}%
}



%%%%%%%
%Stats.
%%%%%%%

\newcommand\statlabel[2]{%For derived statistics. Arguments: title, label
	\labelname{#1}%
	\label{stat:#2}%
}
\newcommand\statref[1]{%
	\nameref{stat:#1}%
}



%%%%%%%
%Tests.
%%%%%%%

\newcommand\testtype[2]{%
	$\text{\skillref{#2}} + \text{\attref{#1}}$%
}
\newcommand\testtypespeciality[3]{%
	$\text{\skillrefspeciality{#2}{#3}} + \text{\attref{#1}}$%
}
