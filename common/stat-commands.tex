\usepackage{amsfonts} %Gives the stuff we use to build \shortminus
\DeclareMathSymbol{\shortminus}{\mathbin}{AMSa}{"39}

\newcommand\negative[1]{$\shortminus$#1}
\newcommand\positive[1]{$+$#1}
\newcommand\stattimes[1]{$\times$#1}

\newcommand\nostat{---}

\newcommand\noskills{None}

\newcommand\anyspeciality{Any}

\newcommand\dice[2][0]{%Takes two arguments, the first of which is optional and defaults to zero.
	#2d%
	\ifnum #1=0%
		%
	\else%
		\ifnum #1>0%
			$+$%
		\else%
			$\shortminus$%
		\fi%
		\StrDel{#1}{-}%Strips the minus from it, essentially giving absolute value.
	\fi%
}

%%%%%%%
%Feats.
%%%%%%%

\makeatletter
\newcommand\feat[6][\@nil]{%Arguments: discipline, title, label, XP cost, prerequisites, text.
	\def\featdiscipline{#1}%
	\ifx\featdiscipline\@nnil%
		\subsection[#2]{#2 [#4 XP]}%
	\else%
		\subsection[#2]{#2 [#4 XP, \capital{\discref{\featdiscipline}}]}%
	\fi%
	\labelname{#2}
	\label{feat:#3}
	\index{#2}
	\textbf{Prerequisites:} {#5}
	
	{#6}
} %TODO: Tags for feats such as 'first circle', instead of just talking about what level of the governing skill it requires?
\makeatother

\newcommand\featref[1]{\nameref{feat:#1}}
\newcommand\featlink[2]{\link{#1}{feat:#2}}
%TODO: Make feat references add the chapter in brackets afterwards, if the feat is in a different chapter.

\newcommand\noprereq{None}

\newcommand\anyfeatdiscipline[1]{any \discref{#1} feat}
\newcommand\Anyfeatdiscipline[1]{\emakefirstuc{\anyfeatdiscipline{#1}}}

%The option argument is the material's article, e.g. "a", "an".
\newcommand\anyfeatmaterial[2][]{any feat using #1 \materialref{#2}}
\newcommand\Anyfeatmaterial[2][]{Any feat using #1 \materialref{#2}} %I can't make \emakefirstuc work here.

\newcommand\prereqfamiliar[1]{\familiarref{#1} familiar}

%%%%%%%%
%Skills.
%%%%%%%%

\makeatletter
\newcommand\skill[3][\@nil]{%Takes three arguments, the first of which is optional and defaults to nothing.
	\def\govdisc{#1}
	%The first argument is the governing discipline, and is currently ignored.
	\subsubsection{#2}%
	\label{skill:#3}%
}
\makeatother

\makeatletter
\newcommand\skillrefinner[2][\@nil]{%Takes two arguments, the first of which is optional and defaults to nothing.
	\def\numrank{#1}%
	#2%
	\ifx\numrank\@nnil%
		%
	\else%
		~#1%
	\fi%
}
\newcommand\skillref[2][\@nil]{%
	\skillrefinner[#1]{\nameref{skill:#2}}%
}
\newcommand\skillrefspeciality[3][\@nil]{%
	\skillrefinner[#1]{\nameref{skill:#2} (#3)}%
}
\newcommand\skillreffamiliar[2][\@nil]{%
	\skillref[#1]{#2} (Familiar)%
}
\makeatother



%%%%%%%%%%%%
%Attributes.
%%%%%%%%%%%%

\newcommand\attribute[2]{%
	\subsection{#1}
	\label{attribute:#2}
}

\newcommand\attref[1]{%
	\nameref{attribute:#1}%
}



%%%%%%%
%Stats.
%%%%%%%

\newcommand\statlabel[2]{%For derived statistics. Arguments: title, label
	\labelname{#1}%
	\label{stat:#2}%
}
\newcommand\statref[1]{%
	\nameref{stat:#1}%
}



%%%%%%%
%Tests.
%%%%%%%

\newcommand\testtype[2]{%
	$\text{\skillref{#2}} + \text{\attref{#1}}$%
}
\newcommand\testtypespeciality[3]{%
	$\text{\skillrefspeciality{#2}{#3}} + \text{\attref{#1}}$%
}
