\newcommand\titleemph[1]{\emph{#1}} %For the titles of books and such, including Coven itself.



%%%%%%%%
%Tables.
%%%%%%%%

%\begin{tabularx} cannot be used inside \newenvironment (see section 5 of tabularx manual).
%We would use \tabularx instead.
%However, environments using \tabularx cannot be nested in further environment definitions.
%Without being able to nest things, this would likely lead to code duplication.
%So we will be using commands, not environments, for tables.
%Note that this means we can use the tabularx environment as normal.

\newcommand\simpletableinner[2]{%Arguments: alignment, table
	\rowcolors{2}{white}{gray!20}
	\begin{tabularx}{\linewidth}{#1}
		#2
	\end{tabularx}
}

\newcommand\simpletable[2]{%Arguments: alignment, table
	\begin{center}
	\simpletableinner{#1}{#2}
	\end{center}
}

\newcommand\notedtable[3]{%Arguments: alignment, table, notes
	\begin{center}
		\begin{threeparttable}
			\simpletableinner{#1}{#2}
			\begin{tablenotes}
				#3
			\end{tablenotes}
		\end{threeparttable}
	\end{center}
}

\newcolumntype{C}{>{\centering\arraybackslash}X} %Defines C as an X column, but centred, used in \atttable.



%%%%%%%%%
%Stories.
%%%%%%%%%

\newcommand\storybreak{\bigskip}

\newcommand\storybreakword[1]{%
	\begin{center}%
		#1%
	\end{center}%
}

\newcommand\storyimage[1]{%
	\begin{figure*}%
		\includegraphics[width=\textwidth]{imgs/#1}%
	\end{figure*}%
}

%Basing upon https://tex.stackexchange.com/a/199989
%Except using starred \chapter
%Starred chapter uses \@makeschapterhead instead of \@makechapterhead
\makeatletter
\newcommand\storychapter[1]{%
	\begingroup
	\let\@makeschapterhead\@gobble
	\chapter*{#1}
	\endgroup
}
\makeatother



%%%%%%%%%%%
%Drop caps.
%%%%%%%%%%%

\newcommand\dropcap[2][]{%
	%xstring macros cannot be edeffed, as they are not expandable.
	%See the xstring manual, section "Expansion of macros, optional argument"
	\StrLeft{#2}{1}[\firstletter]%
	\StrGobbleLeft{#2}{1}[\remainder]%
	\lettrine[ante=#1]{\emakefirstuc{\firstletter}}{\remainder}%
}
