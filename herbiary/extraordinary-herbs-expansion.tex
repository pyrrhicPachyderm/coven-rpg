\chapter{Extraordinary Herbs}
\chaplabel{extraordinary-herbs-expansion}

\creature{Dryad}{Dryads}{dryad}{
	\atttable{2}{4}{1}{1}{2}{2}{5}{2}
}{
	\speed{10}
}{
	\skillref[2]{animals}, \skillref[2]{deception}, \skillref[2]{persuasion}, \skillref[3]{socialising}, \skillref[1]{stealth}
}{
	\capital{\creaturerefplural{dryad}} are parasitic plants, drawing their energy from the oak trees they live within.
	Separate from their host tree, they are humanoid, feminine figures formed from a tangle of woody vines.
	However, they can step inside an oak tree, physically merging with it, and leaving next to no trace of their presence.
	
	\capital{\creaturerefplural{dryad}} are shy, but curious.
	They like to watch people, but don't like to be seen.
	Humans pose the biggest threat to the oak trees they call home, but most \creaturerefplural{dryad} find them strangely alluring, regardless.
	If they are given enough chance to observe a human, and judge them to be kind-hearted, they might approach them.
	They gain the person's trust over a number of meetings until, one day, the person is never seen again.
}{
	\ability{Endophyte}{
		The \creatureref{dryad} can move inside an oak tree.
		While it remains in the tree, a few vines curled around the tree's branches and roots are the only outwardly visible sign of its presence.
		It must remain inside an oak tree for at least 8 hours a day, or suffer {\exhaustion}.
	}
	
	\ability{Oak Stride}{
		The \creatureref{dryad} can move from one oak to another, without emerging from the tree.
		Doing so, the \creatureref{dryad} has an effective \statref{speed} of 40.
		This only works as long as the oak trees are in contact through their root or branches.
	}
}

\creature{Tolling Snowbell}{Tolling Snowbells}{tolling-snowbell}{
	\atttable{\nostat}{\nostat}{\negative 2}{2}{4}{4}{\nostat}{\nostat}
}{
	\speed{0}
}{
	\skillref[3]{botany}, \skillref[1]{healing}
}{
	In the deepest of winters, atop the highest of mountains, men and women can lose themselves in the frozen snows.
	Caught in a blizzard, tumbled from the path, or simply wandered astray, the ice claims many every year.
	But legend tells of a rare, lucky few who found salvation.
	
	As fingers fall to frostbite, delerium sets in.
	The dying person hears their own death knell, the toll of a bell echoing through the frozen landscape.
	Those who seek the source of the sound find a tiny plant, green stems and white flowers poking through the thick snows.
	Then, on the twelfth toll, they slip into unconsciousness.
	
	When they awake, sometimes many days later, they are warm once more.
	They may be in a cave, sheltered from the storm, or even beside the embers of a fire.
	They carry no injuries from their experience; even their frostbitten fingers are restored.
	The only traces are a fine white powder, scattered aroound them, and a faint urge to return to cold places, which may stay with them for many years.
}{
	\ability{Death Knell}{
		A creature dying of cold can hear a series of knells, as if from a church bell, originating from the \creatureref{tolling-snowbell}.
		These knells are loud, carrying for a kilometre or more, but are audible only to the dying.
		The twelfth knell heard by a particular person always coincides with their death.
	}
	
	\ability{Frozen Pollen}{
		When a creature dies of cold beside the \creatureref{tolling-snowbell}, the plant releases its pollen upon the corpse.
		The corpse is immediately restored to life, before its {\soul} has a chance to depart, and is healed of all injuries caused by the cold.
		While the creature remains coated in the pollen, it is immune to all ill effects of cold.
		However, it is under the \creaturerefpossessive{tolling-snowbell} control.
		When the pollen is removed, or thaws, it awakes, with no memory of the time since its death.
	}
}
