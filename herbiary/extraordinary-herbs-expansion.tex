\chapter{Extraordinary Herbs}
\chaplabel{extraordinary-herbs-expansion}

\creature{Dryad}{Dryads}{dryad}{
	\atttable{2}{4}{1}{1}{2}{2}{5}{2}
}{
	\speed{10}
}{
	\skillref[2]{animals}, \skillref[2]{deception}, \skillref[2]{persuasion}, \skillref[3]{socialising}, \skillref[1]{stealth}
}{
	\capital{\creaturerefplural{dryad}} are parasitic plants, drawing their energy from the oak trees they live within.
	Separate from their host tree, they are humanoid, feminine figures formed from a tangle of woody vines.
	However, they can step inside an oak tree, physically merging with it, and leaving next to no trace of their presence.
	
	\capital{\creaturerefplural{dryad}} are shy, but curious.
	They like to watch people, but don't like to be seen.
	Humans pose the biggest threat to the oak trees they call home, but most \creaturerefplural{dryad} find them strangely alluring, regardless.
	If they are given enough chance to observe a human, and judge them to be kind-hearted, they might approach them.
	They gain the person's trust over a number of meetings until, one day, the person is never seen again.
}{
	\ability{Endophyte}{
		The \creatureref{dryad} can move inside an oak tree.
		While it remains in the tree, a few vines curled around the tree's branches and roots are the only outwardly visible sign of its presence.
		It must remain inside an oak tree for at least 8 hours a day, or suffer {\exhaustion}.
	}
	
	\ability{Oak Stride}{
		The \creatureref{dryad} can move from one oak to another, without emerging from the tree.
		Doing so, the \creatureref{dryad} has an effective \statref{speed} of 40.
		This only works as long as the oak trees are in contact through their root or branches.
	}
}
