\chapter{Willing}
\chaplabel{generic-magic}

Willing is the most raw and versatile application of a witch's magic.
Known to many layfolk as sorcery or spellcraft, it is the art of making something true simply by willing it hard enough.
Most willing is performed without any of the accoutrements that accompany other forms of magic, and it doesn't follow the prescribed formulae of rites and brews.
This makes it the weakest form of magic in some ways, but its flexibility and ease of access more than make up for it.
So much so that every witch knows at least the basics.

Like any witchcraft, Willing is something anyone can do if they know how.
But there is a knack to it.
It requires that the witch not only \emph{want} something to be the case, but \emph{believe} that it already is.
That she outright refuses to accept any possibility that it might not, in fact, be the case.
It involves willfully deceiving not only oneself, but also the the very universe.
Most people would never even think to try it, but it is among the first things any aspiring witch must learn.

The line between Willing and Headology can be a little blurred, at times.
Both have the ability to make things true by making people believe them.
Many Willers say that the difference is that Willing affects the real world, while Headology only affects other people's minds.
The Headologists point out that other people are just as much a part of the real world as any old rock is.
Some Headologists say that the difference is that Headology's about convincing other people, while Willing's about convincing yourself.
The Willers point out that it's about more than convincing yourself, it's about convincing the world.
And that includes other people.
A few say that there's no real difference at all, that it's just two ways of thinking about the same thing.
These tend to be the witches who are obnoxiously good at both, and everyone else pointedly ignores them.

\section{Feats}

\feat{Basic Willing}{basic-willing}{
	None
}{
	You can perform very basic acts of Willing upon things you can touch, given a bit of time to focus your mind and an obvious physical cue.
	Examples include:
	\begin{itemize}
		\item Lighting kindling or a candle without a spark, by cupping your hands around it and blowing on it.
		%\item Colouring or mildly flavouring a small pot of water by stirring it.
		\item Scratching writing into stone using just a fingernail.
		\item Rubbing stains out of clothing using your bare hands.
		\item Combing your hair with just your fingers.
	\end{itemize}
	The amount of time required to produce an effect varies depending on the desired outcome, but should be more than an Action without a Test.
	This ability cannot produce a lasting effect by itself.
	You can light a fire, because that sustains itself once ignited, but you cannot create, destroy or melt a pebble.
}

\feat{Kindling}{basic-fire}{
	\featref{basic-willing}
}{
	You've practiced Willing a fire to life, and it's getting a lot easier for you.
	You can now ignite a fire within Close Range as an Action, with nothing more than a quick glare.
	You no longer require kindling, but still need something a fire can catch on fairly easily, such as twigs, cloth or dry leaves.
	Lighting a log or floorboards is still beyond you.
	
	The fire begins small, so will be extinguished by rain or a moderate wind before it can catch.
	A person walking about or wriggling will automatically foil an attempt to ignite their clothes (perhaps without noticing), but a person sitting fairly still may not.
}
