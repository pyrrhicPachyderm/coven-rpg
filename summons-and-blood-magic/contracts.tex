\chapter{Contracts}
\chaplabel{contracts}

\section{Contract Law}
\seclabel{contracts}

\capital{\contracts} are among the most complicated of all magics.
In fact, much of the skill in using {\contracts} lies in making them complicated enough to hide loopholes, or catches, to dupe the other party and come out on top.
The other side of the coin is understanding them well enough to avoid being caught out by the same tricks.

Some players, and some GMs, will revel in this complication, loving the battle of wits as they open and close loopholes in the {\contracts} they write.
Others, however, will think it too much like work, preferring not to bog down play with such details.
As such, there are two ways to use {\contracts} in your game.
The players and GM should agree on which method is being used before anyone takes feats from this chapter, to avoid confusion and disappointment.

The first way is to write them yourself---word for word.
You can even put your character's signature upon them---perhaps in red pen---and use the piece of paper as a prop around the game table.
The GM might offer you the chance to make a {\test} to spot a loophole, if they see one that you haven't, but ultimately, the words written upon the paper are the words that form the {\contract} in-game.

The second method is for you just describe the intent of the contract to the GM.
The GM can then call for a {\test}---possibly {\opposed} by the other party in the {\contract}---for your character to draft the {\contract}.
On a failure, the GM can invent a loophole to catch you out on, while on a success you write a clean {\contract}, or could even catch the other party out.

\capital{\tests} related to {\contracts} use the \skillrefspeciality{lore}{Contracts} skill.
Unlike most \skillref{lore} skills, which are typically paired with \attref{ken}, most \skillrefspeciality{lore}{Contracts} {\tests} will use \attref{wit}.
Writing and spotting loopholes and other tricks is less about rote learning, and more about outwitting the other parties to the {\contract}.

\subsection{Creating Contracts}

In-game, a {\contract} is a magical agreement brought into existence using a written document, signed in \materialref{blood}.
A {\contract} consists of three parts: {\stipulations}, {\penalties}, and {\signatories}.
All three must be present on the written document.
The words upon the document form the binding rules of the {\contract}---any spoken agreements do not matter, only what is written.

There are two steps to creating a {\contract}.
The first is writing the document, specifying the {\stipulations}, the {\penalties}, and who the {\signatories} are to be.
The second is getting each {\signatory} to {\sign} the document.
The document must not be modified during the second step.
If it is altered at any point after the first {\signatory} {\signs}, but before the last {\signatory} {\signs}, then it becomes {\void}.

Once the last {\signatory} signs it, the {\contract} takes effect.
Every {\signatory} is immediately, magically aware of this, regardless of whether they are currently present.
From this point onwards, the document used to create the {\contract} is no longer relevant.
It can be modified, or destroyed, without affecting the terms of the {\contract} itself.

Not just anyone can write a magically binding {\contract}, although anyone can {\sign} one.
The contract must be penned by a person with the \featref{contracts} feat in order to make it magically binding.
However, the person who pens the {\contract} does not need to be one of the {\signatories} on it.

Several feats allow extra {\penalties} or clauses to be specified upon a {\contract}.
The person writing the {\contract} must have all the relevant feats for any additional {\penalties} or clauses they use.
However, several people can collaborate to write a {\contract}, and in doing so can combine their sets of feats to determine the contents of the {\contract}.
To do so, they must all have the \featref{contracts} feat, and be able to pass the written document around, all taking turns writing upon it before anyone begins to {\sign} it.
People without the \featref{contracts} feat may be involved in the discussion, providing input or even advice, but cannot help to pen the {\contract} except to {\sign} it.

\subsection{Signatories}
\seclabel{contract-signatories}

A {\contract} comes into effect when it is {\signed} in \materialref{blood}, by everyone it is to affect directly.
A {\contract} must specify all its {\signatories} as part of the main text, before anyone {\signs} it.
These specified {\signatories} are the only people who may be directly affected by the {\penalties} of the {\contract}---they are the ones who have agreed to its terms.

A {\signature} does not need to be a name; it is often just an ``X'', or a paw-print.
The important aspects are that it is made by the {\signatory}, in the {\signatorypossessive} \materialref{blood}, with the intent to {\sign} the {\contract}.
Note that, although they must intend to {\sign} the {\contract}, they might be doing so reluctantly.
A {\contract} {\signed} under duress is still valid.
Furthermore, there is no requirement that the {\signatory} has even read the {\contract}.

\subsection{Stipulations}
\seclabel{contract-stipulations}

The first part of a {\contract}, apart from the list of {\signatories}, is the {\stipulations}.
These specify a set of things that must (or must not) happen, and which {\signatory} is responsible for ensuring they occur.
For example ``Agatha must ensure that Brynston Tower falls by midnight,'' or ``Agatha must not tell anyone about the Sword of Queen Grima.''

\capital{\stipulations} may specify a time in which they must happen, but they do not have to.
It is not always necessary---``Agatha must not tell anyone about the Sword of Queen Grima'' simply applies forever.
But for cases where it is necessary, its omission provides a common loophole.
For example, ``Agatha must ensure that Brynston Tower falls'' is a {\stipulation} that Agatha can never be {\penalised} for breaking, as there is always the possibility that it will fall at some later date.

If a {\stipulation} is violated---and any one of the {\signatories} becomes aware of the violation---then the {\penalties} take effect upon the {\signatory} who violated them.
If multiple {\signatories} violate their {\stipulations}, they are all {\penalised}.

Note that, unlike most \discref{headology}, it is not only belief that matters.
A {\signatory} must be aware of the violation, but the violation must also have actually occurred.
Also, note that the {\signatory} who learns about the violation may be the same {\signatory} who will be penalised for it---you can never get away with intentionally breaching {\contract}.
If there is some ambiguity about whether a {\stipulation} has been violated, then the GM is the final arbiter.
However, the GM may call for the {\signatories} to argue their cases, perhaps with associated {\tests}.

\subsection{Penalties}
\seclabel{contract-penalties}

Accompanying each {\stipulation} in a {\contract}, there should be a {\contractpenalty}.
One {\contractpenalty} may cover multiple {\stipulations}, and violation of one {\stipulation} may trigger multiple {\penalties}.
However, {\penalties} can only affect the {\signatories} who violated the associated {\stipulations}.

Several varieties of {\contractpenalty} are possible in a {\contract}, but only one is possible for a novice {\contract}-writer: the {\liarscurse}.
Others must be learned through feats.
Every {\signatory} upon a {\contract} becomes aware when a {\stipulation} is violated and a {\contractpenalty} takes effect.

If a {\contractpenalty} has a duration, such as the {\liarscurse}, then the {\contract} must specify how long it lasts.
This might be a set length of time, until a certain condition comes to pass, or even forever.
As with violations of a {\stipulation}, any condition that ends a {\contractpenalty} must be detected by one of the {\signatories} on the {\contract}, and all {\signatories} are made aware when a {\contractpenalty} ends.

\capital{\contracts} can also be written without {\stipulations}, to bring some {\penalties} into effect on some {\signatories} immediately as the {\contract} takes effect.
This may be particularly useful when forcing someone to {\sign} a {\contract} under duress.
It can also be used with {\penalties} that can have some beneficial effect, such a {\contract} that provides \featref{contracts-penalty-location} between all members of a coven.
A {\contract} can mix and match between {\penalties} that occur immediately and those that are imposed by breaking {\stipulations}.

\subsubsection{The Liar's Curse}
\seclabel{liars-curse}

The {\liarscurse} is the standard {\contractpenalty}, available to anyone who can write a {\contract}.
Anyone who breaks a {\contract} and has the {\liarscurse} imposed upon them is magically marked as untrustworthy, a liar and an oath breaker.
This acts as a magical influence on all who interact with the cursed person.
The cursed person's words ring false, and people have a difficult time believing what they say.
This makes social interaction in general quite difficult, and in particular makes it almost impossible to perform \discref{headology} or convince people to {\sign} further {\contracts} with the cursed person.
Interacting with people while under the {\liarscurse} can also be harmful to one's reputation, as people come to recognise you as a liar.

The curse does not make people disbelieve every statement the cursed person makes.
Self evident statements, such as ``the sky is blue'' are unaffected.
But many things can be affected.
For example, threats might come across as crass jokes, rather than being genuinely intimidating.
Even compliments can ring hollow, sounding snide or sarcastic.
Obvious attempts to convince people of something they may or may not believe are most severely affected.
The GM is encouraged to impose large penalties on any {\tests} that might be affected by the curse.
Being under the effect of more than one {\liarscurse} at a time has no additional effect, though {\contracts} can always be written to add the durations of multiple instances together.

The effect of the {\liarscurse} is not obviously supernatural to anyone interacting with the cursed person.
Even people familiar with the existence of the {\liarscurse} can struggle to recognise its effects.
Furthermore, recognising that someone is under the {\liarscurse} does not provide immunity to it---they still sound dishonest, and overcoming it requires deciding to trust them anyway.
Doing so can be dangerous, for someone who says they're under the {\liarscurse} might be lying, after all.

\subsection{Voiding a Contract}
\seclabel{contract-void}

Various situations can cause a {\contract} to become {\void}.
The most common one is when a {\contract} is modified while it is being {\signed}---after the first {\signatory} {\signs}, but before the last one does.

Otherwise, a {\contract} can specify conditions under which it renders itself {\void}.
These conditions must be detected be a {\signatory}, just like any violations of {\stipulations}.

If a {\contract} becomes {\void} before it takes effect---if it modified while it is being {\signed}, for instance---then it can never take effect.
Anybody who later tries to {\sign} the {\contract} is aware that it is {\void}.
An entirely new document must be created in order to bring a {\contract} into effect.

If a {\contract} becomes {\void} some time \emph{after} it takes effect, then all its effects immediately end.
No new {\penalties} can take effect from that {\contract}.
Furthermore, any ongoing {\penalties} immediately end.
Every {\signatory} on the {\contract} becomes aware when an active {\contract} is {\voided}.

\section{Feats}

\feat{Signed in Blood}{contracts}[headology]{20}{
	\noprereq
}{
	You may create {\contracts}.
	Your {\contracts} are limited to only two {\signatories}, unless you are collaborating with someone who can create {\contracts} with more.
}

\feat{Multisign}{contracts-more-signatories}[headology]{10}{
	\skillrefspeciality[1]{lore}{Contracts},
	\featref{contracts}
}{
	You may create {\contracts} with more than two {\signatories}---as many as you like.
}

\feat{Null and Void}{contracts-void}[headology]{10}{
	\skillrefspeciality[1]{lore}{Contracts},
	\featref{contracts}
}{
	It is very difficult to undo a {\contract} once it has come into effect, but with agreement of all the original {\signatories}, it can be done.
	A {\contract} can normally contain clauses that {\void} it under certain conditions.
	Now, when you create a {\contract}, you may add clauses that {\void} previous {\contracts}: either immediately, or under certain conditions.
	These conditions must still be detected by a {\signatory}, as usual.
	
	For this to work, every {\signatory} upon the {\contract} to be {\voided} must be a {\signatory} on the new {\contract}.
	Exactly which previous {\contracts} are to be {\voided} must be specified.
	You may also create a {\contract} without {\stipulations} or {\penalties}, with the sole purpose of {\voiding} earlier {\contracts}.
}

\feat{Dictate Terms}{contracts-collaborate}[headology]{5}{
	\skillrefspeciality[1]{lore}{Contracts},
	\featref{contracts}
}{
	Collaboratively writing a {\contract} usually requires passing the document around, and all physically writing upon it.
	This can be inconvenient if, for instance, one of the participants is inside a \featref{circle-contain}.
	As such, you heave learned to collaborate verbally.
	By dictating parts of the text of a {\contract} word-for-word, you may allow a {\contract}-writer with whom you are collaborating to use any feats you have to incorporate additional {\penalties} or clauses in the {\contract}.
	
	While all witches must learn to write a {\contract} independently before learning this ability, some creatures may have this ability without being able to write a {\contract} themselves. %TODO: Link to demons.
	In this case, note that whoever is actually penning the {\contract} must have the \featref{contracts} feat.
}

\feat{Taglock Escrow}{contracts-penalty-taglock}[headology]{15}{
	\featref{contracts}
}{
	The \materialref{blood} used to {\sign} a contract can always be used as a \materialref{taglock} by someone who holds the physical document.
	However, the {\contract} can also offer up use of this \materialref{taglock} as a {\contractpenalty}.
	You may add an additional {\contractpenalty} to {\contracts} you write.
	This offers the use of the {\penalised} {\signatorypossessive} \materialref{taglock} to one or more of the other {\signatories} for the {\penaltypossessive} duration.
	This does not provide them with a physical object, but they may perform magic as though they had a \materialref{taglock} from the {\penalised} {\signatory}.
	This is simply a generic \materialref{taglock}; it does not count as \materialref{blood} for any magic that requires that specifically.
	This still works even if the physical document and the \materialref{blood} used to {\sign} it have been destroyed.
}

\feat{Blood Escrow}{contracts-penalty-taglock-blood}[headology]{5}{
	\skillrefspeciality[2]{lore}{Contracts},
	\featref{contracts-penalty-taglock}
}{
	It was \materialref{blood} that was used to {\sign} a {\contract}, and with careful wording, you can ensure this is preserved as it is held in escrow.
	You may add an additional {\contractpenalty} to {\contracts} you write.
	This functions exactly as \featref{contracts-penalty-taglock}, except that the \materialref{taglock} offered may be used as though it were specifically \materialref{blood} for any magic that requires such (e.g.\ \featref{symlink-blood}).
	This still does not provide any particular amount of \materialref{blood} for any magic that requires such (e.g.\ \featref{symlink-blood-2}).
}

\feat{Liar's Location}{contracts-penalty-location}[headology]{5}{
	\featref{contracts-penalty-taglock},
	\featref{divination-taglock-location}
}{
	You may bind some \discref{divination} magic into a {\contract} to ensure that anyone who can breaks it can be tracked down.
	You may add an additional {\contractpenalty} to {\contracts} you write.
	This makes the location of {\penalised} {\signatory} known to one or more of the other {\signatories} for the {\penaltypossessive} duration.
	The information is provided as distance and direction to the {\penalised} {\signatory}, as by \featref{divination-taglock-location}.
}
